\documentclass{beamer}

\usepackage[english]{babel}
\usepackage[backend=biber,style=numeric-comp,sorting=none]{biblatex}
\addbibresource{../bibliography.bib}
\usepackage{csquotes}
\usepackage{lmodern}% http://ctan.org/pkg/lm
\usepackage{graphicx}
\usepackage{amsmath}
\usepackage{bm}
\usepackage{xcolor}
\usepackage{caption}
\DeclareMathOperator\arctanh{arctanh}
\DeclareMathOperator\ci{CI}
\DeclareMathOperator\power{power}
\newcommand{\E}{\mathrm{E}}
\newcommand{\Var}{\mathrm{Var}}
\newcommand{\Cov}{\mathrm{Cov}}

\definecolor{rgr}{RGB}{28,144,153}


% turn off mavigation 
\setbeamertemplate{navigation symbols}{}

\title{Power to detect a difference in correlations}
\subtitle{Applications for twin studies}
\author{Carl~Higgs\inst{1}}
\institute[Affiliation] % (optional)
{
  \inst{1}%
  Centre for Epidemiology and Biostatistics\\
  School of Population and Global Health\\
  University of Melbourne
}
\date[March 2018] % (optional)


\begin{document}
  \frame{\titlepage}
  
   \begin{frame}
       \frametitle{Outline}
       \tableofcontents[currentsection,currentsubsection]
   \end{frame}

   
  \section{Heritability and the classic twin study}
  
    \begin{frame}
    \frametitle{The classic twin study}
    The classic twin study exploits the differing degrees of genetic relatedness in identical twins and non-identical twins in order to draw inferences on the heritability of traits
    % In broad terms, heritability is the degree to which variation in a trait or phenotype, such as propensity to gain body weight, or become a centenarian, can be attributed to genetics.
  \end{frame} 
  
  \begin{frame}
    \frametitle{Heritability}
    Variation in traits or phenotypes attributable to genetics
    % In broad terms, heritability is the degree to which variation in a trait or phenotype, such as propensity to gain body weight, or become a centenarian, can be attributed to genetics.
  \end{frame}   
   
  
  \begin{frame}
    \frametitle{Twins and heritability}
    \begin{center}
      \includegraphics[width=\textwidth,height=0.8\textheight,keepaspectratio]{{../figs/twin_images/Identical-fraternal-sperm-egg_modified2}.pdf}
      \hspace*{15pt}\hbox{\scriptsize Credit: ChristinaT3 at English Wikipedia, CC3.0. Modified.}
    \end{center}
    % The word zygote refers to the fertilised cell arising from the union of a sperm and an egg.  Identical twins are referred to as monozygotic as they develop from one zygote, and most often also share a placenta and on rare occasions the amnotic sack. Non-identical or fraternal twins develop from two zygotes.  In short, identical twins share a copy of an identical genome, while non-identical twins don't.
  \end{frame}  
  
  \begin{frame}
    \frametitle{Twins and heritability}
    \begin{center}
      \includegraphics[width=\textwidth,height=0.8\textheight,keepaspectratio]{{../figs/twin_images/my_twin_drawing}.pdf}
    \end{center}
    % 
  \end{frame}  

  \begin{frame}
    \frametitle{Twins and heritability}
    \begin{center}
      \begin{figure}[!htb]
        \minipage{0.48\textwidth}
          \includegraphics[width=\linewidth]{{../figs/twin_images/my_twin_sample_mz}.pdf}
          \caption*{$\,$}
        \endminipage\hfill
        \minipage{0.48\textwidth}%
          \includegraphics[width=\linewidth]{{../figs/twin_images/my_twin_sample_dz}.pdf}
          \caption*{$\,$}
        \endminipage
        \end{figure}
    \end{center}
    % Through comparison of correlation in traits between samples of Mz and Dz twin pairs we can infer contribution of genetics to variation in the phenotype expression (Table XXX), under certain assumptions.  
     $\,$        
  \end{frame} 
  
  \begin{frame}
    \frametitle{Twins and heritability}
    \begin{center}
      \begin{figure}[!htb]
        \minipage{0.48\textwidth}
          \includegraphics[width=\linewidth]{{../figs/twin_images/my_twin_sample_mz}.pdf}
          \caption*{$r(v_{mz1},v_{mz2}) = 1$}
        \endminipage\hfill
        \minipage{0.48\textwidth}%
          \includegraphics[width=\linewidth]{{../figs/twin_images/my_twin_sample_dz}.pdf}
          \caption*{$r(v_{dz1},v_{dz2}) = 0.5$}
        \endminipage
        \end{figure}
    Wholly determined by genetics
    \end{center}
  \end{frame}  
  
  \begin{frame}
    \frametitle{Twins and heritability}
    \begin{center}
      \begin{figure}[!htb]
        \minipage{0.48\textwidth}
          \includegraphics[width=\linewidth]{{../figs/twin_images/my_twin_sample_mz}.pdf}
          \caption*{$r(v_{mz1},v_{mz2}) = 1$}
        \endminipage\hfill
        \minipage{0.48\textwidth}%
          \includegraphics[width=\linewidth]{{../figs/twin_images/my_twin_sample_dz}.pdf}
          \caption*{$r(v_{dz1},v_{dz2}) = 1$}
        \endminipage
        \end{figure}
    Wholly determined by shared environment
    \end{center}
  \end{frame} 
  
  \begin{frame}
    \frametitle{Twins and heritability}
    \begin{center}
      \begin{figure}[!htb]
        \minipage{0.48\textwidth}
          \includegraphics[width=\linewidth]{{../figs/twin_images/my_twin_sample_mz}.pdf}
          \caption*{$r(v_{mz1},v_{mz2}) = 0$}
        \endminipage\hfill
        \minipage{0.48\textwidth}%
          \includegraphics[width=\linewidth]{{../figs/twin_images/my_twin_sample_dz}.pdf}
          \caption*{$r(v_{dz1},v_{dz2}) = 0$}
        \endminipage
        \end{figure}
    Wholly determined by individual environment
    \end{center}
  \end{frame} 
  
  \begin{frame}
    \frametitle{Twins and heritability}
    \begin{center}
      \begin{figure}[!htb]
        \minipage{0.48\textwidth}
          \includegraphics[width=\linewidth]{{../figs/twin_images/my_twin_sample_mz}.pdf}
          \caption*{$0 < r(v_{mz1},v_{mz2}) < 1$}
        \endminipage\hfill
        \minipage{0.48\textwidth}%
          \includegraphics[width=\linewidth]{{../figs/twin_images/my_twin_sample_dz}.pdf}
          \caption*{$0 < r(v_{dz1},v_{dz2}) < 1$}
        \endminipage
        \end{figure}
    Determined by a combination of factors
    \end{center}
  \end{frame} 
 
  \begin{frame}
    \frametitle{Twins and assumptions}
    We assume that...
    \begin{itemize}
      \item the mz and dz twins have been sourced from populations with equivalent environments
      \item  Being an identical twin does not influence one's treatment within and interaction with the environment
      \item trait is normally distributed
      \item equal variance for both Mz and Dz twins
      \item accurate classification of zygosity
      \item accurate measurement of trait
      \item No dominant genetic variance (where one allele reliably masks the expression of the other when the two co-occur at a particular locus)
    \end{itemize}
  \end{frame} 

  \begin{frame}
    \frametitle{Twins and assumptions}
    Assumptions are not always met... (misclassification of zygosity)
    \begin{center}
       \includegraphics[width=\linewidth]{{../figs/twin_images/hospital_error}.pdf}
    \end{center}
  \end{frame} 
 

  \begin{frame}
    \frametitle{Research question}
      \begin{quote}
      A recent publication described a method of calculating the power to detect each of these variance components {\normalfont[additive, dominant, and shared environmental]} under certain assumptions.
      \end{quote}
      \begin{quote}
      However, this method did not address \textcolor{rgr}{power to detect differences in correlations between identical and non-identical twins}, which is an important first step in fitting variance components models. 
      \end{quote}
      \begin{quote}
      \textcolor{rgr}{This project will develop methods for estimating power for this first step, using both theory and simulations}.
      \end{quote}
  \end{frame} 
 
  
  \section{Bivariate distributions} 
  \begin{frame}
    \frametitle{Bivariate normal}
    \[\big(v_1 , v_2\big) \sim \mathcal{N}\bigg(\bm{\mu},\,\Sigma \bigg) \]
    \begin{center}
      % \includegraphics{../figs/r_to_z.pdf}
    \begin{figure}[!htb]
    \minipage{0.49\textwidth}
      \includegraphics[width=\linewidth]{{../figs/bnormal_0_1_n1000_r0.5}.pdf}
    \endminipage\hfill
    \minipage{0.24\textwidth}
      \includegraphics[width=\linewidth]{{../figs/twin_images/my_twin_sample_mz1}.pdf}
      \caption*{$V1_i$}
    \endminipage\hfill
    \minipage{0.24\textwidth}%
      \includegraphics[width=\linewidth]{{../figs/twin_images/my_twin_sample_mz2}.pdf}
      \caption*{$V2_i$}
    \endminipage
    \end{figure}
    \end{center}
  \end{frame}  
  
  \begin{frame}
    \frametitle{Bivariate normal}
    \[\big(v_1 , v_2\big) \sim \mathcal{N}\bigg(\left[\begin{smallmatrix}0\\ 0\end{smallmatrix}\right],\, \left[\begin{smallmatrix}1 & \textcolor{rgr}{.5}\\ \textcolor{rgr}{.5} & 1\end{smallmatrix}\right] \bigg) \]
    \begin{center}
    \begin{figure}[!htb]
    \minipage{0.32\textwidth}
      \includegraphics[width=\linewidth]{{../figs/bnormal_0_1_n1000_r0.5}.pdf}
    \endminipage\hfill
    \minipage{0.32\textwidth}
      \includegraphics[width=\linewidth]{{../figs/bnormal_0_1_n90_r0.5}.pdf}
    \endminipage\hfill
    \minipage{0.32\textwidth}%
      \includegraphics[width=\linewidth]{{../figs/bnormal_0_1_n20_r0.5}.pdf}
    \endminipage
    \end{figure}
    \end{center}
  \end{frame}     

  \begin{frame}
    \frametitle{Bivariate normal}
    \[\big(v_1 , v_2\big) \sim \mathcal{N}\bigg(\left[\begin{smallmatrix}0\\ 0\end{smallmatrix}\right],\, \left[\begin{smallmatrix}1 & \textcolor{rgr}{.1}\\ \textcolor{rgr}{.1} & 1\end{smallmatrix}\right] \bigg) \]
    \begin{center}
    \begin{figure}[!htb]
    \minipage{0.32\textwidth}
      \includegraphics[width=\linewidth]{{../figs/bnormal_0_1_n1000_r0.1}.pdf}
    \endminipage\hfill
    \minipage{0.32\textwidth}
      \includegraphics[width=\linewidth]{{../figs/bnormal_0_1_n90_r0.1}.pdf}
    \endminipage\hfill
    \minipage{0.32\textwidth}%
      \includegraphics[width=\linewidth]{{../figs/bnormal_0_1_n20_r0.1}.pdf}
    \endminipage
    \end{figure}
    \end{center}
  \end{frame}     

  \begin{frame}
    \frametitle{Bivariate normal}
    \[\big(v_1 , v_2\big) \sim \mathcal{N}\bigg(\left[\begin{smallmatrix}0\\ 0\end{smallmatrix}\right],\, \left[\begin{smallmatrix}1 & \textcolor{rgr}{-.8}\\ \textcolor{rgr}{-.8} & 1\end{smallmatrix}\right] \bigg) \]
    \begin{center}
    \begin{figure}[!htb]
    \minipage{0.32\textwidth}
      \includegraphics[width=\linewidth]{{../figs/bnormal_0_1_n1000_r-0.8}.pdf}
    \endminipage\hfill
    \minipage{0.32\textwidth}
      \includegraphics[width=\linewidth]{{../figs/bnormal_0_1_n90_r-0.8}.pdf}
    \endminipage\hfill
    \minipage{0.32\textwidth}%
      \includegraphics[width=\linewidth]{{../figs/bnormal_0_1_n20_r-0.8}.pdf}
    \endminipage
    \end{figure}
    \end{center}
  \end{frame}        

  \begin{frame}
    \frametitle{Bivariate and not so normal... e.g., $\bm{v} \sim Gamma(\bm{\mu},\bm{\phi})$}
    \[\big(v_1 , v_2\big) \sim \mathcal{G}\bigg(\left[\begin{smallmatrix}1.5\\ 1.5\end{smallmatrix}\right],\, \left[\begin{smallmatrix}0.09 & \textcolor{rgr}{-.8}\\ \textcolor{rgr}{-.8} & 0.09\end{smallmatrix}\right] \bigg) \]
    \begin{center}
    \begin{figure}[!htb]
    \minipage{0.32\textwidth}
      \includegraphics[width=\linewidth]{{../figs/bgamma_1.5_0.09_n1000_r-0.8}.pdf}
    \endminipage\hfill
    \minipage{0.32\textwidth}
      \includegraphics[width=\linewidth]{{../figs/bgamma_1.5_0.09_n90_r-0.8}.pdf}
    \endminipage\hfill
    \minipage{0.32\textwidth}%
      \includegraphics[width=\linewidth]{{../figs/bgamma_1.5_0.09_n20_r-0.8}.pdf}
    \endminipage
    \end{figure}
    \end{center}
  \end{frame}   

  \begin{frame}
    \frametitle{Bivariate and not so normal... e.g., $\bm{v} \sim Gamma(\bm{\mu},\bm{\phi})$}
    \[\big(v_1 , v_2\big) \sim \mathcal{G}\bigg(\left[\begin{smallmatrix}1\\ 1\end{smallmatrix}\right],\, \left[\begin{smallmatrix}5 & \textcolor{rgr}{-.8}\\ \textcolor{rgr}{-.8} & 5\end{smallmatrix}\right] \bigg) \]
    \begin{center}
    \begin{figure}[!htb]
    \minipage{0.32\textwidth}
      \includegraphics[width=\linewidth]{{../figs/bgamma_1_5_n1000_r-0.8}.pdf}
    \endminipage\hfill
    \minipage{0.32\textwidth}
      \includegraphics[width=\linewidth]{{../figs/bgamma_1_5_n90_r-0.8}.pdf}
    \endminipage\hfill
    \minipage{0.32\textwidth}%
      \includegraphics[width=\linewidth]{{../figs/bgamma_1_5_n20_r-0.8}.pdf}
    \endminipage
    \end{figure}
    \end{center}
  \end{frame}   
   
  
  \section{Correlation}

  \begin{frame}
    \frametitle{Pearson correlation coefficient}
    The population correlation, an index of linear change in y as x varies is formed of the (assumed) bivariate normal distribution of the respective variables \footfullcite{David1938} \\
    \[\rho = \frac{\Cov(x,y)}{\sigma_x \sigma_y} \]
    and is estimated by \(r\)
  \end{frame}  
  
  \begin{frame}
    \frametitle{Pearson correlation coefficient}
    \[r = \frac{\sum_{i=1}^{n} (x_i - \bar{x})(y_i - \bar{y})}{\sqrt{\sum_{i=1}^{n} (x_i - \bar{x})^2 \sum_{i=1}^{n} (y_i - \bar{y})^2}}\]
  \end{frame}

  \begin{frame}
    \frametitle{Spearman correlation coefficient}
    The Spearman correlation coefficient is calculated using the rank ordered variables for each group; as such it is robust to non-normality.  Subsequent steps are as per the Pearons correlation coefficient.
  \end{frame}   
  \subsection{Map \it{r} to \it{z}} 
  \begin{frame}
    \frametitle{Map \(r\) from \((-1,+1)\) to \((-\infty,+\infty)\) as Fisher's \(z\) \footfullcite{Fisher1915}}
    \[z     = \arctanh(r) = \frac{1}{2} \log_e \frac{1+r}{1-r} \]
    \begin{center}
      \includegraphics{../figs/r_to_z.pdf}
    \end{center}
  \end{frame}

  \subsection{Type 1 and Type 2 error}
  \begin{frame}
    \frametitle{Type 1 and Type 2 error \footfullcite{Cohen1988}}  
    \(\alpha\)
    \begin{itemize}
      \item expected proportion of null hypotheses to be rejected when true 
      \item type 1 error
      \item the classic '0.05' (5\%)\; but 0.1 or any other number may also be chosen 
    \end{itemize}
    
    \(\beta\)
    \begin{itemize}
      \item expected proportion of null hypotheses not rejected when false 
      \item type 2 error
      \item 0.2 (20\%) is a classic choice; gunning for power of \(1 - \beta = 80\)
    \end{itemize}
  \end{frame} 

  \subsection{Hypothesis test for difference in \it{r}}  
  \begin{frame}
    \frametitle{Hypothesis test for difference in correlations  \footfullcite{Cohen1988, Aberson2010}}
    \[\theta     = \arctanh(r_1) - \arctanh(r_2)          \]
    \[se_\theta  = \sqrt{\frac{1}{n_1-3}+\frac{1}{n_2-3}} \]
    \[c_\theta   = q / se_q                               \] 
    Under the null hypothesis, we evaluate \(c_\theta\) against the standard normal distribution for the probability of observing an effect size of such mangitude:
    \[p(\theta)  =  \Phi_{c_\theta/2}^{-1}                \]
      - note: two sided p value; could be onesided
  \end{frame}   

  \subsection{Confidence interval for difference in \it{r}}   
  \begin{frame}
    \frametitle{Confidence interval for difference in correlations}  
    In the scale of \(z\):
    \[\ci_{100(1-\alpha)\%} =\theta \pm c_0 \times se_\theta \] 
    or back in the scale of \(r\):
    \[\ci_{100(1-\alpha)\%} =\tanh \bigg( \theta \pm c_0 \times se_\theta \bigg) \] 
  \end{frame}    
  
  \subsection{Power for difference in correlations}   
  \begin{frame}
    \frametitle{Power for difference in correlations \footfullcite{Cohen1988, Aberson2010}}  
    \[\theta     = \arctanh(r_1) - \arctanh(r_2)            \]
    \[se_\theta    = \sqrt{\frac{1}{n_1-3}+\frac{1}{n_2-3}} \]
    \[c_\theta     = q / se_q                               \] 
    \[c_0   =  \Phi_{\alpha/2}^{-1}                         \]
    \[\beta = \Phi \bigg( c_0 - c_\theta \bigg)             \]
    and
    \[ \power(\theta) = 1 - \beta                           \]
  \end{frame}
  
  \begin{frame}
    \frametitle{Power for difference in correlations}  
    \[ \power = 1- \Phi \bigg(\Phi_{\alpha/2}^{-1} - \frac{\arctanh(r_1) - \arctanh(r_2)}{\sqrt{(n_1-3)^{-1} + (n_2-3)^{-1}}} \bigg) \]
  \end{frame}

  \begin{frame}
    \frametitle{Judging magnitude}  
      \begin{itemize}         
				\item \(r^2\) is the proportion of variance of one variable which can be explained by that of the other.
	      \item \(r^2 \times 100\%\) of the variance in \(y\) is attributable to magnitude of \(x\)    
 	      \item \(\tanh(\theta)^2 \times 100\%\) is the magnitude of difference in variance explained by one group compared with the other.  Although, by squaring we lose the sign indicating direction of effect; this could be restored.
        \item Cohen \footfullcite{Cohen1988} recommends use of \(r^2\) to inform choice of effect size for detection in power calculations (with subject matter knowledge from literature).
      \end{itemize}
  \end{frame}
  
 
  
  % \section{Other approaches to NHST}
  \subsection{Permutation test}
  \begin{frame}
    \frametitle{Permutation test \footfullcite{Efron1993}}  
    \begin{itemize}
		  \item take order statistic (ranked, no ties) representation of combined group correlations \(r_1\) and \(r_2\) 
	  	\item break ties using a Bernoulli trial
	  	\item two vectors:
      \begin{itemize}
				\item \(v\) is vector of order statistics: \(v = \{v_1, v_2,\ldots, v_N\}\)
			  \item \(v\) is corresponding ordered vector of group membership: \(g = \{g_1, g_2,\ldots, g_N\}\)
      \end{itemize}
    \end{itemize}
  \end{frame} 
  
  \begin{frame}
    \frametitle{Permutation test \footfullcite{Efron1993}}  
    \begin{itemize}
	  	\item Permutation lemma: \it{"Under \(H_0: \rho_1 = \rho_2\), the vector g has probability \(1/\binom{N}{n}\) of equaling any one of its possible values"}
      \item so, assuming \(H_0\) of no difference, all permuations of \(z_1\) and \(z_2\) are equally likely
  		\item combine \(n_1 + n_2\) observations from two groups together
      \begin{itemize}
        \item reduces the two sample situation to a single distribution assumed true under \(H_0\).
	    	\item if no difference, should be no discernible pattern of this difference in distributions when re-sampled a sufficiently large number of times
      \end{itemize}
    \end{itemize}
  \end{frame} 
      
  \begin{frame}
    \frametitle{Permutation test \footfullcite{Efron1993}}  
    \begin{itemize}
  		\item without replacement, take sample of size \(n_1\) to represent first group, 
  		\item remaining sample of size \(n_2\) represents second group
	  	\item take difference in means
		  \item repeat a large number of times
		  \item Evaluate: \it{does the original difference lie outside the middle \(100\times(1-\alpha)\)\%  of the re-sampled distribution?} If yes, reject \(H_0\). 
  		\item Permutation \(\alpha\) is probability that the permutation replication \(\hat{\theta}^{\*} \geq \hat{\theta}\) the sample difference, and is evaluated as the proportion of occurances relative to total number of possible permutations
		\item often approximated using Monte Carlo methods
    \end{itemize}
  \end{frame}  

  \subsection{Approach taken by R package cocor} 
  \begin{frame}
    \frametitle{Approach taken by R package cocor}
    cocor \footfullcite{Diedenhofen2015} is a recent implementation of a flexible calculator for inferences on differences in \(r\)
    \begin{center}
      \includegraphics[width=\textwidth,height=0.8\textheight,keepaspectratio]{../figs/Diedenfohen_Musch_cocor_flowchart_2015.pdf}
    \end{center}
  \end{frame}  

  \section{Interactive power calculator web app} 
  \begin{frame}
    \frametitle{Interactive power calculator web app}
    \begin{center}
      \includegraphics[width=\textwidth,height=0.8\textheight,keepaspectratio]{../figs/my_power_calc.pdf}
    \end{center}
  \end{frame}    
  
  \begin{frame}[allowframebreaks]
    \frametitle{Bibliography}
    \printbibliography
  \end{frame}
  
\end{document}