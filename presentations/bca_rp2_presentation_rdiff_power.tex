\documentclass{beamer}

\usepackage[english]{babel}
\usepackage[backend=biber,style=numeric-comp,sorting=none]{biblatex}
\addbibresource{../bibliography.bib}
\usepackage{csquotes}
\usepackage{lmodern}% http://ctan.org/pkg/lm
\usepackage{graphicx}
\usepackage{amsmath}
\usepackage{bm}
\usepackage{xcolor}
\usepackage{caption}
\DeclareMathOperator\arctanh{arctanh}
\DeclareMathOperator\ci{CI}
\DeclareMathOperator\power{power}
\newcommand{\E}{\mathrm{E}}
\newcommand{\Var}{\mathrm{Var}}
\newcommand{\Cov}{\mathrm{Cov}}

\definecolor{rgr}{RGB}{28,144,153}


% turn off mavigation 
\setbeamertemplate{navigation symbols}{}

\title{Power to detect a difference in correlations}
\subtitle{Applications for twin studies}
\author{Carl~Higgs\inst{1}}
\institute[Affiliation] % (optional)
{
  \inst{1}%
  Centre for Epidemiology and Biostatistics\\
  School of Population and Global Health\\
  University of Melbourne
}
\date[March 2018] % (optional)


\begin{document}
  \frame{\titlepage}
  
   \begin{frame}
       \frametitle{Outline}
       \tableofcontents[currentsection,currentsubsection]
   \end{frame}

   
  \section{Heritability and the classic twin study}
  
    \begin{frame}
    \frametitle{The classic twin study}
    The classic twin study exploits the differing degrees of genetic relatedness in identical twins and non-identical twins in order to draw inferences on the heritability of traits
    % In broad terms, heritability is the degree to which variation in a trait or phenotype, such as propensity to gain body weight, or become a centenarian, can be attributed to genetics.
  \end{frame} 
  
  \begin{frame}
    \frametitle{Heritability}
    Variation in traits or phenotypes attributable to genetics
    % In broad terms, heritability is the degree to which variation in a trait or phenotype, such as propensity to gain body weight, or become a centenarian, can be attributed to genetics.
  \end{frame}   
   
  
  \begin{frame}
    \frametitle{Twins and heritability}
    \begin{center}
      \includegraphics[width=\textwidth,height=0.8\textheight,keepaspectratio]{{../figs/twin_images/Identical-fraternal-sperm-egg_modified2}.pdf}
      \hspace*{15pt}\hbox{\scriptsize Credit: ChristinaT3 at English Wikipedia, CC3.0. Modified.}
    \end{center}
    % The word zygote refers to the fertilised cell arising from the union of a sperm and an egg.  Identical twins are referred to as monozygotic as they develop from one zygote, and most often also share a placenta and on rare occasions the amnotic sack. Non-identical or fraternal twins develop from two zygotes.  In short, identical twins share a copy of an identical genome, while non-identical twins don't.
  \end{frame}  
  
  \begin{frame}
    \frametitle{Twins and heritability}
    \begin{center}
      \includegraphics[width=\textwidth,height=0.8\textheight,keepaspectratio]{{../figs/twin_images/my_twin_drawing}.pdf}
    \end{center}
    % 
  \end{frame}  

  \begin{frame}
    \frametitle{Twins and heritability}
    \begin{center}
      \begin{figure}[!htb]
        \minipage{0.48\textwidth}
          \includegraphics[width=\linewidth]{{../figs/twin_images/my_twin_sample_mz}.pdf}
          \caption*{$\,$}
        \endminipage\hfill
        \minipage{0.48\textwidth}%
          \includegraphics[width=\linewidth]{{../figs/twin_images/my_twin_sample_dz}.pdf}
          \caption*{$\,$}
        \endminipage
        \end{figure}
    \end{center}
    % Through comparison of correlation in traits between samples of Mz and Dz twin pairs we can infer contribution of genetics to variation in the phenotype expression (Table XXX), under certain assumptions.  
     $\,$        
  \end{frame} 
  
  \begin{frame}
    \frametitle{Twins and heritability}
    \begin{center}
      \begin{figure}[!htb]
        \minipage{0.48\textwidth}
          \includegraphics[width=\linewidth]{{../figs/twin_images/my_twin_sample_mz}.pdf}
          \caption*{$r(v_{mz1},v_{mz2}) = 1$}
        \endminipage\hfill
        \minipage{0.48\textwidth}%
          \includegraphics[width=\linewidth]{{../figs/twin_images/my_twin_sample_dz}.pdf}
          \caption*{$r(v_{dz1},v_{dz2}) = 0.5$}
        \endminipage
        \end{figure}
    Wholly determined by genetics
    \end{center}
  \end{frame}  
  
  \begin{frame}
    \frametitle{Twins and heritability}
    \begin{center}
      \begin{figure}[!htb]
        \minipage{0.48\textwidth}
          \includegraphics[width=\linewidth]{{../figs/twin_images/my_twin_sample_mz}.pdf}
          \caption*{$r(v_{mz1},v_{mz2}) = 1$}
        \endminipage\hfill
        \minipage{0.48\textwidth}%
          \includegraphics[width=\linewidth]{{../figs/twin_images/my_twin_sample_dz}.pdf}
          \caption*{$r(v_{dz1},v_{dz2}) = 1$}
        \endminipage
        \end{figure}
    Wholly determined by shared environment
    \end{center}
  \end{frame} 
  
  \begin{frame}
    \frametitle{Twins and heritability}
    \begin{center}
      \begin{figure}[!htb]
        \minipage{0.48\textwidth}
          \includegraphics[width=\linewidth]{{../figs/twin_images/my_twin_sample_mz}.pdf}
          \caption*{$r(v_{mz1},v_{mz2}) = 0$}
        \endminipage\hfill
        \minipage{0.48\textwidth}%
          \includegraphics[width=\linewidth]{{../figs/twin_images/my_twin_sample_dz}.pdf}
          \caption*{$r(v_{dz1},v_{dz2}) = 0$}
        \endminipage
        \end{figure}
    Wholly determined by individual environment
    \end{center}
  \end{frame} 
  
  \begin{frame}
    \frametitle{Twins and heritability}
    \begin{center}
      \begin{figure}[!htb]
        \minipage{0.48\textwidth}
          \includegraphics[width=\linewidth]{{../figs/twin_images/my_twin_sample_mz}.pdf}
          \caption*{$0 < r(v_{mz1},v_{mz2}) < 1$}
        \endminipage\hfill
        \minipage{0.48\textwidth}%
          \includegraphics[width=\linewidth]{{../figs/twin_images/my_twin_sample_dz}.pdf}
          \caption*{$0 < r(v_{dz1},v_{dz2}) < 1$}
        \endminipage
        \end{figure}
    Determined by a combination of factors
    \end{center}
  \end{frame} 
 
  \begin{frame}
    \frametitle{Twins and assumptions}
    We assume that...
    \begin{itemize}
      \item the mz and dz twins have been sourced from populations with equivalent environments
      \item  Being an identical twin does not influence one's treatment within and interaction with the environment
      \item trait is normally distributed
      \item equal variance for both Mz and Dz twins
      \item accurate classification of zygosity
      \item accurate measurement of trait
      \item No dominant genetic variance (where one allele reliably masks the expression of the other when the two co-occur at a particular locus)
    \end{itemize}
  \end{frame} 

  \begin{frame}
    \frametitle{Twins and assumptions}
    Assumptions are not always met... (misclassification of zygosity)
    \begin{center}
       \includegraphics[width=\linewidth]{{../figs/twin_images/hospital_error}.pdf}
    \end{center}
  \end{frame} 
 

  \begin{frame}
    \frametitle{Research question}
      \begin{quote}
      A recent publication described a method of calculating the power to detect each of these variance components {\normalfont[additive, dominant, and shared environmental]} under certain assumptions.
      \end{quote}
      \begin{quote}
      However, this method did not address \textcolor{rgr}{power to detect differences in correlations between identical and non-identical twins}, which is an important first step in fitting variance components models. 
      \end{quote}
      \begin{quote}
      \textcolor{rgr}{This project will develop methods for estimating power for this first step, using both theory and simulations}.
      \end{quote}
  \end{frame} 
 
  
  \section{Bivariate distributions} 
  \begin{frame}
    \frametitle{Bivariate normal}
    \[\big(v_1 , v_2\big) \sim \mathcal{N}\bigg(\bm{\mu},\,\Sigma \bigg) \]
    \begin{center}
      % \includegraphics{../figs/r_to_z.pdf}
    \begin{figure}[!htb]
    \minipage{0.49\textwidth}
      \includegraphics[width=\linewidth]{{../figs/bnormal_0_1_n1000_r0.5}.pdf}
    \endminipage\hfill
    \minipage{0.24\textwidth}
      \includegraphics[width=\linewidth]{{../figs/twin_images/my_twin_sample_mz1}.pdf}
      \caption*{$V1_i$}
    \endminipage\hfill
    \minipage{0.24\textwidth}%
      \includegraphics[width=\linewidth]{{../figs/twin_images/my_twin_sample_mz2}.pdf}
      \caption*{$V2_i$}
    \endminipage
    \end{figure}
    \end{center}
  \end{frame}  
  
  \begin{frame}
    \frametitle{Bivariate normal}
    \[\big(v_1 , v_2\big) \sim \mathcal{N}\bigg(\left[\begin{smallmatrix}0\\ 0\end{smallmatrix}\right],\, \left[\begin{smallmatrix}1 & \textcolor{rgr}{.5}\\ \textcolor{rgr}{.5} & 1\end{smallmatrix}\right] \bigg) \]
    \begin{center}
    \begin{figure}[!htb]
    \minipage{0.32\textwidth}
      \includegraphics[width=\linewidth]{{../figs/bnormal_0_1_n1000_r0.5}.pdf}
    \endminipage\hfill
    \minipage{0.32\textwidth}
      \includegraphics[width=\linewidth]{{../figs/bnormal_0_1_n90_r0.5}.pdf}
    \endminipage\hfill
    \minipage{0.32\textwidth}%
      \includegraphics[width=\linewidth]{{../figs/bnormal_0_1_n20_r0.5}.pdf}
    \endminipage
    \end{figure}
    \end{center}
  \end{frame}     

  \begin{frame}
    \frametitle{Bivariate normal}
    \[\big(v_1 , v_2\big) \sim \mathcal{N}\bigg(\left[\begin{smallmatrix}0\\ 0\end{smallmatrix}\right],\, \left[\begin{smallmatrix}1 & \textcolor{rgr}{.1}\\ \textcolor{rgr}{.1} & 1\end{smallmatrix}\right] \bigg) \]
    \begin{center}
    \begin{figure}[!htb]
    \minipage{0.32\textwidth}
      \includegraphics[width=\linewidth]{{../figs/bnormal_0_1_n1000_r0.1}.pdf}
    \endminipage\hfill
    \minipage{0.32\textwidth}
      \includegraphics[width=\linewidth]{{../figs/bnormal_0_1_n90_r0.1}.pdf}
    \endminipage\hfill
    \minipage{0.32\textwidth}%
      \includegraphics[width=\linewidth]{{../figs/bnormal_0_1_n20_r0.1}.pdf}
    \endminipage
    \end{figure}
    \end{center}
  \end{frame}     

  \begin{frame}
    \frametitle{Bivariate normal}
    \[\big(v_1 , v_2\big) \sim \mathcal{N}\bigg(\left[\begin{smallmatrix}0\\ 0\end{smallmatrix}\right],\, \left[\begin{smallmatrix}1 & \textcolor{rgr}{-.8}\\ \textcolor{rgr}{-.8} & 1\end{smallmatrix}\right] \bigg) \]
    \begin{center}
    \begin{figure}[!htb]
    \minipage{0.32\textwidth}
      \includegraphics[width=\linewidth]{{../figs/bnormal_0_1_n1000_r-0.8}.pdf}
    \endminipage\hfill
    \minipage{0.32\textwidth}
      \includegraphics[width=\linewidth]{{../figs/bnormal_0_1_n90_r-0.8}.pdf}
    \endminipage\hfill
    \minipage{0.32\textwidth}%
      \includegraphics[width=\linewidth]{{../figs/bnormal_0_1_n20_r-0.8}.pdf}
    \endminipage
    \end{figure}
    \end{center}
  \end{frame}        

  \begin{frame}
    \frametitle{Bivariate and not so normal... e.g., $\bm{v} \sim Gamma(\bm{\mu},\bm{\phi})$}
    \[\big(v_1 , v_2\big) \sim \mathcal{G}\bigg(\left[\begin{smallmatrix}1.5\\ 1.5\end{smallmatrix}\right],\, \left[\begin{smallmatrix}0.09 & \textcolor{rgr}{-.8}\\ \textcolor{rgr}{-.8} & 0.09\end{smallmatrix}\right] \bigg) \]
    \begin{center}
    \begin{figure}[!htb]
    \minipage{0.32\textwidth}
      \includegraphics[width=\linewidth]{{../figs/bgamma_1.5_0.09_n1000_r-0.8}.pdf}
    \endminipage\hfill
    \minipage{0.32\textwidth}
      \includegraphics[width=\linewidth]{{../figs/bgamma_1.5_0.09_n90_r-0.8}.pdf}
    \endminipage\hfill
    \minipage{0.32\textwidth}%
      \includegraphics[width=\linewidth]{{../figs/bgamma_1.5_0.09_n20_r-0.8}.pdf}
    \endminipage
    \end{figure}
    \end{center}
  \end{frame}   

  \begin{frame}
    \frametitle{Bivariate and not so normal... e.g., $\bm{v} \sim Gamma(\bm{\mu},\bm{\phi})$}
    \[\big(v_1 , v_2\big) \sim \mathcal{G}\bigg(\left[\begin{smallmatrix}1\\ 1\end{smallmatrix}\right],\, \left[\begin{smallmatrix}5 & \textcolor{rgr}{-.8}\\ \textcolor{rgr}{-.8} & 5\end{smallmatrix}\right] \bigg) \]
    \begin{center}
    \begin{figure}[!htb]
    \minipage{0.32\textwidth}
      \includegraphics[width=\linewidth]{{../figs/bgamma_1_5_n1000_r-0.8}.pdf}
    \endminipage\hfill
    \minipage{0.32\textwidth}
      \includegraphics[width=\linewidth]{{../figs/bgamma_1_5_n90_r-0.8}.pdf}
    \endminipage\hfill
    \minipage{0.32\textwidth}%
      \includegraphics[width=\linewidth]{{../figs/bgamma_1_5_n20_r-0.8}.pdf}
    \endminipage
    \end{figure}
    \end{center}
  \end{frame}   
   
  
  \section{Correlation}
  \subsection{Pearson}
  \subsection{Spearman}
  \subsection{ICC}
  \subsection{Partial}
  
  \section{Tests for difference in correlation}
  \subsection{Fisher's Z - formula}
  \subsection{Fisher's Z - simulation}
  \subsection{Zou's confidence interval}
  \subsection{Generalised Test Variable (GTV)}
  \subsection{Signed Log-likelihood ratio (SLR)}
  \subsection{Permutation test}
  
  \section{Power to detect a true difference}
  \subsection{Type 1 and Type 2 error}
  \subsection{formula}
  \subsection{simulation}
  
  \section{Simulation and efficiency}
  \subsection{Complexity: cats in sacks with mice and lice...}
  \subsection{Time testing}
  \subsection{Analysis plan}
  
  \section{results}

  \begin{frame}
    \frametitle{Remaining slides}

  Correlation
  Pearson
  Spearman
  ICC
  Partial
  
  Tests for difference in correlation
  Fisher's Z - formula
  Fisher's Z - simulation
  Zou's confidence interval
  Generalised Test Variable (GTV)
  Signed Log-likelihood ratio (SLR)
  Permutation test
  
  Power to detect a true difference
  Type 1 and Type 2 error
  formula
  simulation
  
  Simulation and efficiency
  Complexity: cats in sacks with mice and lice...
  Time testing
  Analysis plan
  \end{frame}    
  
  
  
  \section{Interactive power calculator web app} 
  \begin{frame}
    \frametitle{Interactive power calculator web app}
    \begin{center}
      \includegraphics[width=\textwidth,height=0.8\textheight,keepaspectratio]{../figs/my_power_calc.pdf}
    \end{center}
  \end{frame}    
  
  \begin{frame}[allowframebreaks]
    \frametitle{Bibliography}
    \printbibliography
  \end{frame}
  
\end{document}