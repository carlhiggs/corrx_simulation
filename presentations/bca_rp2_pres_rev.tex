\documentclass{beamer}

\usepackage[english]{babel}
\usepackage[backend=biber,style=numeric-comp,sorting=none]{biblatex}
\addbibresource{../bibliography.bib}
\usepackage{csquotes}
\usepackage{lmodern}% http://ctan.org/pkg/lm
\usepackage{graphicx}
\usepackage{amsmath}
\usepackage{bm}
\usepackage{xcolor}
\usepackage{caption}
\usepackage{booktabs}
\usepackage{tikz}
\DeclareMathOperator\arctanh{arctanh}
\DeclareMathOperator\ci{CI}
\DeclareMathOperator\abs{abs}
\DeclareMathOperator\power{power}
\newcommand{\E}{\mathrm{E}}
\newcommand{\Var}{\mathrm{Var}}
\newcommand{\Cov}{\mathrm{Cov}}

\definecolor{rgr}{RGB}{28,144,153}
% custom commands
\newcommand*{\lcd}{\raisebox{-0.25ex}{\scalebox{1.2}{$\cdot$}}}
% \newcommand{\bvn}[1]{\begin{equation*} \sim \mathcal{N}\bigg(\left[\begin{smallmatrix}0\\ 0\end{smallmatrix}\right],\, \left[\begin{smallmatrix}1 & {#1} \\ {#1} & 1\end{smallmatrix}\right] \bigg)\end{equation*}}
   
 % \newcommand{\bvg1}[1]{\begin{equation*} \sim \mathcal{Gamma}\bigg(\left[\begin{smallmatrix}1.5\\ 1.5\end{smallmatrix}\right],\, \left[\begin{smallmatrix}.09 & {#1} \\ {#1} & .09\end{smallmatrix}\right] \bigg)\end{equation*}}
       
   
 % \newcommand{\bvg2}[1]{\begin{equation*} \sim \mathcal{Gamma}\bigg(\left[\begin{smallmatrix}1\\ 1\end{smallmatrix}\right],\, \left[\begin{smallmatrix}5 & {#1} \\ {#1} & 5\end{smallmatrix}\right] \bigg)\end{equation*}}
       
% turn off mavigation 
\setbeamertemplate{navigation symbols}{}

\title{Power to detect a difference in correlations}
\subtitle{Applications for twin studies}
\author{Carl~Higgs\inst{1}}
\institute[Affiliation] % (optional)
{
  \inst{1}%
  Centre for Epidemiology and Biostatistics\\
  School of Population and Global Health\\
  University of Melbourne
}
\date[March 2018] % (optional)


\begin{document}

  \section{Research question}
  
  \begin{frame}
    \frametitle{Research question}
      % This should be the first overhead or slide of your presentation and should clearly state a specific question that you addressed during your research.
    Power to detect a difference in correlations between samples of identical (monozygotic) and non-identical (dizygotic) twin pairs
    % \begin{itemize}
      % \item What are the existing methods?
      % \item How do these compare and can they be improved upon?  
      % This question will be addressed through a literature review, and comparison of methods using both theory and simulation.
    % \end{itemize}
    \begin{center}
  \begin{figure}[!htb]
        \minipage{0.49\textwidth}
          \includegraphics[width=\linewidth]{{../figs/twin_images/my_twin_drawing_mz}.pdf}
        \endminipage\hfill
        \minipage{0.49\textwidth}%
          \includegraphics[width=\linewidth]{{../figs/twin_images/my_twin_drawing_dz}.pdf}
        \endminipage
    \end{figure}
    \end{center}
  \end{frame} 
  
  
  \section{Background}  
  \begin{frame}
    \frametitle{Background}
    Let's assume that...
    \begin{itemize}
      \item each pair of twin siblings share the same environmental history
      \item zygosity doesn't influence environmental interactions
      \item trait is normally distributed
      \item accurate classification of zygosity
      \item accurate measurement of trait
      \item No dominant genetic variance (where one allele reliably masks the expression of the other when the two co-occur at a particular locus)
      \item mz and dz twins have been sourced from equivalent populations
    \end{itemize}
  \end{frame} 
  
   % 2 slides:
  
  % This should convince the audience that you understand the issue (e.g. disease and/or risk factors) that you studied, the gaps in knowledge, and that your project is important. That is, why did you do the study?

  % You should discuss what is known about the public health issue and what is unknown? 

  % How will your proposed study address what is unknown and fill any gaps?

  % When applicable, it is important to describe findings and methodological problems from any previous similar studies and how your research will differ from previous studies.

  % \begin{quote}
  % A recent publication described a method of calculating the power to detect each of these variance components {\normalfont[additive, dominant, and shared environmental]} under certain assumptions.
  % \end{quote}
  % \begin{quote}
  % However, this method did not address \textcolor{rgr}{power to detect differences in correlations between identical and non-identical twins}, which is an important first step in fitting variance components models. 
  % \end{quote}
  % \begin{quote}
  % \textcolor{rgr}{This project will develop methods for estimating power for this first step, using both theory and simulations}.
  % \end{quote}
  
  \begin{frame}
    \frametitle{Background}
    Let $x_{zjk}$ be the value of trait $x$ in the $k$th member of the $j$th twin pair having zygosity $z\in\{\text{mz},\text{dz}\}$
    % cite Falconer 1960
    \begin{center}
      % \includegraphics{../figs/r_to_z.pdf}
    \begin{figure}[!htb]
        \minipage{0.39\textwidth}
           \[\big(x_{z \lcd 1} , x_{z \lcd 2}\big) \sim \mathcal{N}\big(\bm{\mu},\,\Sigma \big) \]
           \[r_z = \hat{\rho}_z = \frac{Cov(x_{z \lcd 1} , x_{z \lcd 2})}{\hat{\sigma}_{x_{z \lcd 1}} \hat{\sigma}_{x_{z \lcd 2}}} \]
           \[\hat{\delta} = r_{mz} - r_{dz}\]
           \[\text{heritability} = 2\delta\]
        \endminipage\hfill
        \minipage{0.3\textwidth}
          \includegraphics[width=\linewidth]{{../figs/twin_images/my_twin_drawing_mz}.pdf}
        \endminipage\hfill
        \minipage{0.3\textwidth}%
          \includegraphics[width=\linewidth]{{../figs/twin_images/my_twin_drawing_dz}.pdf}
        \endminipage
    \end{figure}
    \begin{table}\centering
    \begin{tabular}{rcc}
        \toprule
        \textbf{Trait etiology} &	\multicolumn{2}{c}{\textbf{Correlation in twin pairs}} \\
        \cmidrule(lr){2-3} 
         & \textbf{Monozygotic (mz)} & \textbf{Dizogotic (dz)}				\\
        \midrule										
        genetic         &$r_{mz} = 1$& $r_{dz} = 0.5$       \\
        shared env.     &$r_{mz} = 1$& $r_{dz} = 1$         \\  
        individual env. &$r_{mz} = 0$& $r_{dz} = 0$         \\  
        combination     &$0 < r_{mz} < 1$& $0 < r_{dz} < 1$ \\
         \bottomrule 
    \end{tabular}
    \end{table}
    \end{center}
  \end{frame}  


  \section{Methods}
  \begin{frame}
    \frametitle{A skip through the literature}
    % 3) HOW YOU ADDRESSED THE RESEARCH QUESTION (3 slides)
    % This should inform the audience as to what you did during your research to complete your project.
    % You should put a primary emphasis on explaining the statistical issues that you addressed in the project, with clear presentation of models and methods of analysis that you used.

    % bivariate data
    % tests
    % simulation code  Complexity and simulation time
    \begin{columns}[t]
      \column[T]{.4\textwidth}
        \begin{itemize}
          \item Galton, F.  1888
          \item Pearson, K.  1895
          \item Fisher, RA. 1915
          \item David, FN. 1938
          \item Falconer, DS. 1960
          \item Cohen, J.  1988
          \item Neale, M. et al. 1994
          \item Visscher, PM. 2004
          \item Verhulst, B. 2017
          % \item Efron B. & Tibshirani RJ. 1994
          % \item Kazemi M.R & Jafari 
        \end{itemize} 
      \column[T]{.6\textwidth}
          \includegraphics[width=\linewidth]{{../figs/david_corr_bands}.pdf}
    \end{columns}
  \end{frame}   


  \subsection{Power for difference in correlations}   
  \begin{frame}
    \frametitle{Power for difference in correlations \footfullcite{Cohen1988}}  
    \textbf{Frequentist NHST paradigm} \\
    $\alpha$ (Type I error) and $\beta$ (Type II error); Power is $1-\beta$
    \linebreak
    \linebreak
    \textbf{Fisher's $Z$ formula approach}
    \linebreak
    $ \power = 1- \Phi \Bigg(\Phi_{\alpha/2}^{-1} -  \abs\bigg(  \frac{\arctanh(r_{mz}) - \arctanh(r_{dz})}{\sqrt{(n_{mz}-3)^{-1} + (n_{dz}-3)^{-1}}}  \bigg)  \Bigg) $ 
    \linebreak
    \textbf{Simulation approach}
    \begin{itemize}
      \item implement hypothesis test(s) for difference in correlations
      \item take $m$ draws from simulated bivariate mz and dz populations
      \item run hypothesis test(s) on each pair of draws
      \item power given parameters is proportion of tests where $p < \alpha$      
    \end{itemize} 
  \end{frame}  
  
  \begin{frame}
    \frametitle{Simulations}
    Using $r_{mz}$ and $r_{dz}$ derived from simulated bivariate normal and non-normal populations with covariance $\left[\begin{smallmatrix}1 & \rho\\ \rho & 1\end{smallmatrix}\right] $, the following hypothesis tests were run:
    \linebreak
    \begin{itemize}
    \item Fisher's Z: $\frac{\arctanh(r_{mz}) - \arctanh(r_{dz})}{\sqrt{(n_{mz}-3)^{-1} + (n_{dz}-3)^{-1}}}$
    \item Generalized Variable Test (GVT) \footfullcite{Krishnamoorthy2014}
    \item Signed log-likelihood ratio test \footfullcite{Krishnamoorthy2014}  
    \item Permutation test \footfullcite{Efron1994}
    \item Zou's confidence interval \footfullcite{Zou2007}
    \end{itemize}
    \linebreak
    \linebreak
   \textbf{Plan} 
   1000 simulations of all mz dz combinations of $rho$ and $n$ using three distinct distributions and re-run using Spearman correlation (rank based, non-parametric)...
  \end{frame}   

  \section{results}
  \begin{frame}
    \frametitle{Preliminary results: simulation times}
    \linebreak
    $1000\times39^2\times7^2\times6\times3\times2 = 2,234,400,000$ simulations...
    \textbf{Optimisation approach}\footnotemark
    \begin{itemize}
    \item hand coded
    \item functions compiled using either JIT compiler or RCCP 
    \end{itemize}
    \begin{table}\centering
    \begin{tabular}{rcc}
        \toprule
        \textbf{Test} &	\multicolumn{2}{c}{\textbf{Time/1000 runs (secs)}} \\
        \cmidrule(lr){2-3} 
         & \textbf{as is} & \textbf{compiled}				\\
        \midrule										
        Fisher's z (no sim) &         0.03 &   0.02 \\
        Fisher's Z          &         0.56 &   0.36 \\
        GTV (r)             &        18.65 &  16.55 \\
        GTV (R C++)         &              &  12.08 \\
        Permutation (PT)    &      2473.18 &2341.62 \\
        SLR                 &         0.49 &   0.33 \\
        Zou's CI            &         0.38 &   0.28 \\
        \midrule
        \multicolumn{1}{l}{Total (sans r-compiled GTV)} & 2493.29  & 2354.69 \\
        \multicolumn{1}{l}{$\times 1520 \times 49 \times 3 \times 2/60/60/24/365$}  & 35 years  & 33 years \\
        \multicolumn{1}{l}{$-\text{PT}: \times 361 \times 49\times 3 \times 2 /60/60/24$}  & 25 days  & 16 days \\
    \bottomrule 
    \end{tabular}
    \end{table}

    
    
    \footnotetext[1]{Thanks to Koen Simons for advice}
    % 4) RESULTS (2-3 slides)
    % This should inform the audience as to what you found during your research to complete your project.
    % You need to present a summary of the analysis. This could include tables and figures of results and text, again with a primary focus on statistical issues, e.g. possible effects of decisions taken during the modelling and choice of estimation methods.
  \end{frame}   

  \begin{frame}
    \frametitle{Main findings}
    106,134 sets of power estimates comparing 5 tests (wide form)
    \linebreak
    \textbf{Example}
    \linebreak
    Power to detect $\hat\delta_{\rho}$ in Mz and Dz twins $\sim \mathcal{N}\bigg(\left[\begin{smallmatrix}0\\ 0\end{smallmatrix}\right],\, \left[\begin{smallmatrix}1 & {\rho} \\ {\rho} & 1\end{smallmatrix}\right] \bigg)$
    \begin{center}
      \begin{figure}[!htb]
        \minipage{0.49\textwidth}
          \includegraphics[width=\linewidth]{{../figs/corrxplot_slr_example}.pdf}
        \endminipage\hfill
        \minipage{0.49\textwidth}%
          \includegraphics[width=\linewidth]{{../figs/power_n_plot_example}.pdf}
        \endminipage
      \end{figure}
    \end{center}
  \end{frame}   

  \begin{frame}
    \frametitle{Main findings}
    106,134 sets of power estimates comparing 5 tests (wide form)
    \linebreak
    \textbf{Example}
    \linebreak
    Power to detect $\hat\delta_{\rho}$ in Mz and Dz twins $\sim \mathcal{G}\bigg(\left[\begin{smallmatrix}1\\ 1\end{smallmatrix}\right],\, \left[\begin{smallmatrix}5 & {\rho} \\ {\rho} & 5\end{smallmatrix}\right] \bigg)$
    \begin{center}
      \begin{figure}[!htb]
        \minipage{0.49\textwidth}
          \includegraphics[width=\linewidth]{{../figs/corrxplot_slr_example2}.pdf}
        \endminipage\hfill
        \minipage{0.49\textwidth}%
          \includegraphics[width=\linewidth]{{../figs/power_n_plot_example2}.pdf}
        \endminipage
      \end{figure}
    \end{center}
  \end{frame}     
  
  \begin{frame}
    \frametitle{Limitations}
   %  any limitations � concerning either the data and substantive question or the statistical methods.
   \begin{itemize}
     \item Analysis not yet complete
     \item 100 simulations for each parameter combinations
     \begin{itemize}
       \item 1000 is currently processing
     \end{itemize}
     \item ICC not currently implemented
     \item Partial correlations are important, and not yet covered
     \begin{itemize}
       \item incorporation of evaluation of correlation using mixed effects regression approaches could address above two issues
     \end{itemize}
     \item Included tests are not exhaustive
     \item Code could be more efficient, allowing higher resolution estimation
     \begin{itemize}
       \item e.g. original intent of -0.95 through 0.95 in 0.05 increments
     \end{itemize}
   \end{itemize}
  \end{frame}   
  
  \begin{frame}
    \frametitle{Sum up and next steps}
    \tikz\node[opacity=0.3] {\includegraphics[height=\paperheight,width=\paperwidth]{{../figs/webapp_example}.pdf}}

    We have developed an architecture for detecting differences in correlations and comparison of methods which is readily extensible for future applications (new tests, specific requests, new distributions, etc).
   %  This slide(s) should summarise in 2 or 3 sentences your overall findings. It should include a conclusion and possibly a recommendation for further research, 
  \end{frame}  
    
  % \begin{frame}[allowframebreaks]
    % \frametitle{Bibliography}
    % \printbibliography
  % \end{frame}
  
\end{document}