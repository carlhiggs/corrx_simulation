\documentclass{beamer}

\usepackage[english]{babel}
\usepackage[backend=biber,style=numeric-comp,sorting=none]{biblatex}
\addbibresource{../bibliography.bib}
\usepackage{csquotes}
\usepackage{lmodern}% http://ctan.org/pkg/lm
\usepackage{graphicx}
\usepackage{amsmath}
\usepackage{bm}
\usepackage{xcolor}
\usepackage{caption}
\DeclareMathOperator\arctanh{arctanh}
\DeclareMathOperator\ci{CI}
\DeclareMathOperator\power{power}
\newcommand{\E}{\mathrm{E}}
\newcommand{\Var}{\mathrm{Var}}
\newcommand{\Cov}{\mathrm{Cov}}

\definecolor{rgr}{RGB}{28,144,153}


% turn off mavigation 
\setbeamertemplate{navigation symbols}{}

\title{Power to detect a difference in correlations}
\subtitle{Applications for twin studies}
\author{Carl~Higgs\inst{1}}
\institute[Affiliation] % (optional)
{
  \inst{1}%
  Centre for Epidemiology and Biostatistics\\
  School of Population and Global Health\\
  University of Melbourne
}
\date[March 2018] % (optional)


\begin{document}

  \section{Research question}
  
  \begin{frame}
    \frametitle{Research question}
      % This should be the first overhead or slide of your presentation and should clearly state a specific question that you addressed during your research.
    \begin{itemize}
      \item What are the existing methods for estimating power to detect a difference in correlations between identical (monozygotic) and non-identical (dizygotic) twins?
      \item How do these compare and can they be improved upon?  
      \item This question will be addressed through a literature review, and comparison of methods using both theory and simulation.
    \end{itemize}
  \end{frame} 

  \section{Background}
  \begin{frame}
    \frametitle{Background: The classic twin study}
      % 2 slides:
      
      % This should convince the audience that you understand the issue (e.g. disease and/or risk factors) that you studied, the gaps in knowledge, and that your project is important. That is, why did you do the study?

      % You should discuss what is known about the public health issue and what is unknown? 

      % How will your proposed study address what is unknown and fill any gaps?

      % When applicable, it is important to describe findings and methodological problems from any previous similar studies and how your research will differ from previous studies.

      % \begin{quote}
      % A recent publication described a method of calculating the power to detect each of these variance components {\normalfont[additive, dominant, and shared environmental]} under certain assumptions.
      % \end{quote}
      % \begin{quote}
      % However, this method did not address \textcolor{rgr}{power to detect differences in correlations between identical and non-identical twins}, which is an important first step in fitting variance components models. 
      % \end{quote}
      % \begin{quote}
      % \textcolor{rgr}{This project will develop methods for estimating power for this first step, using both theory and simulations}.
      % \end{quote}
    The classic twin study exploits the differing degrees of genetic relatedness in identical twins and non-identical twins in order to draw inferences on the heritability of traits
    % In broad terms, heritability is the degree to which variation in a trait or phenotype, such as propensity to gain body weight, or become a centenarian, can be attributed to genetics.
  
    Verhulst title
  \end{frame} 

  \section{Methods}
  \begin{frame}
    \frametitle{Broad sketch}
    % 3) HOW YOU ADDRESSED THE RESEARCH QUESTION (3 slides)
    % This should inform the audience as to what you did during your research to complete your project.
    % You should put a primary emphasis on explaining the statistical issues that you addressed in the project, with clear presentation of models and methods of analysis that you used.

    % bivariate data
    % tests
    % simulation code  Complexity and simulation time
  \end{frame}   

  
  \begin{frame}
    \frametitle{Tests}
  \end{frame}   

  \section{results}
  \begin{frame}
    \frametitle{Main findings}
    % 4) RESULTS (2-3 slides)
    % This should inform the audience as to what you found during your research to complete your project.
    % You need to present a summary of the analysis. This could include tables and figures of results and text, again with a primary focus on statistical issues, e.g. possible effects of decisions taken during the modelling and choice of estimation methods.
  \end{frame}   

  \begin{frame}
    \frametitle{Main findings}
   % something else
  \end{frame}   

  \begin{frame}
    \frametitle{Limitations}
   %  any limitations � concerning either the data and substantive question or the statistical methods.
  \end{frame}   
  
  \begin{frame}
    \frametitle{Sum up and next steps}
   %  This slide(s) should summarise in 2 or 3 sentences your overall findings. It should include a conclusion and possibly a recommendation for further research, 
  \end{frame}  
    
  \begin{frame}[allowframebreaks]
    \frametitle{Bibliography}
    \printbibliography
  \end{frame}
  
\end{document}