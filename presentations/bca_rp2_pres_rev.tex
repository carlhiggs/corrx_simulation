\documentclass{beamer}

\usepackage[english]{babel}
\usepackage[backend=biber,style=numeric-comp,sorting=none]{biblatex}
\addbibresource{../bibliography.bib}
\usepackage{csquotes}
\usepackage{lmodern}% http://ctan.org/pkg/lm
\usepackage{graphicx}
\usepackage{amsmath}
\usepackage{bm}
\usepackage{xcolor}
\usepackage{caption}
\usepackage{booktabs}
\usepackage{tikz}
\DeclareMathOperator\arctanh{arctanh}
\DeclareMathOperator\ci{CI}
\DeclareMathOperator\abs{abs}
\DeclareMathOperator\power{power}
\newcommand{\E}{\mathrm{E}}
\newcommand{\Var}{\mathrm{Var}}
\newcommand{\Cov}{\mathrm{Cov}}

\definecolor{rgr}{RGB}{28,144,153}
% custom commands
\newcommand*{\lcd}{\raisebox{-0.25ex}{\scalebox{1.2}{$\cdot$}}}
% \newcommand{\bvn}[1]{\begin{equation*} \sim \mathcal{N}\bigg(\left[\begin{smallmatrix}0\\ 0\end{smallmatrix}\right],\, \left[\begin{smallmatrix}1 & {#1} \\ {#1} & 1\end{smallmatrix}\right] \bigg)\end{equation*}}
   
 % \newcommand{\bvg1}[1]{\begin{equation*} \sim \mathcal{Gamma}\bigg(\left[\begin{smallmatrix}1.5\\ 1.5\end{smallmatrix}\right],\, \left[\begin{smallmatrix}.09 & {#1} \\ {#1} & .09\end{smallmatrix}\right] \bigg)\end{equation*}}
       
   
 % \newcommand{\bvg2}[1]{\begin{equation*} \sim \mathcal{Gamma}\bigg(\left[\begin{smallmatrix}1\\ 1\end{smallmatrix}\right],\, \left[\begin{smallmatrix}5 & {#1} \\ {#1} & 5\end{smallmatrix}\right] \bigg)\end{equation*}}
       
% turn off mavigation 
\setbeamertemplate{navigation symbols}{}

\title{Power to detect a difference in correlations}
\subtitle{Applications for twin studies}
\author{Carl~Higgs\inst{}}
\institute[Affiliation] % (optional)
{
  \inst{}%
  Centre for Epidemiology and Biostatistics\\
  School of Population and Global Health\\
  University of Melbourne
}
\date{\\[2em]
{\footnotesize Supervisors: Enes Makalic, Elasma Milanzi and Katrina Scurrah}
}

\begin{document}
  \begin{frame}
  \titlepage
  \end{frame}

  \section{Research question}

  \begin{frame}
    \frametitle{}
      % This should be the first overhead or slide of your presentation and should clearly state a specific question that you addressed during your research.
      \begin{center}
    How much power do we have to detect a difference \\ in correlations between samples of \\ identical (monozygotic, MZ) and non-identical (dizygotic, DZ) \\ twin pairs?
    % \begin{itemize}
      % \item What are the existing methods?
      % \item How do these compare and can they be improved upon?  
      % This question will be addressed through a literature review, and comparison of methods using both theory and simulation.
    % \end{itemize}
   
  \begin{figure}[!htb]
        \minipage{0.49\textwidth}
          \includegraphics[width=\linewidth]{{../figs/twin_images/my_twin_drawing_mz}.pdf}
        \endminipage\hfill
        \minipage{0.49\textwidth}%
          \includegraphics[width=\linewidth]{{../figs/twin_images/my_twin_drawing_dz}.pdf}
        \endminipage
    \end{figure}
    \end{center}
  \end{frame} 
  
  
  % \section{Background}  
  % \begin{frame}
    % \frametitle{Background}
    % Let's assume that...
    % \begin{itemize}
      % \item each pair of twin siblings share the same environmental history
      % \item zygosity doesn't influence twin's social interactions
      % \item trait is normally distributed
      % \item accurate classification of zygosity
      % \item accurate measurement of trait
      % % if there is a difference it may indicate presence of dominance
      % \item MZ and DZ twins have been sourced from equivalent populations
    % \end{itemize}
  % \end{frame} 
  
   % 2 slides:
  
  % This should convince the audience that you understand the issue (e.g. disease and/or risk factors) that you studied, the gaps in knowledge, and that your project is important. That is, why did you do the study?

  % You should discuss what is known about the public health issue and what is unknown? 

  % How will your proposed study address what is unknown and fill any gaps?

  % When applicable, it is important to describe findings and methodological problems from any previous similar studies and how your research will differ from previous studies.

  % \begin{quote}
  % A recent publication described a method of calculating the power to detect each of these variance components {\normalfont[additive, dominant, and shared environmental]} under certain assumptions.
  % \end{quote}
  % \begin{quote}
  % However, this method did not address \textcolor{rgr}{power to detect differences in correlations between identical and non-identical twins}, which is an important first step in fitting variance components models. 
  % \end{quote}
  % \begin{quote}
  % \textcolor{rgr}{This project will develop methods for estimating power for this first step, using both theory and simulations}.
  % \end{quote}
  
  \begin{frame}
    \frametitle{}
    Let $x_{zjk}$ be the value of trait $x$ in the $k$th member of the $j$th twin pair having zygosity $z\in\{\text{MZ},\text{DZ}\}$
    % cite Falconer 1960
    \begin{center}
      % \includegraphics{../figs/r_to_z.pdf}
    \begin{figure}[!htb]
        \minipage{0.3\textwidth}
          \includegraphics[width=\linewidth]{{../figs/twin_images/my_twin_drawing_mz}.pdf}
        \endminipage\hfill
        \minipage{0.39\textwidth}
           \[\big(x_{z \lcd 1} , x_{z \lcd 2}\big) \sim \mathcal{N}\big(\bm{\mu},\,\Sigma \big) \]
           \[r_z = \hat{\rho}_z = \frac{Cov(x_{z \lcd 1} , x_{z \lcd 2})}{\hat{\sigma}_{x_{z \lcd 1}} \hat{\sigma}_{x_{z \lcd 2}}} \]
           \[\hat{\delta_r} = r_{MZ} - r_{DZ}\]
           \[\text{heritability} = 2\delta_{\rho}\]
        \endminipage\hfill
        \minipage{0.3\textwidth}%
          \includegraphics[width=\linewidth]{{../figs/twin_images/my_twin_drawing_dz}.pdf}
        \endminipage
    \end{figure}
    \begin{table}\centering
    \begin{tabular}{rcc}
        \toprule
        \textbf{Trait etiology} &	\multicolumn{2}{c}{\textbf{Correlation in twin pairs}} \\
        \cmidrule(lr){2-3} 
         & \textbf{Monozygotic (MZ)} & \textbf{Dizogotic (DZ)}				\\
        \midrule										
        genetic         &$r_{MZ} = 1$& $r_{DZ} = 0.5$       \\
        shared env.     &$r_{MZ} = 1$& $r_{DZ} = 1$         \\  
        individual env. &$r_{MZ} = 0$& $r_{DZ} = 0$         \\  
        combination     &$0 < r_{MZ} < 1$& $0 < r_{DZ} < 1$ \\
         \bottomrule 
    \end{tabular}
    \end{table}
    \end{center}
  \end{frame}  


  \section{Methods}
  \begin{frame}
    \frametitle{Plan}
    % 3) HOW YOU ADDRESSED THE RESEARCH QUESTION (3 slides)
    % This should inform the audience as to what you did during your research to complete your project.
    % You should put a primary emphasis on explaining the statistical issues that you addressed in the project, with clear presentation of models and methods of analysis that you used.

    % bivariate data
    % tests
    % simulation code  Complexity and simulation time
    \begin{columns}[t]
      \column[T]{.4\textwidth}
        \begin{itemize}
          \item Review literature
            \begin{itemize}
              \item Twins
              \item Correlations
              \item Power
              \item Simulations
            \end{itemize} 
          \item Analysis plan
             \begin{itemize}
               \item Efficiency
               \item Data structure
               \item Outputs
             \end{itemize} 
          \item Simulations
          \item Analysis
          \item Write up
          \item Dissemination
        \end{itemize} 
      \column[T]{.6\textwidth}
          \includegraphics[width=\linewidth]{{../figs/david_corr_bands}.pdf}
    \end{columns}
  \end{frame}   


  \subsection{Power for difference in correlations}   
  \begin{frame}
    \frametitle{Power for difference in correlations}  
    \textbf{Frequentist NHST paradigm} \\
    $\alpha$ (Type I error) and $\beta$ (Type II error); Power is $1-\beta$
    \linebreak
    \linebreak
    \textbf{Fisher's $Z$ formula approach (David, 1938; Cohen 1988)}
    \linebreak
    $ \power = 1- \Phi \Bigg(\Phi_{\alpha/2}^{-1} -  \abs\bigg(  \frac{\arctanh(r_{MZ}) - \arctanh(r_{DZ})}{\sqrt{(n_{MZ}-3)^{-1} + (n_{DZ}-3)^{-1}}}  \bigg)  \Bigg) $ 
    \linebreak
    \textbf{Simulation approach}
    \begin{itemize}
      \item implement hypothesis tests for difference in correlations
      \item $M$ times
        \begin{itemize}
          \item draw from simulated bivariate MZ and DZ populations
          \item run hypothesis tests for difference in sample correlations
        \end{itemize}
      \item power given parameters is proportion of tests where $p < \alpha$      
    \end{itemize} 
  \end{frame}  
  
  \begin{frame}
    \frametitle{Simulations}
    Using $r_{MZ}$ and $r_{DZ}$ derived from simulated bivariate normal and non-normal populations with respective covariance $\left[\begin{smallmatrix}1 & \rho\\ \rho & 1\end{smallmatrix}\right] $, the following hypothesis tests were run:
    \linebreak
    \begin{itemize}
    \item Fisher's Z: $\frac{\arctanh(r_{MZ}) - \arctanh(r_{DZ})}{\sqrt{(n_{MZ}-3)^{-1} + (n_{DZ}-3)^{-1}}}$ {\footnotesize{(David, 1938)}}
    \item Zou's confidence interval  {\footnotesize{(Zou, 2007)}}
    \item Generalized Variable Test {\footnotesize{(GVT; Krishnamoorthy {\&} Lee 2014; Kazemi {\&} Jafari, 2016)}}
    \item Signed log-likelihood ratio test {\footnotesize{(SLR; DiCiccio, 2001; Kazemi {\&} Jafari, 2016)}}
    \item Permutation test {\footnotesize{(PT; Efron {\&} Tibshirani, 1994)}}
    \end{itemize}
    \linebreak
    \linebreak
   \textbf{Plan} 
   1000 simulations of all MZ DZ combinations of $rho$ and $n$ using three distinct distributions two correlation methods... 33 years? Fewer correlations, no PT test: 16 days
  \end{frame}   

  \section{results}
  % \begin{frame}
    % \frametitle{Preliminary results: simulation times}
    % \linebreak
    % $1000\times39^2\times7^2\times6\times3\times2 = 2,234,400,000$ simulations...
    % \textbf{Optimisation approach}\footnotemark
    % \begin{itemize}
    % \item hand coded
    % \item functions compiled using either JIT compiler or RCCP 
    % \end{itemize}
    % \begin{table}\centering
    % \begin{tabular}{rcc}
        % \toprule
        % \textbf{Test} &	\multicolumn{2}{c}{\textbf{Time/1000 runs (secs)}} \\
        % \cmidrule(lr){2-3} 
         % & \textbf{as is} & \textbf{compiled}				\\
        % \midrule										
        % Fisher's z (no sim) &         0.03 &   0.02 \\
        % Fisher's Z          &         0.56 &   0.36 \\
        % Zou's CI            &         0.38 &   0.28 \\
        % GTV (r)             &        18.65 &  16.55 \\
        % GTV (R C++)         &              &  12.08 \\
        % Permutation (PT)    &      2473.18 &2341.62 \\
        % SLR                 &         0.49 &   0.33 \\
        % \midrule
        % \multicolumn{1}{l}{Total (sans r-compiled GTV)} & 2493.29  & 2354.69 \\
        % \multicolumn{1}{l}{$\times 1520 \times 49 \times 3 \times 2/60/60/24/365$}  & 35 years  & 33 years \\
        % \multicolumn{1}{l}{$-\text{PT}: \times 361 \times 49\times 3 \times 2 /60/60/24$}  & 25 days  & 16 days \\
    % \bottomrule 
    % \end{tabular}
    % \end{table}

    
    
    \footnotetext[1]{Thanks to Koen Simons for advice}
    % 4) RESULTS (2-3 slides)
    % This should inform the audience as to what you found during your research to complete your project.
    % You need to present a summary of the analysis. This could include tables and figures of results and text, again with a primary focus on statistical issues, e.g. possible effects of decisions taken during the modelling and choice of estimation methods.
  \end{frame}   

  \begin{frame}
    \frametitle{Main findings}
    \begin{itemize}
      \item 530,670 power estimates comparing: tests, sample size, group size ratio, normality, correlation combinations, correlation methods
      \item Under approximate normality / mild skew, on average: \\SLR test $ \sim 82\%$ power; others $\leq 75\%$. 
    \end{itemize}
    \linebreak
    \small{Power to detect $\hat\delta_{\rho}$ in MZ and DZ twins $\sim \mathcal{G}\bigg(\left[\begin{smallmatrix}1\\ 1\end{smallmatrix}\right],\, \left[\begin{smallmatrix}5 & {\rho} \\ {\rho} & 5\end{smallmatrix}\right] \bigg)$}
    \begin{center}
      \begin{figure}[!htb]
        \minipage{0.49\textwidth}
          \includegraphics[width=\linewidth]{{../figs/corrxplot_slr_example}.pdf}
        \endminipage\hfill
        \minipage{0.49\textwidth}%
          \includegraphics[width=\linewidth]{{../figs/power_n_plot_example}.pdf}
        \endminipage
      \end{figure}
    \end{center}
  \end{frame}   

  \begin{frame}
    \frametitle{Main findings}
    \begin{itemize}
      \item 530,670 power estimates comparing: tests, sample size, group size ratio, normality, correlation combinations, correlation methods
      \item Under extreme skew: \\Fisher's Z formula underestimates req. sample size by $ \sim 50\%$
    \end{itemize}
    \linebreak
    \small{Power to detect $\hat\delta_{\rho}$ in MZ and DZ twins $\sim \mathcal{G}\bigg(\left[\begin{smallmatrix}1\\ 1\end{smallmatrix}\right],\, \left[\begin{smallmatrix}5 & {\rho} \\ {\rho} & 5\end{smallmatrix}\right] \bigg)$}
    \begin{center}
      \begin{figure}[!htb]
        \minipage{0.49\textwidth}
          \includegraphics[width=\linewidth]{{../figs/corrxplot_slr_example2}.pdf}
        \endminipage\hfill
        \minipage{0.49\textwidth}%
          \includegraphics[width=\linewidth]{{../figs/power_n_plot_example2}.pdf}
        \endminipage
      \end{figure}
    \end{center}
  \end{frame}     
  
  \begin{frame}
    \frametitle{Strengths and Limitations}
   %  any limitations � concerning either the data and substantive question or the statistical methods.
  Achievements
   \begin{itemize}
     \item a flexible, extensible architecture
     \item solves future problems
   \end{itemize}  
  But analysis is still in progress:
  \begin{itemize}
     \item 100 simulations for each parameter combinations
     \item 1000 is currently processing
   \end{itemize}
  Yet to be implemented:
  \begin{itemize}
     \item Intraclass correlations
     \item Partial correlations 
     \begin{itemize}
       \item incorporation of evaluation of correlation using mixed effects regression approaches could address above two issues
     \end{itemize}
     \item Improve efficiency to allow higher resolution estimation
   \end{itemize}
  \end{frame}   
  
  \begin{frame}
    \frametitle{Sum up and next steps}
    % \tikz\node[minimum width=\paperwidth, minimum height=\paperheight, anchor=north west] {\includegraphics[height=\paperheight,width=\paperwidth]{{../figs/webapp_example}.pdf}}

    We have developed an architecture for detecting differences in correlations and comparison of methods which is readily extensible for future applications (new tests, specific requests, new distributions, etc).
   %  This slide(s) should summarise in 2 or 3 sentences your overall findings. It should include a conclusion and possibly a recommendation for further research, 
    
 % \makebox[\textwidth][c]{\begin{tikzpicture}
    % \node[minimum width=\paperwidth, minimum height=\paperheight, anchor=north west] (a) {{../figs/webapp_example}.pdf};
  % \end{tikzpicture}
  % }
  
  \end{frame}  
\end{document}