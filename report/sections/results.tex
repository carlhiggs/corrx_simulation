\chapter*{3. Results}
\markboth{Results}{}
\setcounter{chapter}{3}
\setcounter{section}{0}
\addcontentsline{toc}{chapter}{3. Results}

\section{section title}


    \begin{itemize}
      \item 530,670 power estimates comparing: tests, sample size, group size ratio, normality, correlation combinations, correlation methods
      \item Under approximate normality / mild skew, on average: \\SLR test $ \sim 82\%$ power; others $\leq 75\%$. 
    \end{itemize}
    
    
    Power to detect $\hat\delta_{\rho}$ in MZ and DZ twins $\sim \mathcal{N}\bigg(\left[\begin{smallmatrix}0\\ 0\end{smallmatrix}\right],\, \left[\begin{smallmatrix}1 & {\rho} \\ {\rho} & 1\end{smallmatrix}\right] \bigg)$
    
     \begin{center}
      \begin{figure}[!htb]
        \minipage{0.49\textwidth}
          \includegraphics[width=\linewidth]{{../figs/corrxplot_slr_example}.pdf}
        \endminipage\hfill
        \minipage{0.49\textwidth}%
          \includegraphics[width=\linewidth]{{../figs/power_n_plot_example}.pdf}
        \endminipage
      \end{figure}
    \end{center}
    
    
   % \begin{itemize}
      % \item 530,670 power estimates comparing: tests, sample size, group size ratio, normality, correlation combinations, correlation methods
      % \item Under extreme skew: \\Fisher's Z formula underestimates req. sample size by $ \sim 50\%$
    % \end{itemize}
    % \linebreak
    % \small{Power to detect $\hat\delta_{\rho}$ in MZ and DZ twins $\sim \mathcal{G}\bigg(\left[\begin{smallmatrix}1\\ 1\end{smallmatrix}\right],\, \left[\begin{smallmatrix}5 & {\rho} \\ {\rho} & 5\end{smallmatrix}\right] \bigg)$}
    % \begin{center}
      % \begin{figure}[!htb]
        % \minipage{0.49\textwidth}
          % \includegraphics[width=\linewidth]{{../figs/corrxplot_slr_example2}.pdf}
        % \endminipage\hfill
        % \minipage{0.49\textwidth}%
          % \includegraphics[width=\linewidth]{{../figs/power_n_plot_example2}.pdf}
        % \endminipage
      % \end{figure}
    % \end{center}
\begin{landscape}   
  \begin{figure}[htb]
    \centering        
    \minipage{0.25\textwidth}%
      \includegraphics[width=\linewidth]{{../figs/distx_normal_60_120}.pdf}
      % \caption{Example bivariate normal draw with $\rho(0.2,0.5)$}
      % \label{fig:1}
    \endminipage\hfill       
    \minipage{0.25\textwidth}%
      \includegraphics[width=\linewidth]{{../figs/corrx_npower_norm_r.2_r.5_mzdz.5_s100}.pdf}
      % \caption{Sample size estimate: 100 simulations}
      % \label{fig:1}
    \endminipage\hfill
    \minipage{0.25\textwidth}% 
      \includegraphics[width=\linewidth]{{../figs/corrx_npower_norm_r.2_r.5_mzdz.5_s1000}.pdf}
      % \caption{Sample size estimate: 1,000 simulations}
      % \label{fig:2}
    \endminipage\hfill
    \minipage{0.25\textwidth}% 
      \includegraphics[width=\linewidth]{{../figs/corrx_npower_norm_r.2_r.5_mzdz.5_s10000}.pdf}
      % \caption{Sample size estimate: 10,000 simulations}
      % \label{fig:3}
    \endminipage\hfill
    \caption{Left to right: Example bivariate normal draw with $\rho(0.2,0.5)$, and sample size estimates to achieve 80\% power using 100, 1000, and 10000 simulation runs per scenario}
    \label{fig:images1}
    
    \medskip
    \minipage{0.25\textwidth}% 
      \includegraphics[width=\linewidth]{{../figs/distx_gamma_mildskew_60_120}.pdf}
      % \caption{Example bivariate gamma ("mild skew") draw with $\rho(0.2,0.5)$}
      % \label{fig:4}
    \endminipage\hfill
    \minipage{0.25\textwidth}% 
      \includegraphics[width=\linewidth]{{../figs/corrx_npower_mildskew_r.2_r.5_mzdz.5_s100}.pdf}
      % \caption{Sample size estimate: 100 simulations}
      % \label{fig:4}
    \endminipage\hfill
    \minipage{0.25\textwidth}% 
      \includegraphics[width=\linewidth]{{../figs/corrx_npower_mildskew_r.2_r.5_mzdz.5_s1000}.pdf}
      % \caption{Sample size estimate: 1,000 simulations}
      % \label{fig:5}
    \endminipage\hfill
    \minipage{0.25\textwidth}% 
      \includegraphics[width=\linewidth]{{../figs/corrx_npower_mildskew_r.2_r.5_mzdz.5_s10000}.pdf}
      % \caption{Sample size estimate: 10,000 simulations}
      % \label{fig:6}
    \endminipage\hfill
    \caption{Left to right: Example bivariate gamma ("mild skew") draw from population with $\rho(0.2,0.5)$, and sample size estimates to achieve 80\% power using 100, 1000, and 10000 simulation runs per scenario}
    \label{fig:images2}
    
    \medskip
    \minipage{0.25\textwidth}% 
      \includegraphics[width=\linewidth]{{../figs/distx_gamma_extrskew_60_120}.pdf}
      % \caption{Example bivariate gamma ("extreme skew") draw with $\rho(0.2,0.5)$}
      % \label{fig:4}
    \endminipage\hfill
    \minipage{0.25\textwidth}% 
       \includegraphics[width=\linewidth]{{../figs/corrx_npower_extrskew_r.2_r.5_mzdz.5_s100}.pdf}
      % \caption{Sample size estimate: 100 simulations}
      % \label{fig:4}
    \endminipage\hfill
    \minipage{0.25\textwidth}% 
      \includegraphics[width=\linewidth]{{../figs/corrx_npower_extrskew_r.2_r.5_mzdz.5_s1000}.pdf}
      % \caption{Sample size estimate: 1,000 simulations}
      % \label{fig:5}
    \endminipage\hfill
    \minipage{0.25\textwidth}% 
      \includegraphics[width=\linewidth]{{../figs/corrx_npower_extrskew_r.2_r.5_mzdz.5_s10000}.pdf}
      % \caption{Sample size estimate: 10,000 simulations}
      % \label{fig:6}
    \endminipage\hfill
    \caption{Left to right: Example bivariate gamma ("extreme skew") draw from population with $\rho(0.2,0.5)$, and sample size estimates to achieve 80\% power using 100, 1000, and 10000 simulation runs per scenario}
    \label{fig:images3}
  \end{figure}
\end{landscape}