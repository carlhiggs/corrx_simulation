\chapter*{2. Methods}
\markboth{Methods}{}
\setcounter{chapter}{2}
\setcounter{section}{0}
\addcontentsline{toc}{chapter}{2. Methods}


\section{section title}

I conducted a review of the literature relating to twins, correlations, power and simulations to inform development of a plan for our analysis.  

We identified a series of suitable approaches and tests for evaluating differences in Pearson and Spearman correlations in groups.  The Spearman correlation is a non-parametric alternative to Pearson's, using the same formula on the rank ordered variables rather than their raw values.

Other important considerations were the efficiency of our implemented simulation functions, and a well-designed data structure to support our planned as well as future outputs.  

The R programming environment was used for all of our analyses.

    \begin{center}
      % \includegraphics{../figs/r_to_z.pdf}
    \begin{figure}[!htb]
        \minipage{0.3\textwidth}
          \includegraphics[width=\linewidth]{{../figs/twin_images/my_twin_drawing_mz}.pdf}
        \endminipage\hfill
        \minipage{0.39\textwidth}
           \[\big(x_{z \lcd 1} , x_{z \lcd 2}\big) \sim \mathcal{N}\big(\bm{\mu},\,\Sigma \big) \]
           \[r_z = \hat{\rho}_z = \frac{Cov(x_{z \lcd 1} , x_{z \lcd 2})}{\hat{\sigma}_{x_{z \lcd 1}} \hat{\sigma}_{x_{z \lcd 2}}} \]
           \[\hat{\delta_r} = r_{MZ} - r_{DZ}\]
           \[\text{heritability} = 2\delta_{\rho}\]
        \endminipage\hfill
        \minipage{0.3\textwidth}%
          \includegraphics[width=\linewidth]{{../figs/twin_images/my_twin_drawing_dz}.pdf}
        \endminipage
    \end{figure}
    \begin{table}\centering
    \begin{tabular}{rcc}
        \toprule
        \textbf{Trait etiology} &	\multicolumn{2}{c}{\textbf{Correlation in twin pairs}} \\
        \cmidrule(lr){2-3} 
         & \textbf{Monozygotic (MZ)} & \textbf{Dizogotic (DZ)}				\\
        \midrule										
        genetic         &$r_{MZ} = 1$& $r_{DZ} = 0.5$       \\
        shared env.     &$r_{MZ} = 1$& $r_{DZ} = 1$         \\  
        individual env. &$r_{MZ} = 0$& $r_{DZ} = 0$         \\  
        combination     &$0 < r_{MZ} < 1$& $0 < r_{DZ} < 1$ \\
         \bottomrule 
    \end{tabular}
    \end{table}
    \end{center}

\section{Frequentist NHST paradigm} 
    $\alpha$ (Type I error) and $\beta$ (Type II error); Power is $1-\beta$
    \linebreak
    \linebreak
    \textbf{Fisher's $Z$ formula approach (David, 1938; Cohen 1988)}
    \linebreak
    $ \power = 1- \Phi \Bigg(\Phi_{\alpha/2}^{-1} -  \abs\bigg(  \frac{\arctanh(r_{MZ}) - \arctanh(r_{DZ})}{\sqrt{(n_{MZ}-3)^{-1} + (n_{DZ}-3)^{-1}}}  \bigg)  \Bigg) $ 
    \linebreak
    \section{Simulation approach}
    \begin{itemize}
      \item implement hypothesis tests for difference in correlations
      \item $M$ times
        \begin{itemize}
          \item draw from simulated bivariate MZ and DZ populations
          \item run hypothesis tests for difference in sample correlations
        \end{itemize}
      \item power given parameters is proportion of tests where $p < \alpha$      
    \end{itemize} 
    
\subsection{Formula based hypothesis test}

\subsection{Simulation based hypothesis test}
The tests we identified and developed for inclusion were 

\subsubsection{Fisher's Z test}
the simulation equivalent of the Fisher's Z test already discussed
 $$\frac{\arctanh(r_{MZ}) - \arctanh(r_{DZ})}{\sqrt{(n_{MZ}-3)^{-1} + (n_{DZ}-3)^{-1}}}$$
\subsubsection{Zou's confidence interval}
Zou's confidence interval approach which evaluates whether zero lies within the lower and upper bounds of the interval estimate of the difference in correlations, returning 1 if so or otherwise zero.  Over a run of simulations this would be expected to return identical results to the Fisher Z test, but may be more efficient.

\subsubsection{Generalised Variable Test}

The GVT test involves tranformation of the simulated sample correlations into so-called pivotal quantities the difference of which is used to calculate a test statistic and p-value.


\subsubsection{Signed log-likelihood ratio test}
The signed log likelihood ratio test is formulated as the signed difference in sample correlation coefficients multiplied by the square root of the sum of respective coefficients' log-likelihoods.


\subsubsection{Permutation test}
The permutation test is a non-parametric approach which compares the absolute difference of the Z-transformed sample correlations with correlations from a series of group membership permutations using the sample rank orders as values.

\subsection{Efficiency}
To undertake the simulations as planned for combinations of 
\begin{itemize}
  \item correlations at .05 intervals 
  \item exponentially increasing group sizes of 15 through 960
  \item three bivariate distribution types being normal, and gamma with mild skew and extreme skew
  \item two approaches to correlation measurement 
  \item across 6 tests
  \item with 1000 simulations per combination 
\end{itemize}
would have resulted in more than 2 billion results and taken 33 years.  Instead, I reduced my ambition using a .1 correlation resolution and dropped the permutation test which was implemented inefficiently to get this to 16 days.
  


