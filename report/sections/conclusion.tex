\chapter*{5. Conclusions}
\markboth{Conclusions}{}
\setcounter{chapter}{5}
\setcounter{section}{0}
\addcontentsline{toc}{chapter}{5. Conclusion}

To sum up, 
\\
tests were overall quite similar
\\
simulation important for power analysis where notable violation of assumptions of bivariate normality is anticipated  (naive application of analytical Fisher's Z formula can be extremely over-optimistic in required sample estimates)
\\
SLR implementation is biased\; highlights importance of critical consideration power of methods, as higher power may reflect bias.  For this particular SLR test, power estimates in the case of unequal group ratios were upwardly biased.
\\
Power should be considered contextualised using planned conditions for study and subject matter knowledge, and critical consideration of impact of methods employed.
\\
We have created both the architecture for a process, as well as a database of simulation scenarios that can be interrogated.  Both can be expanded as required. I have trialled an interactive power calculator web app, and it is planned incorporate the pre-processed database into this to allow on the fly estimates informed by our pre-processed results.