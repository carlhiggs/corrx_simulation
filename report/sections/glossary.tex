 %%%%%%%%%%%%%%%%%%%%%%%%%%%%%%%%%%%%%%%%%%%%%%%%%%%%%%%%%%%%%%%
% From Springer template
% 
% 
%
%%%%%%%%%%%%%%%%%%%%%%%% Springer %%%%%%%%%%%%%%%%%%%%%%%%%%

\Extrachap{Glossary}
\markboth{Glossary}{}
\runinhead{2-tuple} An $n$-tuple refers to a set of values.  Due to the bivariate focus of our simulations, our simulation functions include a number of parameters which take 2-tuples of values.  For example, sample size \code{n = c(15,60)} specifies an MZ group size of 15 and a DZ group size of 60, total n of 85 and MZ:DZ ratio of 1:4.
\runinhead{argument} A value passed to a parameter in a function call.
\runinhead{composite scenario} Power estimates returned across a series of tests for comparison purposes
\runinhead{correlation} A measure of the magnitude and direction of a linear relationship shared by two variables.
\runinhead{dizygotic} Non-identical twins arising from fertilisation of two seperate fertilised eggs, and as genetically alike as ordinary siblings.
\runinhead{heritability} The degree to which variation in a trait or phenotype, such as propensity to gain body weight, or become a centenarian, can be attributed to shared genetic effects.
\runinhead{monozygotic} Identical twins, developing from the same fertilised egg (zygote) and genetically very similar.
\runinhead{parameter} In the context of computer programming, a parameter is a variable in a function to which arguments may be passed in order to parameterise a particular call to that function.  It is acknowledged that this is a slightly nuanced from the use of the term in the context of statistical modelling, hence inclusino in the glossary.
\runinhead{$r$} Sample estimate of the Pearson correlation coefficient.
\runinhead{$\rho$} (rho) The Pearson correlation coefficient in the population.
\runinhead{scenario} The set of parameters that give rise to a simulation of estimated power for a particular test.
