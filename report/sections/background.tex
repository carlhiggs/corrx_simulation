\chapter*{1. Introduction}
\markboth{1. Introduction}{}
\setcounter{chapter}{1}
\addcontentsline{toc}{chapter}{1. Introduction}

			% o	Summary of background and research methods; why this analysis/ evaluation is important; how and why the data was collected (initial study aims, collection methods, response rates); specifics of data (number, variables, outcomes)
			% Research question, aim, objectives of study
			
			% Research question:
			% What are the existing methods for estimating power to detect a difference in correlations between identical (monozygotic) and non-identical (dizygotic) twins, how do these compare and can they be improved upon? 

\large


% \begin{figure}[htbp]
% \sidecaption[t]
% %\centering
% \fbox{\includegraphics[scale=0.5]{figures/PapuaEndemicity.png}}
% %\picplace{5cm}{2cm} % Give the correct figure height and width in cm
% \caption{caption for figure.}
% \label{fig:pvpr}       % Give a unique label
% \end{figure}

The classic twin study exploits the differing degrees of genetic relatedness in identical (\mz) and non-identical (\dz) twins in order to draw inferences on the heritability of traits.  In broad terms, heritability is the degree to which variation in a trait or phenotype, such as propensity to gain body weight, or become a centenarian, can be attributed to shared genetic effects.  The calculation and comparison of Pearson correlations across the two twin groups is a routine preliminary step undertaken by researchers in this field.  However, a range of factors can impact on researchers' ability to detect a true effect given data.  
\\
\\
This thesis reports on a simulation-based power analysis for the detection of differences in correlations between \mz \ and \dz \ twin pairs under a range of scenarios, and the associated development of R functions and an applied interactive power calculator.  These tools address an identified absence of tools for this fundamental step in the twin study context.
\\
\\
In this chapter we first define key genetic concepts, how these are exploited by the classic twin study, and some of the assumptions which are relied upon in order to make such inferences.  Then, the concept of power analysis and related statistical concepts are defined in order to provide background, justification and notation for the following chapters. 
 
\section{The classic twin study and power}
Identical twins (\mz, MZ) arise from the same zygote, or fertilised egg, and are genetically very similar.  Non-identical twins (\dz, DZ) arise from fertilisation of two seperate eggs, and are as genetically alike as ordinary siblings.  The comparison of phenotypic traits within identical and non-identical twin pair samples allows for partitioning the variance in traits into that attributable to  shared environment, individual environment or to genetics  \cite{Teare2011}.  Known as the classic twin study, this approach can help us to better understand the mechanics of health and disease processes so that we can develop intervention measures which appropriately target the hypothesised causal mechanisms.
% \\
% \\
\tdn{Falconer,  heritability and difference in MZ and DZ; Paraphrase Visscher re heritability and the various sense of this}
% \cite{Visscher2008b}
% In the context of genetics and kinship, Douglas Falconer outlined methods of using comparison of identical and non-identical twins for estimating heritability, and noted some of the assumptions that this involves \cite{Falconer1960}.

\\
\\
Contemporary twin studies use mixed effects and structural equation modelling to evaluate differences in variance components for a particular trait between mono and dizygotic twins accounting for differential within pair similarities related to zygosity \cite{Neale1992,Carlin2005}.  Michael Neale and colleagues developed methods and software to facilitate the analysis of variance components in twin studies using structural equation modelling \cite{Neale1992}.  Brad Verhulst, building on the work of Peter Visscher, developed functions for power analyses in this variance component modelling context, for example detecting a difference in genetic correlations \cite{Verhulst2017,Visscher2004,Visscher2008a}. 
\\
\\
Visscher's work noted that the ratio of MZ to DZ twins is an important parameter to consider with regard to power to estimate variance components in twin studies, involving a trade off between power to detect additive genetic ($>1:1$) or shared environmental effects ($<1:1$) \cite{Visscher2008a}.  Such effects may be the main objects of interest in contemporary studies when planning a study, and power calculators such as that of Verhulst \cite{Verhulst2017} specifically cater to these concerns.  
\\
\\
However, a routine preliminary step undertaken by researchers in this field is the calculation and comparison of Pearson correlations across the two twin groups.  This research project focuses on the reported needs of researchers undertaking this early analysis step.  Indeed, of 26 results returned from a 'Best Match' PubMed database search for "twin pearson difference", five published between 1994 and 2014 made comparisons between MZ and DZ twin groups based on differences in Pearson correlations; based on the stated correlations and group sample sizes, and using a standard formula based power test we implemented (described later), three of these articles reported differences in correlations based on tests with 40\% or lower power (see Appendix \ref{ch:lit}).  While the use of simple statistical tests in the early stages of analysis is justifiable, the application of these and reporting of results as though constituting meaningful evidence may be misleading.  The use of power analysis before a study, or even once recruiting has been completed, may better guide researchers to investigate problems which can more reasonably be solved given the samples and data available, and processes of interest.
\\
\\
In undertaking a twin study we make certain assumptions, understanding that these likely do not strictly hold in practice, but the key one for the purposes of this power analysis is that our data is approximately normally distributed.

\section{Correlation: Historical and statistical background}
\subsection{Pearson product-moment correlation}
In a context of broad social interest in eugenics, Francis Galton described methods which could be used to describe the 'co-relatededness' of variables sourced from closely related family members \cite{Galton1888,Galton1890}.  Karl Pearson, a keen follower of Galton's research and writing in the context of inheritance and natural selection, described  'Galton's function' as a coefficient of correlation \cite{Pearson1895}.  Known today as the product-moment or Pearson correlation coefficient, the function is used to describe the magnitude and direction of linear change in one random variable as another changes \cite{Fisher1990,Casella2002,Diedenhofen2015}.  
\\
\\
Inference based on the Pearson correlation coefficient assumes bivariate normality.  That is, given two random variables $(x_1,x_2)$, we assume they have jointly normal probability distribution \cite{David1938}
$$p(x_1,x_2) =
      \frac{1}{2 \pi  \sigma_{x_1} \sigma_{x_2} \sqrt{1-\rho^2}}
      e^{-\frac{1}{2(1-\rho^2)}\left\{
          \frac{(x_{1}-\mu_{X_1})^2}{\sigma_{X_1}^2} -
          \frac{2\rho(x_{1}-\mu_{X_1})(x-\mu_{X_2})}{\sigma_{X_1} \sigma_{X_2}} +
          \frac{(x_{2}-\mu_{X_2})^2}{\sigma_{X_2}^2} 
        \right\}
      \right)$$
% The bivariate normal distribution has mean $\mu$ and correlation $\Sigma$
% $$\boldsymbol\mu = \begin{pmatrix} \mu_X \\ \mu_Y \end{pmatrix}, \quad \boldsymbol\Sigma = \begin{pmatrix} \sigma_X^2 & \rho \sigma_X \sigma_Y \\ \rho \sigma_X \sigma_Y  & \sigma_Y^2 \end{pmatrix}$$
      
The population correlation $\rho$ (rho) is the covariance $\Cov(x_1,x_2) =  \operatorname{E}[(x_1-\mu_{X_1})(x_2-\mu_{X_2})]$ divided by the product of the two variables' standard deviations $\sigma_X\sigma_Y$.  The estimate of $\rho$ based on a sample of observed data is denoted as $r$.

$$r = \frac{\sum_{i=1}^n (x_{1_i} - \bar{x}_1)(x_{2_i} - \bar{x}_2)}{\sqrt{\sum_{i=1}^n (x_{1_i} - \bar{x}_1)^2 \sum_{i=1}^n (x_{2_i} - \bar{x}_2)^2}}$$
\\
\\
If the variables $\boldsymbol{x}$ are standardised using their respective sample means ${\bar{x}} = \frac{1}{n}\sum_{i=1}x_i$ and standard deviations $s = \frac{1}{\sqrt{n-1}}$ where $n$ is the sample size, as $u$
$$u = \frac{1}{\sqrt{n-1}}(x_i - \bar{u}),$$
then the standardised regression coefficient $b_1$ arising from the simple linear regression of $u_1 = b_0 + b_1 u_2$ is equal to the Pearson correlation \cite{Tu2012}.  

\subsection{Spearman rank correlation}
% Spearman correlation
 In the case of data with outliers which violate our assumptions of bivariate normality, the non-parametric Spearman's correlation $r_s$ may be considered as an alternative to Pearson's: using the same formula for $r$, is is calculated using the rank order transformation of the variables rather than their raw values \cite{Fieller1957,StataCorp2013}. 

\subsection{Alternative approaches to correlation}
The scope of the present study was restricted to Pearson and Spearman correlations, however alternative approaches to estimating correlation should be considered and for completeness reviews of the following are included in appendix \ref{ch:alt}: Kendall's $\tau$; partial correlations; and intra-class correlations.  The latter two are of particular relevance to twin studies.

\subsection{Inference on observed correlations}
Ronald Fisher observed that a geometric transformation of the correlation coefficient using its inverse hyperbolic tangent could be used to approximate a normal distribution \cite{Fisher1915}.  

$$z = \arctanh(r) = \frac{1}{2} \log_e \frac{1+r}{1-r}$$

\begin{figure}[htbp]
\sidecaption[t]
%\centering
\fbox{\includegraphics[scale=1.32]{../figs/r_to_z.pdf}}
%\picplace{5.5cm}{2cm} % Give the correct figure height and width in cm
\caption{Transformation of $r$ to $z$ using Fisher's method.}
\label{fig:ztransform}       % Give a unique label
\end{figure}

Fisher's Z transformation has the effect of mapping the distribution of correlation of coefficients from a domain of $(-1 < r < +1)$ to $(-\infty < z < +\infty)$, and is used for inferences on the Pearson correlation coefficient, as well as the rank-based and ICC varieties described above and in Appendix \ref{ch:alt} \cite{Fisher1990, Fieller1957, Barrett2008}.  Through a comparison of the use of the z-transformation with exact and other methods used in hypothesis tests concerning $r$, David concluded that the z-transformation should be appropriate for most purposes \cite{David1938}.  Being a simple and accurate approximation of the normal distribution, this transformation known as Fisher's Z is ubiquitous in statistical treatment of correlation coefficients, for example when seeking to compare their differences and is employed in functions found in contemporary statistical software packages such as $R$ and Stata \cite{R2018,StataCorp2013}.

\\

\begin{figure}[htbp]
\sidecaption[t]
%\centering
\fbox{\includegraphics[scale=0.65]{../figs/david_corr_bands.pdf}}
%\picplace{5cm}{2cm} % Give the correct figure height and width in cm
\caption{This visualisation of the chance of rejecting the null hypothesis when true given alpha, rho and n was an early inspiration for visualising our simulation results \cite{David1938}.}
\label{fig:davidplot}       % Give a unique label
\end{figure}

\section{Power analysis and hypothesis testing} 
Power analysis involves a compromise between type 1 ($\alpha$) and type 2 ($\beta$) error thresholds, respectively representing the expected proportion of null hypotheses to be rejected when true and not rejected when false  \cite{Freiman1978}.  These could be chosen to suit the requirements of a particular study, but for historical reasons the respective values chosen tend to be 5\% ($\alpha = .05$) and 20\% ($\beta = 0.2$) \cite{Cohen1988}.  

\subsection{Significance testing and controversy}
There are controversies around the choice of such values, relating to what it means to deem something as 'statistically significant', a synonym for $p < \alpha$. A recent commentary suggested redefining significance as $p < 0.005$ \cite{Benjamin2018}; other authors writing in a biostatistics context have advised reporting of exact p-values to be prefered in general to dichotomisation at a somewhat arbitrary $\alpha$ level, and that there is justification for use in some contexts of confidence intervals based on $\alpha=0.1$ \cite{Clayton1993}.  These are epistemological questions, detailed discussion of which is beyond the scope of this research project.  However it is important to acknowledge that the present study is concerned with an applied statistical question (power to detect differences in correlations in groups with distinct genetic relatedness) with strong historical links to the hypothesis testing paradigm.  

\subsection{Importance of power analysis}
Regardless of debates surrounding the use of hypothesis testing and thresholding, the question of power is one of academic rigour with moral implications: if a study is to be conducted with support through significant contribution of public resources in the way of grant funding, there is an ethical imperative to consider $a \ priori$ whether the experiment in question can anticipate production of meaningful evidence \cite{Freiman1978,Cohen1988}.  Power analysis aims to answer this question through consideration of the sample size required to detect an effect of such magnitude to be considered meaningful given the specified values of $\alpha$.  

\subsection{Calculation of power}
The probability of observing a true effect, the power of a statistical test, is equal to the proportion of null hypotheses rejected when false; that is, $1 - \beta$ \cite{Cohen1988}.  Hence, the choice of $\beta$ will indicate directly the intent for power: if in planning a twin study we agree an appropriate value for $\beta$ is 0.2, then we are saying that our power to detect a difference of the required magnitude between identical and non-identical twins should be at least as high as 0.8, or 80\%. The ability to achieve this power depends on sample size, the ratio of MZ to DZ twins, the magnitude of the respective correlations themselves, and the degree to which our assumptions of bivariate normality hold. The robustness of the choice of test method employed to departures from distributional assumptions is also an important consideration.

\section{The aim of this report}
This report describes methods and results of a simulation analysis comprised of more than half a million power estimates, detailing more than 100,000 distinct MZ and DZ twin comparison scenarios evaluated across five different approaches to detecting a difference in correlations.  These can be accessed to answer specific questions.  The task of this report is to evaluate subsets of these and demonstrate the value and validity of the functions giving rise to them, and provide options to researchers such as those of Twins Australia for their analyses which may be expanded upon.