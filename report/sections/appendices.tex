%%Springer template
%%%%%%%%%%%%%%%%%%%%% appendix.tex %%%%%%%%%%%%%%%%%%%%%%%%%%%%%%%%%
%
% sample appendix
%
% Use this file as a template for your own input.
%
%%%%%%%%%%%%%%%%%%%%%%%% Springer-Verlag %%%%%%%%%%%%%%%%%%%%%%%%%%

\part*{Appendices}
\renewcommand\thechapter{\Alph{chapter}}
\markboth{Appendix \thechapter}{}

\setcounter{chapter}{0}
\setcounter{section}{0}
\addcontentsline{toc}{part}{Appendices}

\begin{appendix}
\chapter{Alternative approaches to correlation}
\label{ch:alt} % Always give a unique label
\markboth{Appendix \thechapter}{}

The scope of the present study was restricted to Pearson and Spearman correlations, however alternative approaches to estimating correlation should be considered and for completeness the following are reviewed below: Kendall's $\tau$; partial correlations; and intra-class correlations.  The latter two are of particular relevance to twin studies, however were beyond scope for inclusion for this research project.

\section{Kendall's $\tau$}
An alternate non-parametric option is Kendall's $\tau$ (tau) which provides a summary measure of correlation based on concordance of trend across the sample: pairs are concordant if the product of consecutive rank pair differences is $> 0$, and discordant if this product is $< 0$;  the number of concordant ($C$) and discordant ($D$) pairs are tallied, and the difference ($C � D$) is the score $S$;   Kendall's $\tau_a$ is calculated as $\frac{S}{n(n-1)/2}$, while other variant formulas include further adjustment to account for ties \cite{Fieller1957,StataCorp2013}.

\section{Partial correlation}
Adjustment for additional covariates, say $x_3$, may be introduced resulting in what is known as a partial correlation representing the linear relationship between $x_1$ and $x_2$ adjusting for the  effects of $x_3$ \cite{Fisher1990}. An elegant computational approach to calculating partial correlations drawing on the output of multiple regression analysis using standard software packages uses the test statistic of interest $t = \frac{b}{se}$, the sample size $n$, and number of covariates $k$ \cite{StataCorp2013}.  Using the example where we are interested in the partial correlation of $x_1$ (dependent variable in our regression model) and $x_2$ adjusting for the  effects of $u_3$, or $\rho_{x_1 x_2 \cdot x_3}$, where $k = 2$:

$$\rho_{x_1 x_2 \cdot x_3} = \frac{t_{x_{2}}}{\sqrt{t_{x_{2}}^2 + n - k}}$$

\section{Intra-class correlations}
The correlation approaches described above may be considered to be inter-class correlations: in the twin context, this amounts to looking at variable(s) pertaining to a set of (arbitrary) first members of MZ twins, and comparing with the remaining MZ twins.  This approach overlooks the paired nature of twin data.  In contrast, intra-class correlations (ICCs) draw upon within-pair pooled mean and standard deviation \cite{Fisher1990}.  If we refer to twin membership using the subscript $j$, and let $n$ be number of twin pairs in our sample:

\begin{align*}
\bar{x} &= \frac{1}{2n}\sum_{i=1}^{n} x_{ij}+x_{ij+1}  \\
s^2 &= \frac{1}{2n}\sum_{i=1}^{n} (x_{ij}-\bar{x})^2+(x_{ij+1}-\bar{x})^2  \\
r_{\text{ICC}} &= \frac{1}{ns^2}\sum_{i=1}^{n}  (x_{ij}-\bar{x})(x_{ij+1}-\bar{x})
\end{align*}

The above is considered a more accurate approach to correlation in contexts such as those of twin pairs, which do not have a natural ordering \cite{Fisher1990}.  A number of approaches to statistical modelling can be taken to account for the paired twin data structure \cite{Carlin2005}; a frequently used approach involves the calculation of ICCs for both MZ and DZ twin pairs through use of mixed effects modelling.  Calculated in this way,  $r_{\text{ICC}}$ represents the estimated ratio of between pair variation in a phenotype to total variation; the degree to which $r_{\text{ICC:MZ}}$ is larger than $r_{\text{ICC:DZ}}$ can be inferred to relate to the shared genetic basis for variation in the phenotype \cite{Barrett2008}.

\chapter{Annotated PubMed searches}
\markboth{Appendix \thechapter}{}
\label{ch:lit}
\section{"twin pearson difference" - 15 May 2018}
\markboth{Appendix \thesection}{}
\includepdf[landscape=true,scale=.9, pagecommand={\pagestyle{fancy}},pages=-]{sections/pubmed_result__twin_pearson_difference.pdf}


\chapter{Simulation tables}
\markboth{Appendix \thechapter}{}
\label{ch:simtab}
\section{Colour scale legend}
\includepdf[scale=.3, pagecommand={\pagestyle{fancy}},pages=-]{sections/sa3_colour_scales.pdf}
\section{Power estimates by test, correlation method, distribution, MZ:DZ ratio and sample size holding rho = {0.2,0.5}, using 10,000 simulations per scenario}
\markboth{Appendix \thesection}{}
\label{ch:sim_est}
% \includepdf[scale=.75, pagecommand={\pagestyle{fancy}},pages=-]{sections/sa3_1k_normal.pdf}
% \includepdf[scale=.75, pagecommand={\pagestyle{fancy}},pages=-]{sections/sa3_1k_gamma1.pdf}
% \includepdf[scale=.75, pagecommand={\pagestyle{fancy}},pages=-]{sections/sa3_1k_gamma2.pdf}
\includepdf[scale=.75, pagecommand={\pagestyle{fancy}},pages=-]{sections/sa3_10k_normal.pdf}
\includepdf[scale=.75, pagecommand={\pagestyle{fancy}},pages=-]{sections/sa3_10k_gamma2.pdf}
\includepdf[scale=.75, pagecommand={\pagestyle{fancy}},pages=-]{sections/sa3_10k_gamma1.pdf}

\section{Difference in power estimate using simulations (1,000 simulation estimates - 10,000 simulations estimates) by test, correlation method, distribution, MZ:DZ ratio and sample size holding at rho = {0.2,0.5}.  The 10,000 simulation results are reported in Appendix \ref{ch:sim_est}.  For conciseness, the corresponding results using 1,000 simulations have been omitted but can be gleaned from the following table, for the purposes of establishing validity of using 1,000 simulations.}
\markboth{Appendix \thesection}{}
\label{ch:sim_dif}
\includepdf[scale=.75, pagecommand={\pagestyle{fancy}},pages=-]{sections/sa3_diff_normal.pdf}
\includepdf[scale=.75, pagecommand={\pagestyle{fancy}},pages=-]{sections/sa3_diff_gamma1.pdf}
\includepdf[scale=.75, pagecommand={\pagestyle{fancy}},pages=-]{sections/sa3_diff_gamma2.pdf}


\chapter{GTV test, varying ratio}
\markboth{Appendix \thechapter}{}
\label{ch:contour_gtv}
%% Contour - GTV - vary ratio
\begin{figure}[htb]
  \centering        
  \minipage{0.49\textwidth}%
    \includegraphics[width=\linewidth]{{../figs/corrx_contour_sA1_n180_mzdz.5_gtv_s1000}.pdf}
  \endminipage\hfill 
  \minipage{0.49\textwidth}% 
    \includegraphics[width=\linewidth]{{../figs/corrx_contour_sA2_n180_mzdz1_gtv_s1000}.pdf}
  \endminipage\hfill
  \caption{Left to right, above: GTV test power given rho contour plots for bivariate normal distribution, using MZ:DZ group size ratio of (a) 60:120, and (b) 90:90}
  \label{fig:contour_gtv_norm}  
  
  \medskip
  \minipage{0.49\textwidth}%
    \includegraphics[width=\linewidth]{{../figs/corrx_contour_gamma1_sA1_n180_mzdz.5_gtv_s1000}.pdf}
  \endminipage\hfill 
  \minipage{0.49\textwidth}% 
    \includegraphics[width=\linewidth]{{../figs/corrx_contour_gamma1_sA2_n180_mzdz1_gtv_s1000}.pdf}
  \endminipage\hfill
  \caption{Left to right, above: GTV test power given rho contour plots for bivariate "mild skew" gamma distribution, using MZ:DZ group size ratio of (a) 60:120, and (b) 90:90}
  \label{fig:contour_gtv_gamma1}
  
  \medskip
  \minipage{0.49\textwidth}%
    \includegraphics[width=\linewidth]{{../figs/corrx_contour_gamma2_sA1_n180_mzdz.5_gtv_s1000}.pdf}
  \endminipage\hfill 
  \minipage{0.49\textwidth}% 
    \includegraphics[width=\linewidth]{{../figs/corrx_contour_gamma2_sA2_n180_mzdz1_gtv_s1000}.pdf}
  \endminipage\hfill  
  \caption{Left to right, above: GTV test power given rho contour plots for bivariate "extreme skew" gamma distribution, using MZ:DZ group size ratio of (a) 60:120, and (b) 90:90}
  \label{fig:contour_gtv_gamma2}
\end{figure} 

\chapter{SLR contour plot, varying ratio}
\markboth{Appendix \thechapter}{}
\label{ch:contour_slr}

%% Contour - SLR - vary ratio
\begin{figure}[htb]
  \centering        
  \minipage{0.49\textwidth}%
    \includegraphics[width=\linewidth]{{../figs/corrx_contour_sA1_n180_mzdz.5_slr_s1000}.pdf}
  \endminipage\hfill 
  \minipage{0.49\textwidth}% 
    \includegraphics[width=\linewidth]{{../figs/corrx_contour_sA2_n180_mzdz1_slr_s1000}.pdf}
  \endminipage\hfill
  \caption{Left to right, above: SLR test power given rho contour plots for bivariate normal distribution, using MZ:DZ group size ratio of (a) 60:120, and (b) 90:90}
  \label{fig:contour_slr_norm}  
  
  \medskip
  \minipage{0.49\textwidth}%
    \includegraphics[width=\linewidth]{{../figs/corrx_contour_gamma1_sA1_n180_mzdz.5_slr_s1000}.pdf}
  \endminipage\hfill 
  \minipage{0.49\textwidth}% 
    \includegraphics[width=\linewidth]{{../figs/corrx_contour_gamma1_sA2_n180_mzdz1_slr_s1000}.pdf}
  \endminipage\hfill
  \caption{Left to right, above: SLR test power given rho contour plots for bivariate "mild skew" gamma distribution, using MZ:DZ group size ratio of (a) 60:120, and (b) 90:90}
  \label{fig:contour_slr_gamma1}
  
  \medskip
  \minipage{0.49\textwidth}%
    \includegraphics[width=\linewidth]{{../figs/corrx_contour_gamma2_sA1_n180_mzdz.5_slr_s1000}.pdf}
  \endminipage\hfill 
  \minipage{0.49\textwidth}% 
    \includegraphics[width=\linewidth]{{../figs/corrx_contour_gamma2_sA2_n180_mzdz1_slr_s1000}.pdf}
  \endminipage\hfill  
  \caption{Left to right, above: SLR test power given rho contour plots for bivariate "extreme skew" gamma distribution, using MZ:DZ group size ratio of (a) 60:120, and (b) 90:90}
  \label{fig:contour_slr_gamma2}
\end{figure} 



Code
\chapter{Code}
\markboth{Appendix \thechapter}{}
\label{ch:git}
\section{Screenshot of Bitbucket repository containing R scripts, project documentation and associated links.  The active URLs on the following page may be clicked to view code and version history, etc.  All links were valid at 11 June 2018.}
\markboth{Appendix \thesection}{}
\includepdf[scale=0.8, pagecommand={\pagestyle{fancy}},pages=-]{sections/carlhiggs_bca_rp2_scripts_Bitbucket.pdf}


% Code
% \chapter{R scripts}
% \markboth{Appendix \thechapter}{}
% \label{ch:aDo}
% \section{Simulation}
% \markboth{Appendix \thesection}{}
% \includepdf[scale=0.75, pagecommand={\pagestyle{fancy}},pages=-]{sections/scripts/corr_power_CH.pdf}
% \section{Time testing}
% \label{sec:aDoTime}
% \thispagestyle{empty}
% \markboth{Appendix \thesection}{}
% \includepdf[scale=0.75, pagecommand={\pagestyle{fancy}},pages=-]{sections/scripts/corr_timetest.pdf}
% \section{Power plots}
% \label{sec:aDoPlots}
% \thispagestyle{empty}
% \markboth{Appendix \thesection}{}
% \includepdf[scale=0.75, pagecommand={\pagestyle{fancy}},pages=-]{sections/scripts/corr_power_plots.pdf}
% \section{Shiny app server}
% \label{sec:rshinyserver}
% \thispagestyle{empty}
% \markboth{Appendix \thesection}{}
% \includepdf[scale=0.75, pagecommand={\pagestyle{fancy}},pages=-]{sections/scripts/server.pdf}
% \section{Shiny app UI}
% \label{sec:rshinyui}
% \thispagestyle{empty}
% \markboth{Appendix \thesection}{}
% \includepdf[scale=0.75, pagecommand={\pagestyle{fancy}},pages=-]{sections/scripts/ui.pdf}
\end{appendix}
			