\documentclass[graybox,envcountchap,sectrefs,vecarrow]{svmono}

% to create a note \todo{rephrase sentence}  or
			% \todo[color=green!40]{And a green note}
			% \todo[inline]{A todonote placed in the text}
			% %the above can be used before the end of a figure to advise to update
			% \todo[inline, size=\tiny]{Note with very small font size.}
			% \missingfigure[figcolor=white]{Testing figcolor}

% A4 paper
\paperwidth 210mm
\paperheight 297mm
\pdfpagewidth=\paperwidth
\pdfpageheight=\paperheight

%Remove Springer Logo
\usepackage{xpatch}

\makeatletter
\xpatchcmd{\@maketitle}{{\Large Springer\par}}{}{}{}
\makeatother

%extra url help

% Links and colours

\usepackage[english]{babel}
	\usepackage[pdftex]{hyperref}
	\usepackage{xcolor}
	\hypersetup{
		colorlinks,
		linkcolor={black},
		citecolor={black},
		urlcolor=[RGB]{56 108 176}
	}

% Bibliography
	% \usepackage{natbib,har2nat} 
	% \setcitestyle{authoryear,open={(},close={)}}
	% \bibliographystyle{spmpsci.bst} 
	\usepackage[numbers]{natbib}
	
% Fonts, lists, symbols and matrices
  \usepackage[T1]{fontenc}
  % \usepackage[utf8]{inputenc}
	\usepackage{enumitem}
	\usepackage{graphicx}
  % \usepackage{mathptmx}
  \usepackage{newtxtext,newtxmath}
  \usepackage{amssymb}
  \usepackage{amsmath}
	% \usepackage[retainorgcmds]{IEEEtrantools}
	\usepackage{multirow}
	\usepackage{rotating}
	\usepackage{changepage}
	\usepackage{dcolumn}
	\usepackage{booktabs}
	\usepackage{longtable}

\usepackage{csquotes}
% \usepackage{lmodern}% http://ctan.org/pkg/lm
\usepackage{bm}
\usepackage{tikz}
\usepackage{listings}
	
%geometry for electronic version:
	\usepackage[
	  outer=30mm,
	  inner=30mm,
	  vmargin=20mm,
	  includehead,
	  includefoot,
	  headheight=15pt,
	]{geometry}

%% for printing, use these margins! *********************************************
%	\usepackage[
%	  outer=25mm,			
%	  inner=35mm,
%	  vmargin=20mm,
%	  includehead,
%	  includefoot,
%	  headheight=15pt,
%	]{geometry}
  
  
%Headers and Footers
	\usepackage{fancyhdr}
	\pagestyle{fancy}
	% \fancyhf{}
	%\renewcommand{\headrulewidth}{1pt}    %remove top horizontal bar
	% %\fancyheadoffset{0\textwidth}        % center place top horizontal bar?        
	% \renewcommand{\footrulewidth}{0pt}    %remove bottom horizontal bar
	% \usepackage{lastpage}
	% \rightmark
	 % % \rhead{Student 659810}
	 % % \lhead{}
	\cfoot{}
	% % \cfoot{Page \thepage\ of \pageref{LastPage}}
	\fancyhead[L]{\nouppercase{\leftmark}}
	\fancyhead[R]{\thepage}
	\renewcommand{\chaptermark}[1]{\markright{#1}}
	
	\usepackage{pdflscape}
	
%formatting miscellanies
	\renewcommand\thepart{\Alph{part}}
	\usepackage{capt-of}
	% \addtolength{\hoffset}{-0.5cm}
	% \addtolength{\textwidth}{1cm}

	\usepackage{setspace} %For double spacing


	\usepackage{todonotes} %For to do notes

	% \usepackage[font=large]{caption} %make captions larger (doesn't seem to work)
	
\usepackage{helvet}
\usepackage{courier}
%
\usepackage{type1cm}         

\usepackage{makeidx}         % allows index generation

\usepackage{multicol}        % used for the two-column index
\usepackage[bottom]{footmisc}% places footnotes at page bottom



	\setlength{\parindent}{0em}  %remove all indenting of paragrapgs
	% \setlength{\parskip}{1em} 
	\usepackage{pdfpages} 
	\usepackage{upgreek}

% %For correction positioning of bibliography
% \usepackage{bookmark}
% \bookmarksetup{open,numbered}% I like numbered bookmarks

\usepackage{appendix}    % puts the word 'Appendices' in the ToC
	
%Not sure what this is
		% \makeatletter
		% \newcommand*{\gmshow@textheight}{\textheight}
		% \newdimen\gmshow@@textheight
		% \g@addto@macro\landscape{%
		  % \gmshow@@textheight=\hsize
		  % \renewcommand*{\gmshow@textheight}{\gmshow@@textheight}%
		% }
		% \def\Gm@vrule{%
		  % \vrule width 0.2pt height\gmshow@textheight depth\z@
		% }%
		\makeatother
		% \setlength{\parindent}{0pt}


%Custom commands
\renewcommand{\vec}[1]{\bm{#1}}	
	\newcommand{\tdn}{\todo[inline, color=green!40]}
  % \newcommand{\E}{\mathrm{E}}
  \newcommand{\Var}{\mathrm{Var}}
  \newcommand{\Cov}{\mathrm{Cov}}
  \newcommand*{\lcd}{\raisebox{-0.25ex}{\scalebox{1.2}{$\cdot$}}}
  \newcommand{\code}[1]{\texttt{#1}}
% Custom Colours
	\definecolor{farb1}{rgb}{27,158,119}
	\definecolor{farb2}{rgb}{217,95,2}
	\definecolor{farb3}{rgb}{117,112,179}
	\definecolor{farb4}{rgb}{231,41,138}
	\definecolor{farb5}{rgb}{102,166,30}
	\definecolor{farb6}{rgb}{230,171,2}
	\definecolor{farb7}{rgb}{166,118,29}
	\definecolor{farb8}{rgb}{102,102,102}

  \definecolor{rgr}{RGB}{28,144,153}
  \definecolor{commentgreen}{RGB}{2,112,10}
  
% Custom math operators
  \DeclareMathOperator\arctanh{arctanh}
  \DeclareMathOperator\tanh{tanh}
  \DeclareMathOperator\ci{CI}
  \DeclareMathOperator\abs{abs}
  \DeclareMathOperator\power{power}

% Custom R code settings
\lstset{language=R,
    basicstyle=\linespread{0.6}\small\ttfamily,
    frame=tb,
    stringstyle=\color{commentgreen},
    % otherkeywords={0,1,2,3,4,5,6,7,8,9},
    morekeywords={TRUE,FALSE},
    deletekeywords={data,frame,length,as,character},
    keywordstyle=\color{blue},
    commentstyle=\color{commentgreen},
}
% \lstset{ %
  % language=R,                % the language of the code
  % basicstyle=\footnotesize,           % the size of the fonts that are used for the code
  % numbers=left,                   % where to put the line-numbers
  % numberstyle=\tiny\color{gray},  % the style that is used for the line-numbers
  % stepnumber=2,                   % the step between two line-numbers. If it's 1, each line 
                                  % % will be numbered
  % numbersep=5pt,                  % how far the line-numbers are from the code
  % backgroundcolor=\color{white},      % choose the background color. You must add \usepackage{color}
  % showspaces=false,               % show spaces adding particular underscores
  % showstringspaces=false,         % underline spaces within strings
  % showtabs=false,                 % show tabs within strings adding particular underscores
  % frame=single,                   % adds a frame around the code
  % rulecolor=\color{black},        % if not set, the frame-color may be changed on line-breaks within not-black text (e.g. commens (green here))
  % tabsize=2,                      % sets default tabsize to 2 spaces
  % captionpos=b,                   % sets the caption-position to bottom
  % breaklines=true,                % sets automatic line breaking
  % breakatwhitespace=false,        % sets if automatic breaks should only happen at whitespace
  % title=\lstname,                   % show the filename of files included with \lstinputlisting;
                                  % % also try caption instead of title
  % keywordstyle=\color{blue},          % keyword style
  % commentstyle=\color{dkgreen},       % comment style
  % stringstyle=\color{mauve},         % string literal style
  % escapeinside={\%*}{*)},            % if you want to add a comment within your code
% }  
  
  
%establish correct inset  style for appendices	
\makeatletter
 \newcommand\chapassect{%
\def\@chapter[##1]##2{\ifnum \c@secnumdepth >\m@ne
                       \if@mainmatter
                         \refstepcounter{chapter}%
                         \typeout{\@chapapp\space\thechapter.}%
                         \addcontentsline{toc}{section}%
                                   {\protect\numberline{\thechapter}##1}%
                       \else
                         \addcontentsline{toc}{section}{##1}%
                       \fi
                    \else
                      \addcontentsline{toc}{section}{##1}%
                    \fi
                    \chaptermark{##1}%
                    \addtocontents{lof}{\protect\addvspace{10\p@}}%
                    \addtocontents{lot}{\protect\addvspace{10\p@}}%
                    \if@twocolumn
                      \@topnewpage[\@makechapterhead{##2}]%
                    \else
                      \@makechapterhead{##2}%
                      \@afterheading
                    \fi}%
}
\makeatother
	
\begin{document}

% \listoftodos

% Title page
% �	Include: Title of topic
% �	Your name
% �	Department
% �	University
% �	Course/subject code and name
% �	Date

\author{Carl Higgs \\ 
		\large{School of Population and Global Health \\ 
		The University of Melbourne}}

\title{Power to detect a difference in correlations \\ 
		in identical and non-identical twins}
\subtitle{Simulation study and power calculator utility \\
		\vspace{4.5cm}
		\large{
		\\
		Research project in partial fulfilment of the degree\\
		Master of Biostatistics\\
		  \\
		Supervisors: \\
		Enes Makalic \\
		Elasma Milanzi \\
		Katrina Scurrah
		 }}%


\date{June 2018}

\maketitle%

%% PLAN
% SECTION     Prop. Words
% background 	0.3	   1100 x
% methods    	0.2	    800 0.5
% results    	0.1	    500 0.1
% discussion 	0.3	   1200 0.3
% conclusion 	0.1	    400 0.15


\frontmatter%%%%%%%%%%%%%%%%%%%%%%%%%%%%%%%%%%%%%%%%%%%%%%%%%%%%%%
% Acknowledgements


\begin{dedication}
I gratefully acknowledge the advice and support from my supervisors Elasma, Enes and Katrina through the semester.  In addition, the contribution of advice from my colleague Koen Simons on techniques for improving the efficiency of my code was much appreciated.  Thank you Elasma, Enes, Katrina and Koen.
\index{acknowledgements} 
\end{dedication}

	


\include{sections/supervisor}
\section*{Declaration}

This is to certify that
\begin{enumerate}
	\item the report comprises only my original work towards the MBiostat, under the guidance of my project supervisors,
	\item due acknowledgement has been made in the report to all other material used,
  \item this work has not been previously submitted for academic credit,
	\item the report is less than 4,000 words in length, exclusive of tables, figures, table of contents, bibliography and appendices.
\end{enumerate}
\vspace{5cm}
% Ideally, this length should be changed so the line
% is symmetrically spaced with the text beneath it.
\underline{\hspace{7.52cm}} \par
Carl Higgs, July 2018 \par
\vfill\null
%%	
% Research question:
% What are the existing methods for estimating power to detect a difference in correlations between identical (monozygotic) and non-identical (dizygotic) twins, how do these compare and can they be improved upon? 
\section*{Abstract}


\subsection*{Background}
A power analysis for the detection of differences in correlations between identical and non-identical twin pairs was undertaken to support researchers in the early stages of a classic twin study.

\subsection*{Methods}
\tdn{This question will be addressed through a literature review, and comparison of methods using both theory and simulation.}

\subsection*{Results}

\subsection*{Conclusion}  


\tableofcontents
%%%%%%%%%%%%%%%%%%%%%%acronym.tex%%%%%%%%%%%%%%%%%%%%%%%%%%%%%%%%%%%%%%%%%
% List of acronyms
%%%%%%%%%%%%%%%%%%%%%%%% Springer %%%%%%%%%%%%%%%%%%%%%%%%%%

\extrachap{Acronyms}

\begin{description}[CABR]
\item[DZ]{Dizygotic twins}
\item[ICC]{Intra-class correlation coefficient}
\item[FZ]{Fisher's Z (test)}
\item[GV]{Generalised Variable (test)}
\item[MSLR]{Modified signed log-likelihood ratio (test)}
\item[MZ]{Monozygotic twins}
\item[PT]{Permutation test}
\item[SLR]{Signed log-likelihood ratio (test)}
\end{description}

\mainmatter%%%%%%%%%%%%%%%%%%%%%%%%%%%%%%%%%%%%%%%%%%%%%%%%%%%%%%%
\doublespacing
\chapter*{1. Introduction}
\markboth{1. Introduction}{}
\setcounter{chapter}{1}
\addcontentsline{toc}{chapter}{1. Introduction}

			% o	Summary of background and research methods; why this analysis/ evaluation is important; how and why the data was collected (initial study aims, collection methods, response rates); specifics of data (number, variables, outcomes)
			% Research question, aim, objectives of study
			
			% Research question:
			% What are the existing methods for estimating power to detect a difference in correlations between identical (monozygotic) and non-identical (dizygotic) twins, how do these compare and can they be improved upon? 

\large


% \begin{figure}[htbp]
% \sidecaption[t]
% %\centering
% \fbox{\includegraphics[scale=0.5]{figures/PapuaEndemicity.png}}
% %\picplace{5cm}{2cm} % Give the correct figure height and width in cm
% \caption{caption for figure.}
% \label{fig:pvpr}       % Give a unique label
% \end{figure}

\section{Background}
%% Intro
The classic twin study exploits the differing degrees of genetic relatedness in identical and non-identical twins in order to draw inferences on the heritability of traits.  The calculation and comparison of Pearson correlations across the two twin groups is a routine preliminary step undertaken by researchers in this field.  However, a range of factors can impact on researcher's ability to detect a true effect given data.  This thesis reports on a simulation-based power analysis for the detection of differences in correlations between identical and non-identical twin pairs under a range of scenarios, and the associated development of R functions and an applied interactive power calculator.  These tools address an identified absence of tools for this fundamental step in the twin study context.
 
\subsection{Twins and the classic twin study}
Identical twins arise from the same zygote, or fertilised egg, and are genetically very similar.  Non-identical twins arise from fertilisation of two seperate eggs, and are as genetically alike as ordinary siblings. 

The classic twin study can be used to estimate the proportion of variation in traits which may be attributable to genetics, or heritability.  

The estimates of heritability are conditional on key assumptions.  This chapter will define key genetic concepts, how these are exploited by the classic twin study to estimate the role of heridity for particular traits, and the assumptions which are relied upon in order to make such inferences.

The classic twin study exploits the differing degrees of genetic relatedness in identical and non-identical twins in order to draw inferences on the heritability of traits.  In broad terms, heritability is the degree to which variation in a trait or phenotype, such as propensity to gain body weight, or become a centenarian, can be attributed to genetics.

The comparison of phenotypic traits within identical and non-identical twin pair samples allows for partitioning the variance in traits into that attributable to  shared environment, individual environment or to genetics.  This allows us to better understand the mechanics of health and disease processes so that we can develop intervention measures which appropriately target the hypothesised causal mechanisms.

Contemporary twin studies use mixed effects and structural equation modelling to evaluate differences in variance components for a particular trait between mono and dizygotic twins accounting for differential within pair similarities related to zygosity.  A recent article reported on the development of R commands which can be used to estimate the power to detect a difference in correlations using such models.  

However, a routine preliminary step undertaken by researchers in this field is the calculation and comparison of Pearson correlations across the two twin groups.  This research project focuses on the reported needs of researchers undertaking this early analysis step.

In undertaking a twin study we make certain assumptions, understanding that these likely do not strictly hold in practice, but the key one for the purposes of this power analysis is that our data is approximately normally distributed.

\subsection{Historical background}
The statistical treatment of correlation was popularised by Francis Galton.  In context of broad social interest in eugenics, Galton described methods which could be used to describe the 'co-relatededness' of variables sourced from closely related family members \cite{Galton1888,Galton1890}.  

Karl Pearson was a keen follower of Galton's research, and writing in the context of inheritance and natural selection made use of 'Galton's function', describing it as a coefficient of correlation \cite{Pearson1895}.  

Ronald Fisher observed that a geometric transformation of the correlation cofficient using its inverse hyperbolic tangent could be used to approximate a normal distribution \cite{Fisher1915}.  This has the effect of mapping the distribution of correlation of coefficients from a domain of negative 1 through positive one to negative infinity through positive infinity. Being a simple and accurate approximation of the normal distribution, this transformation known as Fisher's Z is ubiquitous in statistical treatment of correlation coefficients, for example when seeking to compare their differences.

Florence Nightingale David was a protege of Pearson's who suggested that she prepare a volume of numerically accurate tables and interpolated plots of the distribution of the sample correlation coefficient r given n and the population correlation rho which could act as a standard against which to judge approximations such as that of Fisher \cite{David1938}. 
%  This visualisation of David's of the chance of rejecting the null hypothesis when true given alpha and rho and n was an influence on this project's presentation of results.

In the context of genetics and kinship, Douglas Falconer outlined methods of using comparison of identical and non-identical twins for estimating heritability, and noted some of the assumptions that this involves \cite{Falconer1960}.

Michael Neale and colleagues developed methods and software to facilitate the analysis of variance components in twin studies using structural equation modelling \cite{Neale1992}.  Brad Verhulst, building on the work of Peter Visscher, developed functions for power analyses in this variance component modelling context, for example detecting a difference in genetic correlations \cite{Visscher2004,Visscher2008a}.

This project is concerned with power to detect a difference in Pearson correlations as part of preliminary analysis before variance component modelling.

\subsection{Power analysis}
Power analysis involves a compromise between type 1 and type 2 error thresholds, respectively the expected proportion of null hypotheses to be rejected when true and not rejected when false.  These could be chosen to suit the requirements of a particular study, but for historical reasons the usual consensus is for 5\% and 20\%, and this is the parameterisation adopted for the results presented here.

A test statistic for the difference in correlations can be calculated as the difference in Fisher's Z transformed values weighted by the approximate standard error of the difference.  Fisher's Z transformation, the inverse hyperbolic tangent, maps the correlation coefficient from a domain of -1 through 1 to negative infinity through positive infinity, with approximate normal distribution.  This classic formulation is still used in functions found in Stata and R.

To estimate the power using this approach, you first take the difference between a normal reference score given the chosen type 1 error rate and the absolute value of the test statistic. The type 2 error rate is the probability of observing a value of at least this magnitude on the normal distribution.  And 1 minus this value is the power.

Using a simulation approach we take our hypothesis tests and apply them to draws from simulated data designed to mimic our samples through parameterisation using the hypothesised underlying bivariate population distributions.  

So where in our formula we might plug in anticipated or observed coefficients of .2 and .5, in the simulation we use these values to represent the supposed true correlations in the underlying population from which we draw our random samples.  Over a large number of simulations the proportion of hypothesis tests returning p-values lower than our type 1 error threshold is our power estimate.

 
 
%% Power
Power is the probability of detecting a true effect, a key consideration when planning a study.  In the context of differences in correlations undertaken in a twin study, we will expect a range of possible parameters to influence our power: the sample size; the ratio of MZ to DZ twins; how we have defined what constitutes a meaningful difference; the magnitude of the respective correlations; how these are measured.

Through my research project, I planned and carried out a power analysis for the detection of differences in correlations between identical and non-identical twin pairs investigating the influence of such factors.  

 \subsubsection{subsub section title}
 

			
			
\chapter*{2. Methods}
\markboth{Methods}{}
\setcounter{chapter}{2}
\setcounter{section}{0}
\addcontentsline{toc}{chapter}{2. Methods}

The preceding chapter summarised a review of the literature relating to twins, correlations, power and simulations undertaken to inform our approach to analysis.  Through this review we identified a series of suitable approaches and tests for evaluating differences in Pearson and Spearman correlations in two groups. Important considerations were the efficiency of our implemented simulation functions, and a well-designed data structure to support our planned as well as future outputs.  The $R$ programming environment was used for all analyses \cite{R2018}.

\section{Hypothesis tests for difference in correlations}
\subsection{Fisher's Z test (analytical approach)}
A test statistic for the difference in correlations can be calculated as the difference in Fisher's Z transformed values weighted by the approximate standard error of the difference \cite{Fisher1990,David1938}
$$t_{\hat{\theta}} = {\frac{\hat{\theta}}{se_\hat{\theta}}} = {\frac{z_{MZ} - z_{DZ}}{\sqrt{(n_{MZ}-3)^{-1}+(n_{DZ}-3)^{-1}}}} $$

The type 2 error rate $\beta$ is estimated through comparison of this test statistic $t_{\hat{\theta}}$ to a reference value on the standard normal distribution.   Using the cumulative normal distribution function $\Phi$ (Phi), the reference score $q$ is calculated as the normal probability quantile corresponding to our $\alpha$ level divided by the sidedness of our test. 
 $$q = \Phi^{-1}(\alpha/\text{sidedness})$$
Where we refer to sidedness, we mean whether we are concerned with a single- or two-tailed probability.  Here, we are testing the hypothesis that $\rho_{MZ} = \rho_{DZ}$ using a two-tailed p-value, implying 'admissible alternatives' to be the case that $\rho_{MZ}$ is greater than or less than $\rho_{DZ}$, that is, $|\rho_{MZ} - \rho_{DZ}| > 0$ \cite{David1938}.  One-tailed consideration, for example that $\rho_{MZ} > \rho_{DZ}$, is not considered in this report however, the functions developed may be parameterised in this way if desired. 
\\
Employing the concepts detailed above, $\beta$ is calculated as 
$$\beta = \Phi \big( q - t_\theta \big)$$
Our power estimate for the detection of difference in correlations is $\power(\theta) = 1 - \beta$. Putting the above altogether, we calculate power using the Fisher's $z$ test statistic as,

$$ \power = 1- \Phi \Bigg(\Phi_{\alpha/2}^{-1} -  \abs\bigg(  \frac{\arctanh(r_{MZ}) - \arctanh(r_{DZ})}{\sqrt{(n_{MZ}-3)^{-1} + (n_{DZ}-3)^{-1}}}  \bigg)  \Bigg) $$ 

The code we used to implement the analytic Fisher's Z test approach to power calculation in $R$ is displayed in listing \ref{lst:fz_nosim}.

\begin{lstlisting}[float=h,caption={Fisher's Z test (analytic approach)},label={lst:fz_nosim}]
# Fishers Z test - no sim
fz_nosim <- function(r1,r2,n1,n2,
                     alpha = 0.05, sidedness=2,method = "pearson",
                     power = TRUE) {
  # Calculate Fisher's Z
  z1     <- atanh(r1)
  z2     <- atanh(r2)
  
  # Take difference
  zdiff  <- z1-z2
  
  # Calculate standard error and test statistic
  z_se   <- sqrt(1/(n1-3) + 1/(n2-3))
  z_test <- zdiff/z_se
  
  # Optionally return p-value for observing diff at least this large under H0
  z_p    <- sidedness*pnorm(-abs(z_test))
  if (power == FALSE) return("p" = z_p)
  z_ref   <- qnorm(1-alpha/sidedness)
  z_power <- 1-pnorm(z_ref - abs(z_test))
  return(z_power)
}
\end{lstlisting}

The above method is the de facto standard, as used for example in the Stata \code{power two correlations} \cite{StataCorp2013}.  However, other options for evaluating the difference in Pearson or Spearman correlations should be considered.  

\subsection{Fisher's Z test (simulation approach)}
Using a simulation approach we take our hypothesis tests and apply them to draws from simulated data designed to mimic our samples through parameterisation using the hypothesised underlying bivariate population distributions.  
\\
\\
So where in our formula we might plug in hypothesised sample coefficients of 0.2 and 0.5, in the simulation we use these values to represent the true correlations in the underlying population from which we draw our samples.  Over a large number of simulations of bivariate twin data the proportion of hypothesis tests returning p-values lower than our type 1 error threshold is our power estimate.
\\
\\
The simulation-based Fisher's Z test function $R$ code is displayed in listing \ref{lst:fz}.

\begin{lstlisting}[float=h,caption={Fisher's Z test (simulation approach)},label={lst:fz}]
# Fishers Z test
fz <- function(a,b,sidedness=2,method = "pearson") {
  # Two samples
  n1 <- nrow(a)
  n2 <- nrow(b)
   
  # Compute z-transformed sample correlation coefficients
  z1     <- atanh(cor(a,method = method)[2,1])
  z2     <- atanh(cor(b,method = method)[2,1])
  zdiff  <- z1-z2
  
  # calculate standard error and test statistic
  z_se   <- sqrt(1/(n1-3) + 1/(n2-3))
  z_test <- zdiff/z_se
  
  # return p-value
  z_p    <- sidedness*pnorm(-abs(z_test))
  return(z_p)
}
\end{lstlisting}


In addition to applying the Fisher Z test in a simulation context, alternate tests we identified and implemented for inclusion in our simulation study were as follows. 

\subsection{Zou's confidence interval}
Zou's confidence interval is used to calculate a confidence interval for the difference in two correlations, and a hypothesis test employing this method is featured in the R package \code{cocor} \cite{Zou2007,Diedenhofen2015}.  A hypothesis test using Zou's confidence interval evaluates whether zero lies within the lower and upper bounds of the interval estimate of the difference in correlations, returning 1 if so or otherwise zero.  Over a run of simulations this would be expected to return identical results to the Fisher Z test, but may be more efficient.
\\
\\
Zou's approach expands on earlier work \cite{Olkin1995} to calculate a confidence interval for a difference in correlations using a so-called Simple Asymptotic approach, using what Zou refers to as a Modified Asymptotic method \cite{Zou2007}.  Both approaches draw heavily on Fisher's earlier work \cite{Fisher1990}.  The modified asymptotic method of Zou consists of first calculating confidence intervals for the two respective z-transformed correlations (transformed as per Fisher's method, described above):
$$(l_{z_k}, u_{z_k}) = z_k \pm \sqrt{\frac{1}{n_k - 3}} \times \Phi_{\alpha/2}^{-1},\ \text{where} \ k \in \{1,2\}$$

Then, the lower (L) and upper (U) bounds of the modified asymptotic confidence interval for the difference in correlations are calculated:
$$L = r_1 - r_2 - \sqrt{(r_1 - \tanh(l_{z_1}))^2 + (\tanh(u_{z_2})- r_2)^2}$$
$$U = r_1 - r_2 + \sqrt{(\tanh(u_{z_1}) - r_1)^2 + (r_2 - \tanh(l_{z_2}))^2}$$

If zero is within the bounds of the confidence interval for the difference, the test returns as 1, and otherwise 0.
\\
\\
Our implemenation of the Zou's confidence interval test function is displayed in listing \ref{lst:zou}.

\begin{lstlisting}[float=h,caption={Zou's confidence interval},label={lst:zou}]
zou <- function(a,b,alpha = 0.05,sidedness=2,method = "pearson") {
  # From Zou (2007) and used in Cocor (note typo for U in paper; should be '+')
  #  However, really, this is equivalent to fz test for hypothesis testing purposes
  
  # compute z- transformed correlations and differences
  r  <- c(cor(a,method = method)[2,1], cor(b,method = method)[2,1])
  z  <- atanh(r)
  zdiff  <- z[1]-z[2]
  
  # calculate standard error for respective z scores
  n  <- c(nrow(a),nrow(b))
  z_se   <- sqrt(1/(n-3))

  # calculate reference threshold
  z_ref  <- qnorm(1-alpha/sidedness)
  
  # calculate respective confidence intervals
  ci_mat <- matrix(c(-1,-1,1,1),nrow = 2, ncol = 2, dimnames =list(c("Mz","Dz"),c("l","u")))
  z_ci   <- z + ci_mat * z_se * z_ref
  r_ci   <- tanh(z_ci)
  
  # calculate Zou's Modified Asymptoptic confidence interval for difference in correlations
  L      <- r[1]-r[2] - sqrt((r[1]      - r_ci[1,1])^2 + (r_ci[2,2] - r[2]     )^2)
  U      <- r[1]-r[2] + sqrt((r_ci[1,2] - r[1]     )^2 + (r[2]      - r_ci[2,1])^2)
  r_diff_ci <- c(L,U)
  
  # return test value (0 or 1, however, in the power context this resolves to same outcome as p)
  ci_test <- (L < 0) && (0 < U)
  return(c(ci_test,r_diff_ci))
}
\end{lstlisting}



\subsection{Generalised Variable Test}

The generalised variable (GV) test involves transformation of the simulated sample correlations into so-called pivotal quantities the difference of which is used to calculate a test statistic and p-value \cite{Krishnamoorthy2014}. Synthesising two reported approaches \cite{Krishnamoorthy2007,Kazemi2016} this test was first implemented as an example by my supervisor Enes Makalic in a Matlab script, and subsequently adapted by myself as a function in R.  A compiled version using RCPP to leverage C++ routines for random number draws was suggested by my colleague Koen Simons, and adopted to improve the function's run time. However, this later version was not compatible with the parallelised simulation approach, and in this context the non-RCCP 'GVT-r' version was used.
\\
\\
Given two bivariate normal samples $k\in\{1,2\}$, the sample correlation coefficients $r_k$ are used to estimate two respective quantities $r_k^* = \frac{r_k}{\sqrt(1-r_k^2)}$, and the generalised variables $G_{\rho_k}$:
$$G_{\rho_k} = \frac{r_k^*\sqrt{W_k} - U_k}{\sqrt{(r_k^*\sqrt(W_k) - U_k)^2 + V_k}}$$
where,
$$U_k \sim N(0,1) ,\ V_k \sim \chi_{n_k - 1}^2 ,\ \text{and} \ W_k \sim \chi_{n_k-2}^2$$
A p-value using the GV test is calculated as twice the value of the smaller of two quantities: the proportion of differences in $G_{\rho_k}$ less than 0, and the proportion greater than 0.

The GV test function $R$ code is displayed in listing \ref{lst:gvtr}.

\begin{lstlisting}[float=h,caption={GV test (R version)},label={lst:gvtr}]
gvt_r <- function(a,b,M=1e4,method = "pearson") {
  # Two samples
  n1 <- nrow(a)
  n2 <- nrow(b)
  
  # Compute sample correlation coefficients
  r1 <- cor(a,method = method)[2,1]
  r2 <- cor(b,method = method)[2,1]
  r  <- c(r1,r2)
  
  # Generate random numbers
  V2     <- matrix(data=0, nrow = M, ncol = 2)
  V2[,1] <- rchisq(M, df = n1-1, ncp = 1)
  V2[,2] <- rchisq(M, df = n2-1, ncp = 1)
  
  W2     <- matrix(data=0, nrow = M, ncol = 2)
  W2[,1] <- rchisq(M, df = n1-2, ncp = 1)
  W2[,2] <- rchisq(M, df = n2-2, ncp = 1)
  
  Z <-matrix(data = rnorm(2*M), nrow=M, ncol = 2)
  
  # Compute test statistic
  rstar <- r/sqrt(1-r^2)
  top   <- c(sqrt(W2[,1])*rstar[1],sqrt(W2[,2])*rstar[2]) - Z
  G     <- top / sqrt( top^2 + V2 )
  
  # Compute p value
  Grho <- G[,1] - G[,2];
  p    <- 2*min( mean(Grho<0), mean(Grho>0) ); 
  return(p)
}
\end{lstlisting}

\subsection{Signed log-likelihood ratio test}
The signed log likelihood ratio (SLR) test is formulated as the signed difference in sample correlation coefficients multiplied by the square root of the sum of respective coefficients' log-likelihoods.  The test here is a partial implementation of a recently reported modified signed log-likelihood ratio (MSLR) test  for differences in two bivariate normal correlations \cite{Kazemi2016}. The SLR and MSLR tests are well established general hypothesis tests \cite{Barndorff1986,Barndorff1991,Diciccio2001,Krishnamoorthy2014}, the novelty in Kazemi and Jafari's approach being the applied context of difference in correlations. However, we (myself, nor my supervisors) were unable to successfully replicate the 'modified' portion of Kazemi and Jafari's reported algorithm.  Due to time constraints, and noting that the 'unmodified' SLR test appeared to return p-values similar to the other hypothesis tests it was decided that inclusion of the SLR test would be a valid option to consider.
\\
\\
The SLR test function $R$ code is displayed in listing \ref{lst:slr}.

\begin{lstlisting}[float=h,caption={Signed log-likelihood ratio test},label={lst:slr}]
slr <- function(a,b,M=1e4,sidedness=2,method = "pearson") {
  # Signed Log-likelihood Ratio test (an 'unmodified' version of test 
  # described in Krishnamoorthy and Lee, Kazemi and Jafari , DiCiccio etc)
  # Two samples
  n  <- c(nrow(a),nrow(b))
  
  # Compute z-transformed sample correlation coefficients
  r  <- c(cor(a,method = method)[2,1], cor(b,method = method)[2,1])
  z  <- atanh(r)
  
  # Calculate average z as a plug in value
  rf <- tanh(mean(z))
  
  # calcaulte SLR
  slr <-sign(r[1]-r[2])*sqrt(sum(n*log(((1-rf*r)^2)/((1-r^2)*(1-rf^2)))))
  
  # return p-value
  p    <- 2 * (1 - pnorm(abs(slr))); 
  return(p)
}
\end{lstlisting}

\subsection{Permutation test}
The permutation test is a non-parametric approach which compares the absolute 
difference of the Z-transformed sample correlations with those using correlations 
from a series of group membership permutations using the sample rank orders as 
values.  Under a hypothesis of no difference in correlation, those differences arising from permutations would be assumed to be equally likely as those observed, or anticipated to be observed \cite{Efron1993}.  Across a series of $M$ permutations (in this study, 10,000), a $p$-value is calculated as the proportion of permutation derived absolute differences (($\abs(z_{MZ}^* - z_{DZ}^*)$)) of greater magnitude than $\abs(z_{MZ} - z_{DZ})$.
\\
\\
The implementation of this permutation test in $R$ is displayed in listing \ref{lst:pt}.

\begin{lstlisting}[float=h,caption={Permutation test},label={lst:pt}]
pt <- function(a,b,M=1e4,sidedness=2,method = "pearson") {
  # Based on Efron and Tibshirani, 1993
  # Store size, and calculate z-transformed correlations
  n  <- c(nrow(a),nrow(b))
  r  <- c(cor(a,method = method)[2,1], cor(b,method = method)[2,1])
  z  <- atanh(r)
  
  # Store rank-ordered vector representations, in one column
  v  <- cbind(rank(rbind(a[,1],b[,1]),ties.method = "random"),
              rank(rbind(a[,2],b[,2]),ties.method = "random"))
  # label rows
  rownames(v) <- c(rep("A",n[1]),rep("B",n[2]))
  
  # initial empty test vector
  rtest <- numeric(0)
  
  # run M permutations (default is 10,000),
  #  - returns test that absolute magnitude of difference
  #    is at least as great as that of the input z-transformed corr. diff.
  for (i in 1:M){
    permute <- cbind(v,rbinom(sum(n),1,0.5))
    rstar   <- c(cor(permute[permute[,3]==0,c(1,2)],method = method)[2,1],
                 cor(permute[permute[,3]==1,c(1,2)],method = method)[2,1])
    zstar   <- atanh(rstar)
    rtest   <- c(rtest,
                 abs(zstar[1]-zstar[2]) > abs(z[1]-z[2]))
    } 
  
  # return p-value: proportion of test results at least as large as obs'd
  p <- mean(rtest)
  return(p)
}
\end{lstlisting}

\section{Simulation}
\subsection{Approach for one simulation}
A function \code{corr\_diff\_test()} was developed to undertake a single comparative simulation of any of the above tests (listing \ref{lst:corr_diff}). Within a single simulation, each simulation based  test is evaluated using the same samples drawn from the two simulated bivariate populations as input, returning a $p$-value.  The analytic Fisher Z test returns either a $p$-value or a power estimate based on the population parameters.  Note that in the detailing of computational aspects of our methodology we use the term 'parameter' to refer to one of the options which may be specified within a function; an argument is the specific value which is passed to that parameter. For example, the (population) parameter $\rho$ for the respective MZ and DZ groups may be defined by specifying the argument \code{rho = c(-0.65,0.2)}. The main parameters which can be specified in the function call to \code{corr\_diff\_test()} are described with example arguments in table \ref{table:corr_params}.  In the following text we refer to the set of parameters that give rise to a simulation as a scenario.

\begin{lstlisting}[float=h,caption={Single run simulation code},label={lst:corr_diff}]
corr_diff_test <- function(rho = c(.2,.5), n = c(30,90), distr = "normal",
                    param1a = c(0,0), param1b = c(0,0),param2a = c(1,1), param2b = c(1,1),
                    alpha = 0.05, sidedness = 2, test = c("fz","gtv","pt","slr","zou"),
                    method ="pearson", lower.tri = FALSE) {
  if(lower.tri==TRUE){
    # optionally, only calculate results for lower matrix half 
    #   when comparing across all correlation combinations
    if(rho[1] < rho[2]) { 
      return(NA)
    }
  }
  # initialise empty results vector
  results <- list()
  
  # if requested, process analytical Fisher's Z
  if ("fz_nosim" %in% test) {
    results[["fz_nosim"]] <- fz_ns_compiled(rho[1],rho[2],n[1],n[2], 
                                      alpha = 0.05, sidedness = 2, method = method, power = FALSE)
    if(length(test)==1) return(results)
  }
  # process selected hypothesis tests, each using same draw of simulated data
  require("simstudy")
  a <- genCorGen(n[1], nvars = 2, params1 = param1a, params2 = param2a,  
                dist = distr, corMatrix = matrix(c(1, rho[1], rho[1], 1), ncol = 2), 
                wide = TRUE)[,2:3]
  b <- genCorGen(n[2], nvars = 2, params1 = param1b, params2 = param2b,  
                dist = distr, corMatrix = matrix(c(1, rho[2], rho[2], 1), ncol = 2), 
                wide = TRUE)[,2:3]
  if ("fz"       %in% test) results[["fz"]]       <- fz_compiled(a,b)
  if ("gtv"      %in% test) results[["gtv"]]      <- gtv(a,b) # uses rccp ; so elsewise compiled
  if ("gtvr"     %in% test) results[["gtvr"]]      <- gtv_compiled(a,b) 
  if ("pt"       %in% test) results[["pt"]]       <- pt_compiled(a,b)
  if ("slr"      %in% test) results[["slr"]]      <- slr_compiled(a,b)
  if ("zou"      %in% test) results[["zou"]]      <- zou_compiled(a,b)[1]
  return(rbind(results[test]))
}
\end{lstlisting}

The function uses the $R$ package \code{simstudy} function \code{genCorGen} to generate bivariate correlated data for the simulated MZ (group \code{a} in the code above) and DZ (group \code{b}) twin pair samples \cite{simstudy2018}.  The choice of available distributions and parameterisations is normal($\mu,\sigma$), binomial(probability $p$), Poisson(rate $\lambda$) gamma($\mu,\text{dispersion} \ k$), or uniform(min, max).  In addition to specifying sample size, distribution and parameterisation, a correlation matrix may be specified; this was used to parameterise the underlying population correlations from which bivariate samples should be drawn.  Three distinct distribution types were modelled in our simulation based power analysis: normal, 'mild' skew and 'extreme' skew. These are respectively explained in the captions of Figures \ref{fig:dist_norm}, \ref{fig:dist_gamma1}, and \ref{fig:dist_gamma2}, which illustrate example sample draws from these distributions.  

\begin{figure}[htbp]
\sidecaption[t]
%\centering
\fbox{\includegraphics[scale=0.52]{{../figs/distx_normal_60_120}.pdf}}
%\picplace{5cm}{2cm} % Give the correct figure height and width in cm
\caption{An example of the bivariate normal scenario, with distributional assumptions asymptotically met.  Both variables are standardised with mean $\mu=0$ and standard deviation $\sigma=1$.}
 % and distribution $\sim N\bigg{\boldsymbol\mu = \begin{pmatrix} 0 \\ 0 \end{pmatrix}, \quad \boldsymbol\Sigma = \begin{pmatrix} 1 & \rho \\ \rho  & 1 \end{pmatrix} \bigg)$.
% \caption{An example of the bivariate normal scenario, under which our distributional assumptions are asymptotically met.  This was specified with both variables standardised having mean $\mu=0$ and standard deviation $\sigma=1$ and distribution $\sim N\bigg{\boldsymbol\mu = \begin{pmatrix} 0 \\ 0 \end{pmatrix}, \quad \boldsymbol\Sigma = \begin{pmatrix} 1 & \rho \\ \rho  & 1 \end{pmatrix} \bigg)$.}
\label{fig:dist_norm}       % Give a unique label
\end{figure}

\begin{figure}[htbp]
\sidecaption[t]
%\centering
\fbox{\includegraphics[scale=0.52]{{../figs/distx_gamma_mildskew_60_120}.pdf}}
%\picplace{5cm}{2cm} % Give the correct figure height and width in cm
\caption{A 'mild skew' scenario based on a gamma distribution with mean 1.5 and dispersion 0.09 (which the genCorGen function uses to inform shape and scale parameters for the distribution).  This parameterisation was chosen through experimentation with the intent to represent a mild departure from an assumed normal population distribution, with a slight positive skew}
\label{fig:dist_gamma1}       % Give a unique label
\end{figure}

\begin{figure}[htbp]
\sidecaption[t]
%\centering
\fbox{\includegraphics[scale=0.52]{{../figs/distx_gamma_extrskew_60_120}.pdf}}
%\picplace{5cm}{2cm} % Give the correct figure height and width in cm
\caption{An 'extreme skew' scenario based on a gamma distribution with mean 1 and dispersion 5.  This results in an extreme positive skew to the distribution, analogous to that of biological processes where most observations will be clustered around a certain value, however some outliers may be extremely elevated.}
\label{fig:dist_gamma2}% Give a unique label
\end{figure}


\begin{table}\centering
\caption{Description of parameter options for single simulation \label{table:corr_params}}
\begin{tabular}{cll}
  \toprule
  \textbf{Parameter} & \textbf{Description} & \textbf{Example arguments} \\ [0.5ex] 
  \midrule
  \code{method}    & Correlation method to use for testing difference                  & \code{'pearson'}                 \\
  \code{rho}       & $\rho$ for each group's bivariate distribution                    & \code{c(-0.21,0.59)}             \\
  \code{n}         & Sample size for groups 1 (MZ) and  2 (DZ)                         & \code{c(30,60)}                  \\
  \code{dist}      & Distribution to be used for both groups' bivariate distribution \ & \code{'normal'}                  \\
  \code{param1a}   & Distribution parameter 1 for respective samples in group 1        & \code{c(0,0)}                    \\
  \code{param1b}   & Distribution parameter 1 for respective samples in group 2        & \code{c(0,0)}                    \\
  \code{param2a}   & Distribution parameter 2 for respective samples in group 1        & \code{c(1,1)}                    \\
  \code{param2b}   & Distribution parameter 2 for respective samples in group 2        & \code{c(1,1)}                    \\
  \code{test}      & Tests to be evaluated                                             & \code{c("fz\_nosim","fz","gtv")} \\
  \code{alpha}     & $\alpha$ value to use for hypothesis tests                        & \code{0.05}                      \\
  \code{sidedness} & Sidedness for hypothesis tests                                    & \code{2}                         \\
  \bottomrule 
\end{tabular}
\end{table}

% whether only to compute results for where $rho_1 < rho_2$, which across a series of correlation combinations conceived as a matrix can be used to optionally returne only the lower triangular matrix of results.

\subsection{Approach for multiple simulations}
The function described above runs a single simulation of drawing from samples from two bivariate populations.  However, for asymptotic normality to hold --- the long run approximation of a normal distribution due to the Central Limit Theorem \cite{Casella2002} --- we know we must run many more simulations to achieve a fair assessment of the proportion of null hypotheses rejected when false for when using a particular hypothesis test under a particular scenario (ie. set of parameterisations).
\\
\\
A wrapper function \code{corr\_power} to allows for the \code{corr\_diff\_test()} function to be called $M$ times for a given scenario, returning a power estimate for each test specified in the function call.  These power estimates are derived from the series of simulated $p$-values for each test considered, or directly in the case of the analytic Fisher's Z test.

\section{Scenario combinations}
The flexibility of the simulation commands we defined allows for an very broad array of scenarios to be considered.  A researcher could use these tools as is to aid in the development of a statistical analysis plan for their study.  For this report we had to decide on a limited subset of these possibilities.  There are two drivers for the choice made: the number of scenario combinations, and the time taken to run each of these.  Our initial plan for the series of scenarios is reported in table \ref{table:combos}.
\\

\begin{table}\centering
\caption{Description of parameter options for single simulation \label{table:combos}}
\begin{tabular}{llr}
  \toprule
  \textbf{Parameter} & \multicolumn{2}{c}{\textbf{Initial plan}} \\ 
  \cmidrule(lr){2-3} 
   & \textbf{Argument resolution} & \textbf{Combinations}	\\
  \midrule
  \code{method}    & Pearson and Spearman correlations                                 & 2 \\
  \code{rho}       & $\rho$ combinations\: -0.95 through 0.95 at 0.5 resolution         & $39^2 = 1521$ \\
  \code{n}         & group combinations\: 15, 30, 60, 120, 240, 480, 960                      & $7^2 = 49$ \\
  \code{dist}      & normal, 'mild skew', 'extreme skew'                               &  3 \\
  \code{param1a}   & dictated by distribution choice, above                            & -  \\
  \code{param1b}   & dictated by distribution choice, above (equal to param1a)         & -  \\
  \code{param2a}   & dictated by distribution choice, above                            & -  \\
  \code{param2b}   & dictated by distribution choice, above (equal to param2a)         & -  \\
  \code{test}      & Fisher's Z (analytic and sim), Zou's CI, GVT, SLR, PT             & 6  \\
  \code{alpha}     & .05                                                               & 1  \\
  \code{sidedness} & 2                                                                 & 1  \\
  \midrule
  Total scenarios &                                                            & 2,683,044  \\
  Total simulations & 1,000 simulations for each scenario                   & 2,683,044,000 \\
  Total simulations & 100, 1,000 and 10,000 simulations for each scenario (unrealistic!) &  29,781,788,400 \\
  \bottomrule 
\end{tabular}
\end{table}

\\
We decided \textit{a prior} to set some parameters as fixed, : we only conducted two-sided tests with $\alpha$ of 0.05;  having decided to limit ourselves to three basic distributional forms (normal, and two kinds of non-normal using distinct gamma distribution parameterisations) the distribution parameters were fixed to achieve these forms; the respective simulations of MZ and DZ twin pair samples each use the same distributional form, although the population correlation and sample size may vary (e.g. we don't compare bivariate normal MZ with a gamma skewed DZ); the simulated bivariate data for each twin group shares the same parameterisation (e.g. for normal data both variables have $\mu$ 0 and $\sigma$ 1).  Nevertheless, we were aware that our intial plan would be over-ambitious: were each scenario to be processed consecutively, each taking 1 second to process at a steady rate optimistically with no computer crashes we might expect 1,000 simulations of each scenario in Table \ref{table:combos} to take $2,683,044,000/60/60/24/365 \approx 85$ years!  Under advice from my colleague Koen Simons, I undertook time tests of 1,000 iterations of each function and their byte code compiled versions, employing only the most efficient versions.  The permutation test implementation was particularly inefficient, and given time constraints for refactoring code this was abandoned.  Noting that the resolution of correlation combinations was a major contributor to anticipated length of processing time, this was reduced to comparison of correlations from -0.9 through 0.9 at a 0.1 resolution resulting in 361 instead of 1,521 correlation combinations.  Based on the preliminary time tests, the initial and revised time estimates are displayed along with function time results in Table \ref{{table:times}}.  These estimates suggested an anticipated running time of 16 days, based on my personal Core2Duo i7 laptop with 16gb RAM.
\\

\begin{table}\centering
\caption{Description of parameter options for single simulation \label{table:times}}
\begin{tabular}{rcc}
  \toprule
  \textbf{Test} &	\multicolumn{2}{c}{\textbf{Time/1000 runs (secs)}} \\
  \cmidrule(lr){2-3} 
   & \textbf{as is} & \textbf{compiled}				\\
  \midrule										
  Fisher's z (no sim) &         0.03 &    0.02 \\
  Fisher's Z          &         0.56 &    0.36 \\
  GTV (r)             &        18.65 &   16.55 \\
  GTV (R C++)         &              &   12.08 \\
  Permutation (PT)    &      2473.18 & 2341.62 \\
  SLR                 &         0.49 &    0.33 \\
  Zou's CI            &         0.38 &    0.28 \\
  \midrule
  \multicolumn{1}{l}{Total 1 (PT, R-compiled GTV)}       &	2493.29 & 2359.16  \\
  \multicolumn{1}{l}{Total 2 (No PT, RCCP-compiled GTV)} &	       	& 13.07    \\
  \multicolumn{1}{l}{Total 3 (No PT, R-compiled GTV)}    & 	20.11	  & 17.54    \\
  \multicolumn{1}{l}{Initial scenario: Total 1 $\times 1521 \times 49 \times 3 \times 2/60/60/24/365$}  & 35 years  & 33 years \\
  \multicolumn{1}{l}{Best scenario: Total 2 $\times 361 \times 49\times 3 \times 2 /60/60/24$}  & 25 days  & 16 days \\
  \multicolumn{1}{l}{Next best scenario: Total 3 $\times 361 \times 49\times 3 \times 2 /60/60/24$}  &    & 21 days \\
\bottomrule 
\end{tabular}
\end{table}
\\
All parameterisations giving rise to a particular scenario were recorded to ensure accurate scenario retrieval and ability to interpolate power estimates based on a subset of scenarios.  The initial format we used in our post-time estimation run was the R \code{data.table} format, as it was reportedly well optimised 'under the hood'.  The data.table was set up in wide format with rows per scenario (106,134 rows) and columns per parameter with five additional columns to contain power estimates per test (13 columns, once processed).  This format was used to process sets of results using 100 simulations, and select scenarios using 1,000 simulations and 10,000 simulations.  The 100 simulation set of results was used as initial proof of concept for the approach; the scenarios using 1,000 results provide the main results for the inferences made in this report; the 10,000 simulations scenarios were produced in order to evaluate sensitivity of simulated power estimates to the number of simulation runs, and in particular establish validity of the choice of 1,000 simulations for main results.

\section{Evaluating power}
The results of the simulation analysis using the above options are comprised of more than half a million power estimates, when taken in long form.  These can be accessed to answer specific questions.  The task of this report is to evaluate subsets of these and demostrate the value and validity of the functions giving rise to them, and provide options to researchers such as those of Twins Australia for their analyses which may be expanded upon.


 comparisons vis impact on power and required sample size to achieve this: number
of simulations (justifies 1000 simulation approach); choice of ratio; choice of test method; impact of
non-normality; magnitude of correlations; etc. I have put more old style contour plots in appendix,
which may be referred to � specific examples for GTV and SLR


results in format of images
\\
\\
contour plot
\\
\\  
interpolation using monotonic increasing spline function
\\
\\
Required sample size to achieve 80% power
\\
\\  
Difference to achieve 80% power

\chapter*{3. Results}
\markboth{Results}{}
\setcounter{chapter}{3}
\setcounter{section}{0}
\addcontentsline{toc}{chapter}{3. Results}

\section{section title}


    \begin{itemize}
      \item 530,670 power estimates comparing: tests, sample size, group size ratio, normality, correlation combinations, correlation methods
      \item Under approximate normality / mild skew, on average: \\SLR test $ \sim 82\%$ power; others $\leq 75\%$. 
    \end{itemize}
    
    
    Power to detect $\hat\delta_{\rho}$ in MZ and DZ twins $\sim \mathcal{N}\bigg(\left[\begin{smallmatrix}0\\ 0\end{smallmatrix}\right],\, \left[\begin{smallmatrix}1 & {\rho} \\ {\rho} & 1\end{smallmatrix}\right] \bigg)$
    
     \begin{center}
      \begin{figure}[!htb]
        \minipage{0.49\textwidth}
          \includegraphics[width=\linewidth]{{../figs/corrxplot_slr_example}.pdf}
        \endminipage\hfill
        \minipage{0.49\textwidth}%
          \includegraphics[width=\linewidth]{{../figs/power_n_plot_example}.pdf}
        \endminipage
      \end{figure}
    \end{center}
    
    
   % \begin{itemize}
      % \item 530,670 power estimates comparing: tests, sample size, group size ratio, normality, correlation combinations, correlation methods
      % \item Under extreme skew: \\Fisher's Z formula underestimates req. sample size by $ \sim 50\%$
    % \end{itemize}
    % \linebreak
    % \small{Power to detect $\hat\delta_{\rho}$ in MZ and DZ twins $\sim \mathcal{G}\bigg(\left[\begin{smallmatrix}1\\ 1\end{smallmatrix}\right],\, \left[\begin{smallmatrix}5 & {\rho} \\ {\rho} & 5\end{smallmatrix}\right] \bigg)$}
    % \begin{center}
      % \begin{figure}[!htb]
        % \minipage{0.49\textwidth}
          % \includegraphics[width=\linewidth]{{../figs/corrxplot_slr_example2}.pdf}
        % \endminipage\hfill
        % \minipage{0.49\textwidth}%
          % \includegraphics[width=\linewidth]{{../figs/power_n_plot_example2}.pdf}
        % \endminipage
      % \end{figure}
    % \end{center}
\begin{landscape}   
  \begin{figure}[htb]
    \centering        
    \minipage{0.25\textwidth}%
      \includegraphics[width=\linewidth]{{../figs/distx_normal_60_120}.pdf}
      % \caption{Example bivariate normal draw with $\rho(0.2,0.5)$}
      % \label{fig:1}
    \endminipage\hfill       
    \minipage{0.25\textwidth}%
      \includegraphics[width=\linewidth]{{../figs/corrx_npower_norm_r.2_r.5_mzdz.5_s100}.pdf}
      % \caption{Sample size estimate: 100 simulations}
      % \label{fig:1}
    \endminipage\hfill
    \minipage{0.25\textwidth}% 
      \includegraphics[width=\linewidth]{{../figs/corrx_npower_norm_r.2_r.5_mzdz.5_s1000}.pdf}
      % \caption{Sample size estimate: 1,000 simulations}
      % \label{fig:2}
    \endminipage\hfill
    \minipage{0.25\textwidth}% 
      \includegraphics[width=\linewidth]{{../figs/corrx_npower_norm_r.2_r.5_mzdz.5_s10000}.pdf}
      % \caption{Sample size estimate: 10,000 simulations}
      % \label{fig:3}
    \endminipage\hfill
    \caption{Left to right: Example bivariate normal draw with $\rho(0.2,0.5)$, and sample size estimates to achieve 80\% power using 100, 1000, and 10000 simulation runs per scenario}
    \label{fig:images1}
    
    \medskip
    \minipage{0.25\textwidth}% 
      \includegraphics[width=\linewidth]{{../figs/distx_gamma_mildskew_60_120}.pdf}
      % \caption{Example bivariate gamma ("mild skew") draw with $\rho(0.2,0.5)$}
      % \label{fig:4}
    \endminipage\hfill
    \minipage{0.25\textwidth}% 
      \includegraphics[width=\linewidth]{{../figs/corrx_npower_mildskew_r.2_r.5_mzdz.5_s100}.pdf}
      % \caption{Sample size estimate: 100 simulations}
      % \label{fig:4}
    \endminipage\hfill
    \minipage{0.25\textwidth}% 
      \includegraphics[width=\linewidth]{{../figs/corrx_npower_mildskew_r.2_r.5_mzdz.5_s1000}.pdf}
      % \caption{Sample size estimate: 1,000 simulations}
      % \label{fig:5}
    \endminipage\hfill
    \minipage{0.25\textwidth}% 
      \includegraphics[width=\linewidth]{{../figs/corrx_npower_mildskew_r.2_r.5_mzdz.5_s10000}.pdf}
      % \caption{Sample size estimate: 10,000 simulations}
      % \label{fig:6}
    \endminipage\hfill
    \caption{Left to right: Example bivariate gamma ("mild skew") draw from population with $\rho(0.2,0.5)$, and sample size estimates to achieve 80\% power using 100, 1000, and 10000 simulation runs per scenario}
    \label{fig:images2}
    
    \medskip
    \minipage{0.25\textwidth}% 
      \includegraphics[width=\linewidth]{{../figs/distx_gamma_extrskew_60_120}.pdf}
      % \caption{Example bivariate gamma ("extreme skew") draw with $\rho(0.2,0.5)$}
      % \label{fig:4}
    \endminipage\hfill
    \minipage{0.25\textwidth}% 
       \includegraphics[width=\linewidth]{{../figs/corrx_npower_extrskew_r.2_r.5_mzdz.5_s100}.pdf}
      % \caption{Sample size estimate: 100 simulations}
      % \label{fig:4}
    \endminipage\hfill
    \minipage{0.25\textwidth}% 
      \includegraphics[width=\linewidth]{{../figs/corrx_npower_extrskew_r.2_r.5_mzdz.5_s1000}.pdf}
      % \caption{Sample size estimate: 1,000 simulations}
      % \label{fig:5}
    \endminipage\hfill
    \minipage{0.25\textwidth}% 
      \includegraphics[width=\linewidth]{{../figs/corrx_npower_extrskew_r.2_r.5_mzdz.5_s10000}.pdf}
      % \caption{Sample size estimate: 10,000 simulations}
      % \label{fig:6}
    \endminipage\hfill
    \caption{Left to right: Example bivariate gamma ("extreme skew") draw from population with $\rho(0.2,0.5)$, and sample size estimates to achieve 80\% power using 100, 1000, and 10000 simulation runs per scenario}
    \label{fig:images3}
  \end{figure}
\end{landscape}
\chapter*{4. Discussion}
\markboth{Discussion}{}
\setcounter{chapter}{4}
\setcounter{section}{0}
\addcontentsline{toc}{chapter}{4. Discussion}
% put in comments for each paragraph to remind yourself to stay on topic!!

% \cite{Price2007,Rahimi2014,Naing2014}

\section{section title}
Despite the reduced parameter set, our results are comprised of more than half a million power estimates which can be accessed to answer specific questions.

While the main use of such results are for considering particular scenarios, we can average over these for marginal estimates; under approximate normality the SLR test was estimated to have 82% power on average, while the other tests had approximately 75%.  This result which is reflected in this plot here for example, with the higher power estimate from SLR estimate suggesting a smaller required sample size, is surprising.

The fitted contour plot on the left here compares power to detect a difference using all correlation combinations given groups sizes, in this example, under a bivariate normal distribution.  The blue line indicates the 80% power threshold.

On the right sample size estimates to achieve 80% power given population correlation coefficients of 0.2 and 0.5 are compared across the 5 implemented tests.  Stata produces two correlation power estimates with plots like these using the Fisher Z formula.  We often assume that our tests are robust to departures from normality, but it is interesting to ask how would this perform if normality assumption were severely violated?

We can see here the extreme under estimation in sample size required to achieve 80% power given an extreme positively skewed bivariate distribution when using the formula based approach compared with the simulation-based tests. 

The SLR test appears a strong performer here - but its systematic elevation at tails of distribution is concerning -- does this reflect a systematic bias, and perhaps innacurate rather than truly improved performance?  

When no difference in correlations is present under bivariate normal distribution we would expect power equivalent to our alpha level of 0.05, which, approximately, the other tests have; however, the SLR test has approximately 20\%.  This suggests its estimates here are upwardly biased.

After the SLR test the GTV was the next most powerful, but only marginally moreso than the Fisher's Z based tests.



We developed a flexible and extensible architecture geared to solving future problems.


However, the programming and analysis took longer than anticipated.  The results I have shown today were processed using 100 simulations per parameter combination.  The 1000 simulation results should be completed to update my results in a day or so.

There is more we can do to make this of more particular use in the twin context, for example by evaluating correlations in simulations using multivariable regression methods we could account for partial correlations; if we did this using mixed effects methods we could also consider power for difference in intra-class correlations.

Improve efficiency to allow higher resolution estimation

\chapter*{5. Conclusions}
\markboth{Conclusions}{}
\setcounter{chapter}{5}
\setcounter{section}{0}
\addcontentsline{toc}{chapter}{5. Conclusion}

To sum up, 
\\
tests were overall quite similar
\\
simulation important for power analysis where notable violation of assumptions of bivariate normality is anticipated  (naive application of analytical Fisher's Z formula can be extremely over-optimistic in required sample estimates)
\\
SLR implementation is biased\; highlights importance of critical consideration power of methods, as higher power may reflect bias.  For this particular SLR test, power estimates in the case of unequal group ratios were upwardly biased.
\\
Power should be considered contextualised using planned conditions for study and subject matter knowledge, and critical consideration of impact of methods employed.
\\
We have created both the architecture for a process, as well as a database of simulation scenarios that can be interrogated.  Both can be expanded as required. I have trialled an interactive power calculator web app, and it is planned incorporate the pre-processed database into this to allow on the fly estimates informed by our pre-processed results.

\backmatter%%%%%%%%%%%%%%%%%%%%%%%%%%%%%%%%%%%%%%%%%%%%%%%%%%%%%%%
 %%%%%%%%%%%%%%%%%%%%%%%%%%%%%%%%%%%%%%%%%%%%%%%%%%%%%%%%%%%%%%%
% From Springer template
% 
% 
%
%%%%%%%%%%%%%%%%%%%%%%%% Springer %%%%%%%%%%%%%%%%%%%%%%%%%%

\Extrachap{Glossary}
\markboth{Glossary}{}
\runinhead{correlation} A measure of the magnitude and direction of a linear relationship shared by two variables.
\runinhead{dizygotic} Non-identical twins arising from fertilisation of two seperate fertilised eggs, and as genetically alike as ordinary siblings.
\runinhead{heritability} The degree to which variation in a trait or phenotype, such as propensity to gain body weight, or become a centenarian, can be attributed to shared genetic effects.
\runinhead{monozygotic} Identical twins, developing from the same fertilised egg (zygote) and genetically very similar.
\runinhead{$r$} Sample estimate of the Pearson correlation coefficient.
\runinhead{$\rho$} (rho) The Pearson correlation coefficient in the population.

%%References - extra code to display as own chapter, without header, and in contents only once
\let\oldaddcontentsline\addcontentsline
\let\oldsection\section

\chapter*{References}
\addcontentsline{toc}{chapter}{References}
\markboth{Rerences}{}
\renewcommand\refname{}
% \lhead{\itshape References}
\begingroup
\small

\renewcommand{\addcontentsline}[3]{}% Remove functionality of \addcontentsline
\renewcommand{\section}[2]{}% Remove functionality of \section
\begin{thebibliography}{}

\documentclass[12pt,a4paper,titlepage,twoside,openright]{article}
%%%Reference loader
% A4 paper
\paperwidth 210mm
\paperheight 297mm
\pdfpagewidth=\paperwidth
\pdfpageheight=\paperheight

%Remove Springer Logo
\usepackage{xpatch}

\makeatletter
\xpatchcmd{\@maketitle}{{\Large Springer\par}}{}{}{}
\makeatother

%extra url help

%Headers and Footers
	% \usepackage{fancyhdr}
	% \pagestyle{fancy}
	% \fancyhf{}
	% \renewcommand{\headrulewidth}{0pt}    %remove top horizontal bar
	% %\fancyheadoffset{0\textwidth}        % center place top horizontal bar?
	% \renewcommand{\footrulewidth}{0pt}    %remove bottom horizontal bar
	% \usepackage{lastpage}
	\rightmark
	 % % \rhead{Student 659810}
	 % % \lhead{}
	% \cfoot{Page \thepage}
	% % \cfoot{Page \thepage\ of \pageref{LastPage}}

	

% Links and colours
	\usepackage[pdftex]{hyperref}
	\usepackage{xcolor}
	\hypersetup{
		colorlinks,
		linkcolor={black},
		citecolor={black},
		urlcolor=[RGB]{56 108 176}
	}

% Bibliography
	% \usepackage{natbib,har2nat} 
	% \setcitestyle{authoryear,open={(},close={)}}
	\usepackage[numbers]{natbib}
	% \bibliographystyle{spbasic2} 
	\bibliographystyle{unsrt_spbasic} 
	
	
% Lists, symbols nad matrices
	\usepackage{enumitem}
	\usepackage{amsmath}
	\usepackage{graphicx}
	\usepackage[retainorgcmds]{IEEEtrantools}
	\usepackage{multirow}
	\usepackage{rotating}
	\usepackage{changepage}
	\usepackage{dcolumn}
	\usepackage{booktabs}
	\usepackage{longtable}
	\usepackage[
	  outer=25mm,
	  inner=35mm,
	  vmargin=20mm,
	  includehead,
	  includefoot,
	  headheight=15pt,
	]{geometry}
	\usepackage{pdflscape}
	
%formatting miscellanies
	\renewcommand\thepart{\Alph{part}}
	\usepackage{capt-of}
	% \addtolength{\hoffset}{-0.5cm}
	% \addtolength{\textwidth}{1cm}

	\usepackage{setspace} %For double spacing


	\usepackage{todonotes} %For to do notes

\usepackage{mathptmx}
\usepackage{helvet}
\usepackage{courier}
%
\usepackage{type1cm}         

\usepackage{makeidx}         % allows index generation

\usepackage{multicol}        % used for the two-column index
\usepackage[bottom]{footmisc}% places footnotes at page bottom



	% \setlength{\parindent}{4em}  %remove all indenting of paragrapgs
	% \setlength{\parskip}{1em} 
	\usepackage{pdfpages} 
	\usepackage{upgreek}


	
%Not sure what this is
		% \makeatletter
		% \newcommand*{\gmshow@textheight}{\textheight}
		% \newdimen\gmshow@@textheight
		% \g@addto@macro\landscape{%
		  % \gmshow@@textheight=\hsize
		  % \renewcommand*{\gmshow@textheight}{\gmshow@@textheight}%
		% }
		% \def\Gm@vrule{%
		  % \vrule width 0.2pt height\gmshow@textheight depth\z@
		% }%
		\makeatother
		% \setlength{\parindent}{0pt}

	
%Custom commands
	% \newcommand{\plas}{\emph{Plasmodium}}

	\newcommand{\tdn}{\todo[inline, color=green!40]}

% Custom Colours
	\definecolor{farb1}{rgb}{27	 158 119 }
	\definecolor{farb2}{rgb}{217 95	 2   }
	\definecolor{farb3}{rgb}{117 112 179 }
	\definecolor{farb4}{rgb}{231 41	 138 }
	\definecolor{farb5}{rgb}{102 166 30  }
	\definecolor{farb6}{rgb}{230 171 2   }
	\definecolor{farb7}{rgb}{166 118 29  }
	\definecolor{farb8}{rgb}{102 102 102 }



\begin{document}


\chapter*{1. Introduction}
\markboth{1. Introduction}{}
\setcounter{chapter}{1}
\addcontentsline{toc}{chapter}{1. Introduction}

			% o	Summary of background and research methods; why this analysis/ evaluation is important; how and why the data was collected (initial study aims, collection methods, response rates); specifics of data (number, variables, outcomes)
			% Research question, aim, objectives of study
			
			% Research question:
			% What are the existing methods for estimating power to detect a difference in correlations between identical (monozygotic) and non-identical (dizygotic) twins, how do these compare and can they be improved upon? 

\large


% \begin{figure}[htbp]
% \sidecaption[t]
% %\centering
% \fbox{\includegraphics[scale=0.5]{figures/PapuaEndemicity.png}}
% %\picplace{5cm}{2cm} % Give the correct figure height and width in cm
% \caption{caption for figure.}
% \label{fig:pvpr}       % Give a unique label
% \end{figure}

\section{Background}
%% Intro
The classic twin study exploits the differing degrees of genetic relatedness in identical and non-identical twins in order to draw inferences on the heritability of traits.  The calculation and comparison of Pearson correlations across the two twin groups is a routine preliminary step undertaken by researchers in this field.  However, a range of factors can impact on researcher's ability to detect a true effect given data.  This thesis reports on a simulation-based power analysis for the detection of differences in correlations between identical and non-identical twin pairs under a range of scenarios, and the associated development of R functions and an applied interactive power calculator.  These tools address an identified absence of tools for this fundamental step in the twin study context.
 
\subsection{Twins and the classic twin study}
Identical twins arise from the same zygote, or fertilised egg, and are genetically very similar.  Non-identical twins arise from fertilisation of two seperate eggs, and are as genetically alike as ordinary siblings. 

The classic twin study can be used to estimate the proportion of variation in traits which may be attributable to genetics, or heritability.  

The estimates of heritability are conditional on key assumptions.  This chapter will define key genetic concepts, how these are exploited by the classic twin study to estimate the role of heridity for particular traits, and the assumptions which are relied upon in order to make such inferences.

The classic twin study exploits the differing degrees of genetic relatedness in identical and non-identical twins in order to draw inferences on the heritability of traits.  In broad terms, heritability is the degree to which variation in a trait or phenotype, such as propensity to gain body weight, or become a centenarian, can be attributed to genetics.

The comparison of phenotypic traits within identical and non-identical twin pair samples allows for partitioning the variance in traits into that attributable to  shared environment, individual environment or to genetics.  This allows us to better understand the mechanics of health and disease processes so that we can develop intervention measures which appropriately target the hypothesised causal mechanisms.

Contemporary twin studies use mixed effects and structural equation modelling to evaluate differences in variance components for a particular trait between mono and dizygotic twins accounting for differential within pair similarities related to zygosity.  A recent article reported on the development of R commands which can be used to estimate the power to detect a difference in correlations using such models.  

However, a routine preliminary step undertaken by researchers in this field is the calculation and comparison of Pearson correlations across the two twin groups.  This research project focuses on the reported needs of researchers undertaking this early analysis step.

In undertaking a twin study we make certain assumptions, understanding that these likely do not strictly hold in practice, but the key one for the purposes of this power analysis is that our data is approximately normally distributed.

\subsection{Historical background}
The statistical treatment of correlation was popularised by Francis Galton.  In context of broad social interest in eugenics, Galton described methods which could be used to describe the 'co-relatededness' of variables sourced from closely related family members \cite{Galton1888,Galton1890}.  

Karl Pearson was a keen follower of Galton's research, and writing in the context of inheritance and natural selection made use of 'Galton's function', describing it as a coefficient of correlation \cite{Pearson1895}.  

Ronald Fisher observed that a geometric transformation of the correlation cofficient using its inverse hyperbolic tangent could be used to approximate a normal distribution \cite{Fisher1915}.  This has the effect of mapping the distribution of correlation of coefficients from a domain of negative 1 through positive one to negative infinity through positive infinity. Being a simple and accurate approximation of the normal distribution, this transformation known as Fisher's Z is ubiquitous in statistical treatment of correlation coefficients, for example when seeking to compare their differences.

Florence Nightingale David was a protege of Pearson's who suggested that she prepare a volume of numerically accurate tables and interpolated plots of the distribution of the sample correlation coefficient r given n and the population correlation rho which could act as a standard against which to judge approximations such as that of Fisher \cite{David1938}. 
%  This visualisation of David's of the chance of rejecting the null hypothesis when true given alpha and rho and n was an influence on this project's presentation of results.

In the context of genetics and kinship, Douglas Falconer outlined methods of using comparison of identical and non-identical twins for estimating heritability, and noted some of the assumptions that this involves \cite{Falconer1960}.

Michael Neale and colleagues developed methods and software to facilitate the analysis of variance components in twin studies using structural equation modelling \cite{Neale1992}.  Brad Verhulst, building on the work of Peter Visscher, developed functions for power analyses in this variance component modelling context, for example detecting a difference in genetic correlations \cite{Visscher2004,Visscher2008a}.

This project is concerned with power to detect a difference in Pearson correlations as part of preliminary analysis before variance component modelling.

\subsection{Power analysis}
Power analysis involves a compromise between type 1 and type 2 error thresholds, respectively the expected proportion of null hypotheses to be rejected when true and not rejected when false.  These could be chosen to suit the requirements of a particular study, but for historical reasons the usual consensus is for 5\% and 20\%, and this is the parameterisation adopted for the results presented here.

A test statistic for the difference in correlations can be calculated as the difference in Fisher's Z transformed values weighted by the approximate standard error of the difference.  Fisher's Z transformation, the inverse hyperbolic tangent, maps the correlation coefficient from a domain of -1 through 1 to negative infinity through positive infinity, with approximate normal distribution.  This classic formulation is still used in functions found in Stata and R.

To estimate the power using this approach, you first take the difference between a normal reference score given the chosen type 1 error rate and the absolute value of the test statistic. The type 2 error rate is the probability of observing a value of at least this magnitude on the normal distribution.  And 1 minus this value is the power.

Using a simulation approach we take our hypothesis tests and apply them to draws from simulated data designed to mimic our samples through parameterisation using the hypothesised underlying bivariate population distributions.  

So where in our formula we might plug in anticipated or observed coefficients of .2 and .5, in the simulation we use these values to represent the supposed true correlations in the underlying population from which we draw our random samples.  Over a large number of simulations the proportion of hypothesis tests returning p-values lower than our type 1 error threshold is our power estimate.

 
 
%% Power
Power is the probability of detecting a true effect, a key consideration when planning a study.  In the context of differences in correlations undertaken in a twin study, we will expect a range of possible parameters to influence our power: the sample size; the ratio of MZ to DZ twins; how we have defined what constitutes a meaningful difference; the magnitude of the respective correlations; how these are measured.

Through my research project, I planned and carried out a power analysis for the detection of differences in correlations between identical and non-identical twin pairs investigating the influence of such factors.  

 \subsubsection{subsub section title}
 

			

\chapter*{2. Methods}
\markboth{Methods}{}
\setcounter{chapter}{2}
\setcounter{section}{0}
\addcontentsline{toc}{chapter}{2. Methods}

The preceding chapter summarised a review of the literature relating to twins, correlations, power and simulations undertaken to inform our approach to analysis.  Through this review we identified a series of suitable approaches and tests for evaluating differences in Pearson and Spearman correlations in two groups. Important considerations were the efficiency of our implemented simulation functions, and a well-designed data structure to support our planned as well as future outputs.  The $R$ programming environment was used for all analyses \cite{R2018}.

\section{Hypothesis tests for difference in correlations}
\subsection{Fisher's Z test (analytical approach)}
A test statistic for the difference in correlations can be calculated as the difference in Fisher's Z transformed values weighted by the approximate standard error of the difference \cite{Fisher1990,David1938}
$$t_{\hat{\theta}} = {\frac{\hat{\theta}}{se_\hat{\theta}}} = {\frac{z_{MZ} - z_{DZ}}{\sqrt{(n_{MZ}-3)^{-1}+(n_{DZ}-3)^{-1}}}} $$

The type 2 error rate $\beta$ is estimated through comparison of this test statistic $t_{\hat{\theta}}$ to a reference value on the standard normal distribution.   Using the cumulative normal distribution function $\Phi$ (Phi), the reference score $q$ is calculated as the normal probability quantile corresponding to our $\alpha$ level divided by the sidedness of our test. 
 $$q = \Phi^{-1}(\alpha/\text{sidedness})$$
Where we refer to sidedness, we mean whether we are concerned with a single- or two-tailed probability.  Here, we are testing the hypothesis that $\rho_{MZ} = \rho_{DZ}$ using a two-tailed p-value, implying 'admissible alternatives' to be the case that $\rho_{MZ}$ is greater than or less than $\rho_{DZ}$, that is, $|\rho_{MZ} - \rho_{DZ}| > 0$ \cite{David1938}.  One-tailed consideration, for example that $\rho_{MZ} > \rho_{DZ}$, is not considered in this report however, the functions developed may be parameterised in this way if desired. 
\\
Employing the concepts detailed above, $\beta$ is calculated as 
$$\beta = \Phi \big( q - t_\theta \big)$$
Our power estimate for the detection of difference in correlations is $\power(\theta) = 1 - \beta$. Putting the above altogether, we calculate power using the Fisher's $z$ test statistic as,

$$ \power = 1- \Phi \Bigg(\Phi_{\alpha/2}^{-1} -  \abs\bigg(  \frac{\arctanh(r_{MZ}) - \arctanh(r_{DZ})}{\sqrt{(n_{MZ}-3)^{-1} + (n_{DZ}-3)^{-1}}}  \bigg)  \Bigg) $$ 

The code we used to implement the analytic Fisher's Z test approach to power calculation in $R$ is displayed in listing \ref{lst:fz_nosim}.

\begin{lstlisting}[float=h,caption={Fisher's Z test (analytic approach)},label={lst:fz_nosim}]
# Fishers Z test - no sim
fz_nosim <- function(r1,r2,n1,n2,
                     alpha = 0.05, sidedness=2,method = "pearson",
                     power = TRUE) {
  # Calculate Fisher's Z
  z1     <- atanh(r1)
  z2     <- atanh(r2)
  
  # Take difference
  zdiff  <- z1-z2
  
  # Calculate standard error and test statistic
  z_se   <- sqrt(1/(n1-3) + 1/(n2-3))
  z_test <- zdiff/z_se
  
  # Optionally return p-value for observing diff at least this large under H0
  z_p    <- sidedness*pnorm(-abs(z_test))
  if (power == FALSE) return("p" = z_p)
  z_ref   <- qnorm(1-alpha/sidedness)
  z_power <- 1-pnorm(z_ref - abs(z_test))
  return(z_power)
}
\end{lstlisting}

The above method is the de facto standard, as used for example in the Stata \code{power two correlations} \cite{StataCorp2013}.  However, other options for evaluating the difference in Pearson or Spearman correlations should be considered.  

\subsection{Fisher's Z test (simulation approach)}
Using a simulation approach we take our hypothesis tests and apply them to draws from simulated data designed to mimic our samples through parameterisation using the hypothesised underlying bivariate population distributions.  
\\
\\
So where in our formula we might plug in hypothesised sample coefficients of 0.2 and 0.5, in the simulation we use these values to represent the true correlations in the underlying population from which we draw our samples.  Over a large number of simulations of bivariate twin data the proportion of hypothesis tests returning p-values lower than our type 1 error threshold is our power estimate.
\\
\\
The simulation-based Fisher's Z test function $R$ code is displayed in listing \ref{lst:fz}.

\begin{lstlisting}[float=h,caption={Fisher's Z test (simulation approach)},label={lst:fz}]
# Fishers Z test
fz <- function(a,b,sidedness=2,method = "pearson") {
  # Two samples
  n1 <- nrow(a)
  n2 <- nrow(b)
   
  # Compute z-transformed sample correlation coefficients
  z1     <- atanh(cor(a,method = method)[2,1])
  z2     <- atanh(cor(b,method = method)[2,1])
  zdiff  <- z1-z2
  
  # calculate standard error and test statistic
  z_se   <- sqrt(1/(n1-3) + 1/(n2-3))
  z_test <- zdiff/z_se
  
  # return p-value
  z_p    <- sidedness*pnorm(-abs(z_test))
  return(z_p)
}
\end{lstlisting}


In addition to applying the Fisher Z test in a simulation context, alternate tests we identified and implemented for inclusion in our simulation study were as follows. 

\subsection{Zou's confidence interval}
Zou's confidence interval is used to calculate a confidence interval for the difference in two correlations, and a hypothesis test employing this method is featured in the R package \code{cocor} \cite{Zou2007,Diedenhofen2015}.  A hypothesis test using Zou's confidence interval evaluates whether zero lies within the lower and upper bounds of the interval estimate of the difference in correlations, returning 1 if so or otherwise zero.  Over a run of simulations this would be expected to return identical results to the Fisher Z test, but may be more efficient.
\\
\\
Zou's approach expands on earlier work \cite{Olkin1995} to calculate a confidence interval for a difference in correlations using a so-called Simple Asymptotic approach, using what Zou refers to as a Modified Asymptotic method \cite{Zou2007}.  Both approaches draw heavily on Fisher's earlier work \cite{Fisher1990}.  The modified asymptotic method of Zou consists of first calculating confidence intervals for the two respective z-transformed correlations (transformed as per Fisher's method, described above):
$$(l_{z_k}, u_{z_k}) = z_k \pm \sqrt{\frac{1}{n_k - 3}} \times \Phi_{\alpha/2}^{-1},\ \text{where} \ k \in \{1,2\}$$

Then, the lower (L) and upper (U) bounds of the modified asymptotic confidence interval for the difference in correlations are calculated:
$$L = r_1 - r_2 - \sqrt{(r_1 - \tanh(l_{z_1}))^2 + (\tanh(u_{z_2})- r_2)^2}$$
$$U = r_1 - r_2 + \sqrt{(\tanh(u_{z_1}) - r_1)^2 + (r_2 - \tanh(l_{z_2}))^2}$$

If zero is within the bounds of the confidence interval for the difference, the test returns as 1, and otherwise 0.
\\
\\
Our implemenation of the Zou's confidence interval test function is displayed in listing \ref{lst:zou}.

\begin{lstlisting}[float=h,caption={Zou's confidence interval},label={lst:zou}]
zou <- function(a,b,alpha = 0.05,sidedness=2,method = "pearson") {
  # From Zou (2007) and used in Cocor (note typo for U in paper; should be '+')
  #  However, really, this is equivalent to fz test for hypothesis testing purposes
  
  # compute z- transformed correlations and differences
  r  <- c(cor(a,method = method)[2,1], cor(b,method = method)[2,1])
  z  <- atanh(r)
  zdiff  <- z[1]-z[2]
  
  # calculate standard error for respective z scores
  n  <- c(nrow(a),nrow(b))
  z_se   <- sqrt(1/(n-3))

  # calculate reference threshold
  z_ref  <- qnorm(1-alpha/sidedness)
  
  # calculate respective confidence intervals
  ci_mat <- matrix(c(-1,-1,1,1),nrow = 2, ncol = 2, dimnames =list(c("Mz","Dz"),c("l","u")))
  z_ci   <- z + ci_mat * z_se * z_ref
  r_ci   <- tanh(z_ci)
  
  # calculate Zou's Modified Asymptoptic confidence interval for difference in correlations
  L      <- r[1]-r[2] - sqrt((r[1]      - r_ci[1,1])^2 + (r_ci[2,2] - r[2]     )^2)
  U      <- r[1]-r[2] + sqrt((r_ci[1,2] - r[1]     )^2 + (r[2]      - r_ci[2,1])^2)
  r_diff_ci <- c(L,U)
  
  # return test value (0 or 1, however, in the power context this resolves to same outcome as p)
  ci_test <- (L < 0) && (0 < U)
  return(c(ci_test,r_diff_ci))
}
\end{lstlisting}



\subsection{Generalised Variable Test}

The generalised variable (GV) test involves transformation of the simulated sample correlations into so-called pivotal quantities the difference of which is used to calculate a test statistic and p-value \cite{Krishnamoorthy2014}. Synthesising two reported approaches \cite{Krishnamoorthy2007,Kazemi2016} this test was first implemented as an example by my supervisor Enes Makalic in a Matlab script, and subsequently adapted by myself as a function in R.  A compiled version using RCPP to leverage C++ routines for random number draws was suggested by my colleague Koen Simons, and adopted to improve the function's run time. However, this later version was not compatible with the parallelised simulation approach, and in this context the non-RCCP 'GVT-r' version was used.
\\
\\
Given two bivariate normal samples $k\in\{1,2\}$, the sample correlation coefficients $r_k$ are used to estimate two respective quantities $r_k^* = \frac{r_k}{\sqrt(1-r_k^2)}$, and the generalised variables $G_{\rho_k}$:
$$G_{\rho_k} = \frac{r_k^*\sqrt{W_k} - U_k}{\sqrt{(r_k^*\sqrt(W_k) - U_k)^2 + V_k}}$$
where,
$$U_k \sim N(0,1) ,\ V_k \sim \chi_{n_k - 1}^2 ,\ \text{and} \ W_k \sim \chi_{n_k-2}^2$$
A p-value using the GV test is calculated as twice the value of the smaller of two quantities: the proportion of differences in $G_{\rho_k}$ less than 0, and the proportion greater than 0.

The GV test function $R$ code is displayed in listing \ref{lst:gvtr}.

\begin{lstlisting}[float=h,caption={GV test (R version)},label={lst:gvtr}]
gvt_r <- function(a,b,M=1e4,method = "pearson") {
  # Two samples
  n1 <- nrow(a)
  n2 <- nrow(b)
  
  # Compute sample correlation coefficients
  r1 <- cor(a,method = method)[2,1]
  r2 <- cor(b,method = method)[2,1]
  r  <- c(r1,r2)
  
  # Generate random numbers
  V2     <- matrix(data=0, nrow = M, ncol = 2)
  V2[,1] <- rchisq(M, df = n1-1, ncp = 1)
  V2[,2] <- rchisq(M, df = n2-1, ncp = 1)
  
  W2     <- matrix(data=0, nrow = M, ncol = 2)
  W2[,1] <- rchisq(M, df = n1-2, ncp = 1)
  W2[,2] <- rchisq(M, df = n2-2, ncp = 1)
  
  Z <-matrix(data = rnorm(2*M), nrow=M, ncol = 2)
  
  # Compute test statistic
  rstar <- r/sqrt(1-r^2)
  top   <- c(sqrt(W2[,1])*rstar[1],sqrt(W2[,2])*rstar[2]) - Z
  G     <- top / sqrt( top^2 + V2 )
  
  # Compute p value
  Grho <- G[,1] - G[,2];
  p    <- 2*min( mean(Grho<0), mean(Grho>0) ); 
  return(p)
}
\end{lstlisting}

\subsection{Signed log-likelihood ratio test}
The signed log likelihood ratio (SLR) test is formulated as the signed difference in sample correlation coefficients multiplied by the square root of the sum of respective coefficients' log-likelihoods.  The test here is a partial implementation of a recently reported modified signed log-likelihood ratio (MSLR) test  for differences in two bivariate normal correlations \cite{Kazemi2016}. The SLR and MSLR tests are well established general hypothesis tests \cite{Barndorff1986,Barndorff1991,Diciccio2001,Krishnamoorthy2014}, the novelty in Kazemi and Jafari's approach being the applied context of difference in correlations. However, we (myself, nor my supervisors) were unable to successfully replicate the 'modified' portion of Kazemi and Jafari's reported algorithm.  Due to time constraints, and noting that the 'unmodified' SLR test appeared to return p-values similar to the other hypothesis tests it was decided that inclusion of the SLR test would be a valid option to consider.
\\
\\
The SLR test function $R$ code is displayed in listing \ref{lst:slr}.

\begin{lstlisting}[float=h,caption={Signed log-likelihood ratio test},label={lst:slr}]
slr <- function(a,b,M=1e4,sidedness=2,method = "pearson") {
  # Signed Log-likelihood Ratio test (an 'unmodified' version of test 
  # described in Krishnamoorthy and Lee, Kazemi and Jafari , DiCiccio etc)
  # Two samples
  n  <- c(nrow(a),nrow(b))
  
  # Compute z-transformed sample correlation coefficients
  r  <- c(cor(a,method = method)[2,1], cor(b,method = method)[2,1])
  z  <- atanh(r)
  
  # Calculate average z as a plug in value
  rf <- tanh(mean(z))
  
  # calcaulte SLR
  slr <-sign(r[1]-r[2])*sqrt(sum(n*log(((1-rf*r)^2)/((1-r^2)*(1-rf^2)))))
  
  # return p-value
  p    <- 2 * (1 - pnorm(abs(slr))); 
  return(p)
}
\end{lstlisting}

\subsection{Permutation test}
The permutation test is a non-parametric approach which compares the absolute 
difference of the Z-transformed sample correlations with those using correlations 
from a series of group membership permutations using the sample rank orders as 
values.  Under a hypothesis of no difference in correlation, those differences arising from permutations would be assumed to be equally likely as those observed, or anticipated to be observed \cite{Efron1993}.  Across a series of $M$ permutations (in this study, 10,000), a $p$-value is calculated as the proportion of permutation derived absolute differences (($\abs(z_{MZ}^* - z_{DZ}^*)$)) of greater magnitude than $\abs(z_{MZ} - z_{DZ})$.
\\
\\
The implementation of this permutation test in $R$ is displayed in listing \ref{lst:pt}.

\begin{lstlisting}[float=h,caption={Permutation test},label={lst:pt}]
pt <- function(a,b,M=1e4,sidedness=2,method = "pearson") {
  # Based on Efron and Tibshirani, 1993
  # Store size, and calculate z-transformed correlations
  n  <- c(nrow(a),nrow(b))
  r  <- c(cor(a,method = method)[2,1], cor(b,method = method)[2,1])
  z  <- atanh(r)
  
  # Store rank-ordered vector representations, in one column
  v  <- cbind(rank(rbind(a[,1],b[,1]),ties.method = "random"),
              rank(rbind(a[,2],b[,2]),ties.method = "random"))
  # label rows
  rownames(v) <- c(rep("A",n[1]),rep("B",n[2]))
  
  # initial empty test vector
  rtest <- numeric(0)
  
  # run M permutations (default is 10,000),
  #  - returns test that absolute magnitude of difference
  #    is at least as great as that of the input z-transformed corr. diff.
  for (i in 1:M){
    permute <- cbind(v,rbinom(sum(n),1,0.5))
    rstar   <- c(cor(permute[permute[,3]==0,c(1,2)],method = method)[2,1],
                 cor(permute[permute[,3]==1,c(1,2)],method = method)[2,1])
    zstar   <- atanh(rstar)
    rtest   <- c(rtest,
                 abs(zstar[1]-zstar[2]) > abs(z[1]-z[2]))
    } 
  
  # return p-value: proportion of test results at least as large as obs'd
  p <- mean(rtest)
  return(p)
}
\end{lstlisting}

\section{Simulation}
\subsection{Approach for one simulation}
A function \code{corr\_diff\_test()} was developed to undertake a single comparative simulation of any of the above tests (listing \ref{lst:corr_diff}). Within a single simulation, each simulation based  test is evaluated using the same samples drawn from the two simulated bivariate populations as input, returning a $p$-value.  The analytic Fisher Z test returns either a $p$-value or a power estimate based on the population parameters.  Note that in the detailing of computational aspects of our methodology we use the term 'parameter' to refer to one of the options which may be specified within a function; an argument is the specific value which is passed to that parameter. For example, the (population) parameter $\rho$ for the respective MZ and DZ groups may be defined by specifying the argument \code{rho = c(-0.65,0.2)}. The main parameters which can be specified in the function call to \code{corr\_diff\_test()} are described with example arguments in table \ref{table:corr_params}.  In the following text we refer to the set of parameters that give rise to a simulation as a scenario.

\begin{lstlisting}[float=h,caption={Single run simulation code},label={lst:corr_diff}]
corr_diff_test <- function(rho = c(.2,.5), n = c(30,90), distr = "normal",
                    param1a = c(0,0), param1b = c(0,0),param2a = c(1,1), param2b = c(1,1),
                    alpha = 0.05, sidedness = 2, test = c("fz","gtv","pt","slr","zou"),
                    method ="pearson", lower.tri = FALSE) {
  if(lower.tri==TRUE){
    # optionally, only calculate results for lower matrix half 
    #   when comparing across all correlation combinations
    if(rho[1] < rho[2]) { 
      return(NA)
    }
  }
  # initialise empty results vector
  results <- list()
  
  # if requested, process analytical Fisher's Z
  if ("fz_nosim" %in% test) {
    results[["fz_nosim"]] <- fz_ns_compiled(rho[1],rho[2],n[1],n[2], 
                                      alpha = 0.05, sidedness = 2, method = method, power = FALSE)
    if(length(test)==1) return(results)
  }
  # process selected hypothesis tests, each using same draw of simulated data
  require("simstudy")
  a <- genCorGen(n[1], nvars = 2, params1 = param1a, params2 = param2a,  
                dist = distr, corMatrix = matrix(c(1, rho[1], rho[1], 1), ncol = 2), 
                wide = TRUE)[,2:3]
  b <- genCorGen(n[2], nvars = 2, params1 = param1b, params2 = param2b,  
                dist = distr, corMatrix = matrix(c(1, rho[2], rho[2], 1), ncol = 2), 
                wide = TRUE)[,2:3]
  if ("fz"       %in% test) results[["fz"]]       <- fz_compiled(a,b)
  if ("gtv"      %in% test) results[["gtv"]]      <- gtv(a,b) # uses rccp ; so elsewise compiled
  if ("gtvr"     %in% test) results[["gtvr"]]      <- gtv_compiled(a,b) 
  if ("pt"       %in% test) results[["pt"]]       <- pt_compiled(a,b)
  if ("slr"      %in% test) results[["slr"]]      <- slr_compiled(a,b)
  if ("zou"      %in% test) results[["zou"]]      <- zou_compiled(a,b)[1]
  return(rbind(results[test]))
}
\end{lstlisting}

The function uses the $R$ package \code{simstudy} function \code{genCorGen} to generate bivariate correlated data for the simulated MZ (group \code{a} in the code above) and DZ (group \code{b}) twin pair samples \cite{simstudy2018}.  The choice of available distributions and parameterisations is normal($\mu,\sigma$), binomial(probability $p$), Poisson(rate $\lambda$) gamma($\mu,\text{dispersion} \ k$), or uniform(min, max).  In addition to specifying sample size, distribution and parameterisation, a correlation matrix may be specified; this was used to parameterise the underlying population correlations from which bivariate samples should be drawn.  Three distinct distribution types were modelled in our simulation based power analysis: normal, 'mild' skew and 'extreme' skew. These are respectively explained in the captions of Figures \ref{fig:dist_norm}, \ref{fig:dist_gamma1}, and \ref{fig:dist_gamma2}, which illustrate example sample draws from these distributions.  

\begin{figure}[htbp]
\sidecaption[t]
%\centering
\fbox{\includegraphics[scale=0.52]{{../figs/distx_normal_60_120}.pdf}}
%\picplace{5cm}{2cm} % Give the correct figure height and width in cm
\caption{An example of the bivariate normal scenario, with distributional assumptions asymptotically met.  Both variables are standardised with mean $\mu=0$ and standard deviation $\sigma=1$.}
 % and distribution $\sim N\bigg{\boldsymbol\mu = \begin{pmatrix} 0 \\ 0 \end{pmatrix}, \quad \boldsymbol\Sigma = \begin{pmatrix} 1 & \rho \\ \rho  & 1 \end{pmatrix} \bigg)$.
% \caption{An example of the bivariate normal scenario, under which our distributional assumptions are asymptotically met.  This was specified with both variables standardised having mean $\mu=0$ and standard deviation $\sigma=1$ and distribution $\sim N\bigg{\boldsymbol\mu = \begin{pmatrix} 0 \\ 0 \end{pmatrix}, \quad \boldsymbol\Sigma = \begin{pmatrix} 1 & \rho \\ \rho  & 1 \end{pmatrix} \bigg)$.}
\label{fig:dist_norm}       % Give a unique label
\end{figure}

\begin{figure}[htbp]
\sidecaption[t]
%\centering
\fbox{\includegraphics[scale=0.52]{{../figs/distx_gamma_mildskew_60_120}.pdf}}
%\picplace{5cm}{2cm} % Give the correct figure height and width in cm
\caption{A 'mild skew' scenario based on a gamma distribution with mean 1.5 and dispersion 0.09 (which the genCorGen function uses to inform shape and scale parameters for the distribution).  This parameterisation was chosen through experimentation with the intent to represent a mild departure from an assumed normal population distribution, with a slight positive skew}
\label{fig:dist_gamma1}       % Give a unique label
\end{figure}

\begin{figure}[htbp]
\sidecaption[t]
%\centering
\fbox{\includegraphics[scale=0.52]{{../figs/distx_gamma_extrskew_60_120}.pdf}}
%\picplace{5cm}{2cm} % Give the correct figure height and width in cm
\caption{An 'extreme skew' scenario based on a gamma distribution with mean 1 and dispersion 5.  This results in an extreme positive skew to the distribution, analogous to that of biological processes where most observations will be clustered around a certain value, however some outliers may be extremely elevated.}
\label{fig:dist_gamma2}% Give a unique label
\end{figure}


\begin{table}\centering
\caption{Description of parameter options for single simulation \label{table:corr_params}}
\begin{tabular}{cll}
  \toprule
  \textbf{Parameter} & \textbf{Description} & \textbf{Example arguments} \\ [0.5ex] 
  \midrule
  \code{method}    & Correlation method to use for testing difference                  & \code{'pearson'}                 \\
  \code{rho}       & $\rho$ for each group's bivariate distribution                    & \code{c(-0.21,0.59)}             \\
  \code{n}         & Sample size for groups 1 (MZ) and  2 (DZ)                         & \code{c(30,60)}                  \\
  \code{dist}      & Distribution to be used for both groups' bivariate distribution \ & \code{'normal'}                  \\
  \code{param1a}   & Distribution parameter 1 for respective samples in group 1        & \code{c(0,0)}                    \\
  \code{param1b}   & Distribution parameter 1 for respective samples in group 2        & \code{c(0,0)}                    \\
  \code{param2a}   & Distribution parameter 2 for respective samples in group 1        & \code{c(1,1)}                    \\
  \code{param2b}   & Distribution parameter 2 for respective samples in group 2        & \code{c(1,1)}                    \\
  \code{test}      & Tests to be evaluated                                             & \code{c("fz\_nosim","fz","gtv")} \\
  \code{alpha}     & $\alpha$ value to use for hypothesis tests                        & \code{0.05}                      \\
  \code{sidedness} & Sidedness for hypothesis tests                                    & \code{2}                         \\
  \bottomrule 
\end{tabular}
\end{table}

% whether only to compute results for where $rho_1 < rho_2$, which across a series of correlation combinations conceived as a matrix can be used to optionally returne only the lower triangular matrix of results.

\subsection{Approach for multiple simulations}
The function described above runs a single simulation of drawing from samples from two bivariate populations.  However, for asymptotic normality to hold --- the long run approximation of a normal distribution due to the Central Limit Theorem \cite{Casella2002} --- we know we must run many more simulations to achieve a fair assessment of the proportion of null hypotheses rejected when false for when using a particular hypothesis test under a particular scenario (ie. set of parameterisations).
\\
\\
A wrapper function \code{corr\_power} to allows for the \code{corr\_diff\_test()} function to be called $M$ times for a given scenario, returning a power estimate for each test specified in the function call.  These power estimates are derived from the series of simulated $p$-values for each test considered, or directly in the case of the analytic Fisher's Z test.

\section{Scenario combinations}
The flexibility of the simulation commands we defined allows for an very broad array of scenarios to be considered.  A researcher could use these tools as is to aid in the development of a statistical analysis plan for their study.  For this report we had to decide on a limited subset of these possibilities.  There are two drivers for the choice made: the number of scenario combinations, and the time taken to run each of these.  Our initial plan for the series of scenarios is reported in table \ref{table:combos}.
\\

\begin{table}\centering
\caption{Description of parameter options for single simulation \label{table:combos}}
\begin{tabular}{llr}
  \toprule
  \textbf{Parameter} & \multicolumn{2}{c}{\textbf{Initial plan}} \\ 
  \cmidrule(lr){2-3} 
   & \textbf{Argument resolution} & \textbf{Combinations}	\\
  \midrule
  \code{method}    & Pearson and Spearman correlations                                 & 2 \\
  \code{rho}       & $\rho$ combinations\: -0.95 through 0.95 at 0.5 resolution         & $39^2 = 1521$ \\
  \code{n}         & group combinations\: 15, 30, 60, 120, 240, 480, 960                      & $7^2 = 49$ \\
  \code{dist}      & normal, 'mild skew', 'extreme skew'                               &  3 \\
  \code{param1a}   & dictated by distribution choice, above                            & -  \\
  \code{param1b}   & dictated by distribution choice, above (equal to param1a)         & -  \\
  \code{param2a}   & dictated by distribution choice, above                            & -  \\
  \code{param2b}   & dictated by distribution choice, above (equal to param2a)         & -  \\
  \code{test}      & Fisher's Z (analytic and sim), Zou's CI, GVT, SLR, PT             & 6  \\
  \code{alpha}     & .05                                                               & 1  \\
  \code{sidedness} & 2                                                                 & 1  \\
  \midrule
  Total scenarios &                                                            & 2,683,044  \\
  Total simulations & 1,000 simulations for each scenario                   & 2,683,044,000 \\
  Total simulations & 100, 1,000 and 10,000 simulations for each scenario (unrealistic!) &  29,781,788,400 \\
  \bottomrule 
\end{tabular}
\end{table}

\\
We decided \textit{a prior} to set some parameters as fixed, : we only conducted two-sided tests with $\alpha$ of 0.05;  having decided to limit ourselves to three basic distributional forms (normal, and two kinds of non-normal using distinct gamma distribution parameterisations) the distribution parameters were fixed to achieve these forms; the respective simulations of MZ and DZ twin pair samples each use the same distributional form, although the population correlation and sample size may vary (e.g. we don't compare bivariate normal MZ with a gamma skewed DZ); the simulated bivariate data for each twin group shares the same parameterisation (e.g. for normal data both variables have $\mu$ 0 and $\sigma$ 1).  Nevertheless, we were aware that our intial plan would be over-ambitious: were each scenario to be processed consecutively, each taking 1 second to process at a steady rate optimistically with no computer crashes we might expect 1,000 simulations of each scenario in Table \ref{table:combos} to take $2,683,044,000/60/60/24/365 \approx 85$ years!  Under advice from my colleague Koen Simons, I undertook time tests of 1,000 iterations of each function and their byte code compiled versions, employing only the most efficient versions.  The permutation test implementation was particularly inefficient, and given time constraints for refactoring code this was abandoned.  Noting that the resolution of correlation combinations was a major contributor to anticipated length of processing time, this was reduced to comparison of correlations from -0.9 through 0.9 at a 0.1 resolution resulting in 361 instead of 1,521 correlation combinations.  Based on the preliminary time tests, the initial and revised time estimates are displayed along with function time results in Table \ref{{table:times}}.  These estimates suggested an anticipated running time of 16 days, based on my personal Core2Duo i7 laptop with 16gb RAM.
\\

\begin{table}\centering
\caption{Description of parameter options for single simulation \label{table:times}}
\begin{tabular}{rcc}
  \toprule
  \textbf{Test} &	\multicolumn{2}{c}{\textbf{Time/1000 runs (secs)}} \\
  \cmidrule(lr){2-3} 
   & \textbf{as is} & \textbf{compiled}				\\
  \midrule										
  Fisher's z (no sim) &         0.03 &    0.02 \\
  Fisher's Z          &         0.56 &    0.36 \\
  GTV (r)             &        18.65 &   16.55 \\
  GTV (R C++)         &              &   12.08 \\
  Permutation (PT)    &      2473.18 & 2341.62 \\
  SLR                 &         0.49 &    0.33 \\
  Zou's CI            &         0.38 &    0.28 \\
  \midrule
  \multicolumn{1}{l}{Total 1 (PT, R-compiled GTV)}       &	2493.29 & 2359.16  \\
  \multicolumn{1}{l}{Total 2 (No PT, RCCP-compiled GTV)} &	       	& 13.07    \\
  \multicolumn{1}{l}{Total 3 (No PT, R-compiled GTV)}    & 	20.11	  & 17.54    \\
  \multicolumn{1}{l}{Initial scenario: Total 1 $\times 1521 \times 49 \times 3 \times 2/60/60/24/365$}  & 35 years  & 33 years \\
  \multicolumn{1}{l}{Best scenario: Total 2 $\times 361 \times 49\times 3 \times 2 /60/60/24$}  & 25 days  & 16 days \\
  \multicolumn{1}{l}{Next best scenario: Total 3 $\times 361 \times 49\times 3 \times 2 /60/60/24$}  &    & 21 days \\
\bottomrule 
\end{tabular}
\end{table}
\\
All parameterisations giving rise to a particular scenario were recorded to ensure accurate scenario retrieval and ability to interpolate power estimates based on a subset of scenarios.  The initial format we used in our post-time estimation run was the R \code{data.table} format, as it was reportedly well optimised 'under the hood'.  The data.table was set up in wide format with rows per scenario (106,134 rows) and columns per parameter with five additional columns to contain power estimates per test (13 columns, once processed).  This format was used to process sets of results using 100 simulations, and select scenarios using 1,000 simulations and 10,000 simulations.  The 100 simulation set of results was used as initial proof of concept for the approach; the scenarios using 1,000 results provide the main results for the inferences made in this report; the 10,000 simulations scenarios were produced in order to evaluate sensitivity of simulated power estimates to the number of simulation runs, and in particular establish validity of the choice of 1,000 simulations for main results.

\section{Evaluating power}
The results of the simulation analysis using the above options are comprised of more than half a million power estimates, when taken in long form.  These can be accessed to answer specific questions.  The task of this report is to evaluate subsets of these and demostrate the value and validity of the functions giving rise to them, and provide options to researchers such as those of Twins Australia for their analyses which may be expanded upon.


 comparisons vis impact on power and required sample size to achieve this: number
of simulations (justifies 1000 simulation approach); choice of ratio; choice of test method; impact of
non-normality; magnitude of correlations; etc. I have put more old style contour plots in appendix,
which may be referred to � specific examples for GTV and SLR


results in format of images
\\
\\
contour plot
\\
\\  
interpolation using monotonic increasing spline function
\\
\\
Required sample size to achieve 80% power
\\
\\  
Difference to achieve 80% power

\chapter*{3. Results}
\markboth{Results}{}
\setcounter{chapter}{3}
\setcounter{section}{0}
\addcontentsline{toc}{chapter}{3. Results}

\section{section title}


    \begin{itemize}
      \item 530,670 power estimates comparing: tests, sample size, group size ratio, normality, correlation combinations, correlation methods
      \item Under approximate normality / mild skew, on average: \\SLR test $ \sim 82\%$ power; others $\leq 75\%$. 
    \end{itemize}
    
    
    Power to detect $\hat\delta_{\rho}$ in MZ and DZ twins $\sim \mathcal{N}\bigg(\left[\begin{smallmatrix}0\\ 0\end{smallmatrix}\right],\, \left[\begin{smallmatrix}1 & {\rho} \\ {\rho} & 1\end{smallmatrix}\right] \bigg)$
    
     \begin{center}
      \begin{figure}[!htb]
        \minipage{0.49\textwidth}
          \includegraphics[width=\linewidth]{{../figs/corrxplot_slr_example}.pdf}
        \endminipage\hfill
        \minipage{0.49\textwidth}%
          \includegraphics[width=\linewidth]{{../figs/power_n_plot_example}.pdf}
        \endminipage
      \end{figure}
    \end{center}
    
    
   % \begin{itemize}
      % \item 530,670 power estimates comparing: tests, sample size, group size ratio, normality, correlation combinations, correlation methods
      % \item Under extreme skew: \\Fisher's Z formula underestimates req. sample size by $ \sim 50\%$
    % \end{itemize}
    % \linebreak
    % \small{Power to detect $\hat\delta_{\rho}$ in MZ and DZ twins $\sim \mathcal{G}\bigg(\left[\begin{smallmatrix}1\\ 1\end{smallmatrix}\right],\, \left[\begin{smallmatrix}5 & {\rho} \\ {\rho} & 5\end{smallmatrix}\right] \bigg)$}
    % \begin{center}
      % \begin{figure}[!htb]
        % \minipage{0.49\textwidth}
          % \includegraphics[width=\linewidth]{{../figs/corrxplot_slr_example2}.pdf}
        % \endminipage\hfill
        % \minipage{0.49\textwidth}%
          % \includegraphics[width=\linewidth]{{../figs/power_n_plot_example2}.pdf}
        % \endminipage
      % \end{figure}
    % \end{center}
\begin{landscape}   
  \begin{figure}[htb]
    \centering        
    \minipage{0.25\textwidth}%
      \includegraphics[width=\linewidth]{{../figs/distx_normal_60_120}.pdf}
      % \caption{Example bivariate normal draw with $\rho(0.2,0.5)$}
      % \label{fig:1}
    \endminipage\hfill       
    \minipage{0.25\textwidth}%
      \includegraphics[width=\linewidth]{{../figs/corrx_npower_norm_r.2_r.5_mzdz.5_s100}.pdf}
      % \caption{Sample size estimate: 100 simulations}
      % \label{fig:1}
    \endminipage\hfill
    \minipage{0.25\textwidth}% 
      \includegraphics[width=\linewidth]{{../figs/corrx_npower_norm_r.2_r.5_mzdz.5_s1000}.pdf}
      % \caption{Sample size estimate: 1,000 simulations}
      % \label{fig:2}
    \endminipage\hfill
    \minipage{0.25\textwidth}% 
      \includegraphics[width=\linewidth]{{../figs/corrx_npower_norm_r.2_r.5_mzdz.5_s10000}.pdf}
      % \caption{Sample size estimate: 10,000 simulations}
      % \label{fig:3}
    \endminipage\hfill
    \caption{Left to right: Example bivariate normal draw with $\rho(0.2,0.5)$, and sample size estimates to achieve 80\% power using 100, 1000, and 10000 simulation runs per scenario}
    \label{fig:images1}
    
    \medskip
    \minipage{0.25\textwidth}% 
      \includegraphics[width=\linewidth]{{../figs/distx_gamma_mildskew_60_120}.pdf}
      % \caption{Example bivariate gamma ("mild skew") draw with $\rho(0.2,0.5)$}
      % \label{fig:4}
    \endminipage\hfill
    \minipage{0.25\textwidth}% 
      \includegraphics[width=\linewidth]{{../figs/corrx_npower_mildskew_r.2_r.5_mzdz.5_s100}.pdf}
      % \caption{Sample size estimate: 100 simulations}
      % \label{fig:4}
    \endminipage\hfill
    \minipage{0.25\textwidth}% 
      \includegraphics[width=\linewidth]{{../figs/corrx_npower_mildskew_r.2_r.5_mzdz.5_s1000}.pdf}
      % \caption{Sample size estimate: 1,000 simulations}
      % \label{fig:5}
    \endminipage\hfill
    \minipage{0.25\textwidth}% 
      \includegraphics[width=\linewidth]{{../figs/corrx_npower_mildskew_r.2_r.5_mzdz.5_s10000}.pdf}
      % \caption{Sample size estimate: 10,000 simulations}
      % \label{fig:6}
    \endminipage\hfill
    \caption{Left to right: Example bivariate gamma ("mild skew") draw from population with $\rho(0.2,0.5)$, and sample size estimates to achieve 80\% power using 100, 1000, and 10000 simulation runs per scenario}
    \label{fig:images2}
    
    \medskip
    \minipage{0.25\textwidth}% 
      \includegraphics[width=\linewidth]{{../figs/distx_gamma_extrskew_60_120}.pdf}
      % \caption{Example bivariate gamma ("extreme skew") draw with $\rho(0.2,0.5)$}
      % \label{fig:4}
    \endminipage\hfill
    \minipage{0.25\textwidth}% 
       \includegraphics[width=\linewidth]{{../figs/corrx_npower_extrskew_r.2_r.5_mzdz.5_s100}.pdf}
      % \caption{Sample size estimate: 100 simulations}
      % \label{fig:4}
    \endminipage\hfill
    \minipage{0.25\textwidth}% 
      \includegraphics[width=\linewidth]{{../figs/corrx_npower_extrskew_r.2_r.5_mzdz.5_s1000}.pdf}
      % \caption{Sample size estimate: 1,000 simulations}
      % \label{fig:5}
    \endminipage\hfill
    \minipage{0.25\textwidth}% 
      \includegraphics[width=\linewidth]{{../figs/corrx_npower_extrskew_r.2_r.5_mzdz.5_s10000}.pdf}
      % \caption{Sample size estimate: 10,000 simulations}
      % \label{fig:6}
    \endminipage\hfill
    \caption{Left to right: Example bivariate gamma ("extreme skew") draw from population with $\rho(0.2,0.5)$, and sample size estimates to achieve 80\% power using 100, 1000, and 10000 simulation runs per scenario}
    \label{fig:images3}
  \end{figure}
\end{landscape}
\chapter*{4. Discussion}
\markboth{Discussion}{}
\setcounter{chapter}{4}
\setcounter{section}{0}
\addcontentsline{toc}{chapter}{4. Discussion}
% put in comments for each paragraph to remind yourself to stay on topic!!

% \cite{Price2007,Rahimi2014,Naing2014}

\section{section title}
Despite the reduced parameter set, our results are comprised of more than half a million power estimates which can be accessed to answer specific questions.

While the main use of such results are for considering particular scenarios, we can average over these for marginal estimates; under approximate normality the SLR test was estimated to have 82% power on average, while the other tests had approximately 75%.  This result which is reflected in this plot here for example, with the higher power estimate from SLR estimate suggesting a smaller required sample size, is surprising.

The fitted contour plot on the left here compares power to detect a difference using all correlation combinations given groups sizes, in this example, under a bivariate normal distribution.  The blue line indicates the 80% power threshold.

On the right sample size estimates to achieve 80% power given population correlation coefficients of 0.2 and 0.5 are compared across the 5 implemented tests.  Stata produces two correlation power estimates with plots like these using the Fisher Z formula.  We often assume that our tests are robust to departures from normality, but it is interesting to ask how would this perform if normality assumption were severely violated?

We can see here the extreme under estimation in sample size required to achieve 80% power given an extreme positively skewed bivariate distribution when using the formula based approach compared with the simulation-based tests. 

The SLR test appears a strong performer here - but its systematic elevation at tails of distribution is concerning -- does this reflect a systematic bias, and perhaps innacurate rather than truly improved performance?  

When no difference in correlations is present under bivariate normal distribution we would expect power equivalent to our alpha level of 0.05, which, approximately, the other tests have; however, the SLR test has approximately 20\%.  This suggests its estimates here are upwardly biased.

After the SLR test the GTV was the next most powerful, but only marginally moreso than the Fisher's Z based tests.



We developed a flexible and extensible architecture geared to solving future problems.


However, the programming and analysis took longer than anticipated.  The results I have shown today were processed using 100 simulations per parameter combination.  The 1000 simulation results should be completed to update my results in a day or so.

There is more we can do to make this of more particular use in the twin context, for example by evaluating correlations in simulations using multivariable regression methods we could account for partial correlations; if we did this using mixed effects methods we could also consider power for difference in intra-class correlations.

Improve efficiency to allow higher resolution estimation

\chapter*{5. Conclusions}
\markboth{Conclusions}{}
\setcounter{chapter}{5}
\setcounter{section}{0}
\addcontentsline{toc}{chapter}{5. Conclusion}

To sum up, 
\\
tests were overall quite similar
\\
simulation important for power analysis where notable violation of assumptions of bivariate normality is anticipated  (naive application of analytical Fisher's Z formula can be extremely over-optimistic in required sample estimates)
\\
SLR implementation is biased\; highlights importance of critical consideration power of methods, as higher power may reflect bias.  For this particular SLR test, power estimates in the case of unequal group ratios were upwardly biased.
\\
Power should be considered contextualised using planned conditions for study and subject matter knowledge, and critical consideration of impact of methods employed.
\\
We have created both the architecture for a process, as well as a database of simulation scenarios that can be interrogated.  Both can be expanded as required. I have trialled an interactive power calculator web app, and it is planned incorporate the pre-processed database into this to allow on the fly estimates informed by our pre-processed results.
 %%%%%%%%%%%%%%%%%%%%%%%%%%%%%%%%%%%%%%%%%%%%%%%%%%%%%%%%%%%%%%%
% From Springer template
% 
% 
%
%%%%%%%%%%%%%%%%%%%%%%%% Springer %%%%%%%%%%%%%%%%%%%%%%%%%%

\Extrachap{Glossary}
\markboth{Glossary}{}
\runinhead{correlation} A measure of the magnitude and direction of a linear relationship shared by two variables.
\runinhead{dizygotic} Non-identical twins arising from fertilisation of two seperate fertilised eggs, and as genetically alike as ordinary siblings.
\runinhead{heritability} The degree to which variation in a trait or phenotype, such as propensity to gain body weight, or become a centenarian, can be attributed to shared genetic effects.
\runinhead{monozygotic} Identical twins, developing from the same fertilised egg (zygote) and genetically very similar.
\runinhead{$r$} Sample estimate of the Pearson correlation coefficient.
\runinhead{$\rho$} (rho) The Pearson correlation coefficient in the population.




\bibliography{malEps_bib2}   


\end{document}

%%NOTE! Manually reorder v2.bbl to reflect numeric ordering

\end{thebibliography}{}
\endgroup

% 
\documentclass[12pt,a4paper,titlepage,twoside,openright]{article}
%%%Reference loader
% A4 paper
\paperwidth 210mm
\paperheight 297mm
\pdfpagewidth=\paperwidth
\pdfpageheight=\paperheight

%Remove Springer Logo
\usepackage{xpatch}

\makeatletter
\xpatchcmd{\@maketitle}{{\Large Springer\par}}{}{}{}
\makeatother

%extra url help

%Headers and Footers
	% \usepackage{fancyhdr}
	% \pagestyle{fancy}
	% \fancyhf{}
	% \renewcommand{\headrulewidth}{0pt}    %remove top horizontal bar
	% %\fancyheadoffset{0\textwidth}        % center place top horizontal bar?
	% \renewcommand{\footrulewidth}{0pt}    %remove bottom horizontal bar
	% \usepackage{lastpage}
	\rightmark
	 % % \rhead{Student 659810}
	 % % \lhead{}
	% \cfoot{Page \thepage}
	% % \cfoot{Page \thepage\ of \pageref{LastPage}}

	

% Links and colours
	\usepackage[pdftex]{hyperref}
	\usepackage{xcolor}
	\hypersetup{
		colorlinks,
		linkcolor={black},
		citecolor={black},
		urlcolor=[RGB]{56 108 176}
	}

% Bibliography
	% \usepackage{natbib,har2nat} 
	% \setcitestyle{authoryear,open={(},close={)}}
	\usepackage[numbers]{natbib}
	% \bibliographystyle{spbasic2} 
	\bibliographystyle{unsrt_spbasic} 
	
	
% Lists, symbols nad matrices
	\usepackage{enumitem}
	\usepackage{amsmath}
	\usepackage{graphicx}
	\usepackage[retainorgcmds]{IEEEtrantools}
	\usepackage{multirow}
	\usepackage{rotating}
	\usepackage{changepage}
	\usepackage{dcolumn}
	\usepackage{booktabs}
	\usepackage{longtable}
	\usepackage[
	  outer=25mm,
	  inner=35mm,
	  vmargin=20mm,
	  includehead,
	  includefoot,
	  headheight=15pt,
	]{geometry}
	\usepackage{pdflscape}
	
%formatting miscellanies
	\renewcommand\thepart{\Alph{part}}
	\usepackage{capt-of}
	% \addtolength{\hoffset}{-0.5cm}
	% \addtolength{\textwidth}{1cm}

	\usepackage{setspace} %For double spacing


	\usepackage{todonotes} %For to do notes

\usepackage{mathptmx}
\usepackage{helvet}
\usepackage{courier}
%
\usepackage{type1cm}         

\usepackage{makeidx}         % allows index generation

\usepackage{multicol}        % used for the two-column index
\usepackage[bottom]{footmisc}% places footnotes at page bottom



	% \setlength{\parindent}{4em}  %remove all indenting of paragrapgs
	% \setlength{\parskip}{1em} 
	\usepackage{pdfpages} 
	\usepackage{upgreek}


	
%Not sure what this is
		% \makeatletter
		% \newcommand*{\gmshow@textheight}{\textheight}
		% \newdimen\gmshow@@textheight
		% \g@addto@macro\landscape{%
		  % \gmshow@@textheight=\hsize
		  % \renewcommand*{\gmshow@textheight}{\gmshow@@textheight}%
		% }
		% \def\Gm@vrule{%
		  % \vrule width 0.2pt height\gmshow@textheight depth\z@
		% }%
		\makeatother
		% \setlength{\parindent}{0pt}

	
%Custom commands
	% \newcommand{\plas}{\emph{Plasmodium}}

	\newcommand{\tdn}{\todo[inline, color=green!40]}

% Custom Colours
	\definecolor{farb1}{rgb}{27	 158 119 }
	\definecolor{farb2}{rgb}{217 95	 2   }
	\definecolor{farb3}{rgb}{117 112 179 }
	\definecolor{farb4}{rgb}{231 41	 138 }
	\definecolor{farb5}{rgb}{102 166 30  }
	\definecolor{farb6}{rgb}{230 171 2   }
	\definecolor{farb7}{rgb}{166 118 29  }
	\definecolor{farb8}{rgb}{102 102 102 }



\begin{document}


\chapter*{1. Introduction}
\markboth{1. Introduction}{}
\setcounter{chapter}{1}
\addcontentsline{toc}{chapter}{1. Introduction}

			% o	Summary of background and research methods; why this analysis/ evaluation is important; how and why the data was collected (initial study aims, collection methods, response rates); specifics of data (number, variables, outcomes)
			% Research question, aim, objectives of study
			
			% Research question:
			% What are the existing methods for estimating power to detect a difference in correlations between identical (monozygotic) and non-identical (dizygotic) twins, how do these compare and can they be improved upon? 

\large


% \begin{figure}[htbp]
% \sidecaption[t]
% %\centering
% \fbox{\includegraphics[scale=0.5]{figures/PapuaEndemicity.png}}
% %\picplace{5cm}{2cm} % Give the correct figure height and width in cm
% \caption{caption for figure.}
% \label{fig:pvpr}       % Give a unique label
% \end{figure}

\section{Background}
%% Intro
The classic twin study exploits the differing degrees of genetic relatedness in identical and non-identical twins in order to draw inferences on the heritability of traits.  The calculation and comparison of Pearson correlations across the two twin groups is a routine preliminary step undertaken by researchers in this field.  However, a range of factors can impact on researcher's ability to detect a true effect given data.  This thesis reports on a simulation-based power analysis for the detection of differences in correlations between identical and non-identical twin pairs under a range of scenarios, and the associated development of R functions and an applied interactive power calculator.  These tools address an identified absence of tools for this fundamental step in the twin study context.
 
\subsection{Twins and the classic twin study}
Identical twins arise from the same zygote, or fertilised egg, and are genetically very similar.  Non-identical twins arise from fertilisation of two seperate eggs, and are as genetically alike as ordinary siblings. 

The classic twin study can be used to estimate the proportion of variation in traits which may be attributable to genetics, or heritability.  

The estimates of heritability are conditional on key assumptions.  This chapter will define key genetic concepts, how these are exploited by the classic twin study to estimate the role of heridity for particular traits, and the assumptions which are relied upon in order to make such inferences.

The classic twin study exploits the differing degrees of genetic relatedness in identical and non-identical twins in order to draw inferences on the heritability of traits.  In broad terms, heritability is the degree to which variation in a trait or phenotype, such as propensity to gain body weight, or become a centenarian, can be attributed to genetics.

The comparison of phenotypic traits within identical and non-identical twin pair samples allows for partitioning the variance in traits into that attributable to  shared environment, individual environment or to genetics.  This allows us to better understand the mechanics of health and disease processes so that we can develop intervention measures which appropriately target the hypothesised causal mechanisms.

Contemporary twin studies use mixed effects and structural equation modelling to evaluate differences in variance components for a particular trait between mono and dizygotic twins accounting for differential within pair similarities related to zygosity.  A recent article reported on the development of R commands which can be used to estimate the power to detect a difference in correlations using such models.  

However, a routine preliminary step undertaken by researchers in this field is the calculation and comparison of Pearson correlations across the two twin groups.  This research project focuses on the reported needs of researchers undertaking this early analysis step.

In undertaking a twin study we make certain assumptions, understanding that these likely do not strictly hold in practice, but the key one for the purposes of this power analysis is that our data is approximately normally distributed.

\subsection{Historical background}
The statistical treatment of correlation was popularised by Francis Galton.  In context of broad social interest in eugenics, Galton described methods which could be used to describe the 'co-relatededness' of variables sourced from closely related family members \cite{Galton1888,Galton1890}.  

Karl Pearson was a keen follower of Galton's research, and writing in the context of inheritance and natural selection made use of 'Galton's function', describing it as a coefficient of correlation \cite{Pearson1895}.  

Ronald Fisher observed that a geometric transformation of the correlation cofficient using its inverse hyperbolic tangent could be used to approximate a normal distribution \cite{Fisher1915}.  This has the effect of mapping the distribution of correlation of coefficients from a domain of negative 1 through positive one to negative infinity through positive infinity. Being a simple and accurate approximation of the normal distribution, this transformation known as Fisher's Z is ubiquitous in statistical treatment of correlation coefficients, for example when seeking to compare their differences.

Florence Nightingale David was a protege of Pearson's who suggested that she prepare a volume of numerically accurate tables and interpolated plots of the distribution of the sample correlation coefficient r given n and the population correlation rho which could act as a standard against which to judge approximations such as that of Fisher \cite{David1938}. 
%  This visualisation of David's of the chance of rejecting the null hypothesis when true given alpha and rho and n was an influence on this project's presentation of results.

In the context of genetics and kinship, Douglas Falconer outlined methods of using comparison of identical and non-identical twins for estimating heritability, and noted some of the assumptions that this involves \cite{Falconer1960}.

Michael Neale and colleagues developed methods and software to facilitate the analysis of variance components in twin studies using structural equation modelling \cite{Neale1992}.  Brad Verhulst, building on the work of Peter Visscher, developed functions for power analyses in this variance component modelling context, for example detecting a difference in genetic correlations \cite{Visscher2004,Visscher2008a}.

This project is concerned with power to detect a difference in Pearson correlations as part of preliminary analysis before variance component modelling.

\subsection{Power analysis}
Power analysis involves a compromise between type 1 and type 2 error thresholds, respectively the expected proportion of null hypotheses to be rejected when true and not rejected when false.  These could be chosen to suit the requirements of a particular study, but for historical reasons the usual consensus is for 5\% and 20\%, and this is the parameterisation adopted for the results presented here.

A test statistic for the difference in correlations can be calculated as the difference in Fisher's Z transformed values weighted by the approximate standard error of the difference.  Fisher's Z transformation, the inverse hyperbolic tangent, maps the correlation coefficient from a domain of -1 through 1 to negative infinity through positive infinity, with approximate normal distribution.  This classic formulation is still used in functions found in Stata and R.

To estimate the power using this approach, you first take the difference between a normal reference score given the chosen type 1 error rate and the absolute value of the test statistic. The type 2 error rate is the probability of observing a value of at least this magnitude on the normal distribution.  And 1 minus this value is the power.

Using a simulation approach we take our hypothesis tests and apply them to draws from simulated data designed to mimic our samples through parameterisation using the hypothesised underlying bivariate population distributions.  

So where in our formula we might plug in anticipated or observed coefficients of .2 and .5, in the simulation we use these values to represent the supposed true correlations in the underlying population from which we draw our random samples.  Over a large number of simulations the proportion of hypothesis tests returning p-values lower than our type 1 error threshold is our power estimate.

 
 
%% Power
Power is the probability of detecting a true effect, a key consideration when planning a study.  In the context of differences in correlations undertaken in a twin study, we will expect a range of possible parameters to influence our power: the sample size; the ratio of MZ to DZ twins; how we have defined what constitutes a meaningful difference; the magnitude of the respective correlations; how these are measured.

Through my research project, I planned and carried out a power analysis for the detection of differences in correlations between identical and non-identical twin pairs investigating the influence of such factors.  

 \subsubsection{subsub section title}
 

			

\chapter*{2. Methods}
\markboth{Methods}{}
\setcounter{chapter}{2}
\setcounter{section}{0}
\addcontentsline{toc}{chapter}{2. Methods}

The preceding chapter summarised a review of the literature relating to twins, correlations, power and simulations undertaken to inform our approach to analysis.  Through this review we identified a series of suitable approaches and tests for evaluating differences in Pearson and Spearman correlations in two groups. Important considerations were the efficiency of our implemented simulation functions, and a well-designed data structure to support our planned as well as future outputs.  The $R$ programming environment was used for all analyses \cite{R2018}.

\section{Hypothesis tests for difference in correlations}
\subsection{Fisher's Z test (analytical approach)}
A test statistic for the difference in correlations can be calculated as the difference in Fisher's Z transformed values weighted by the approximate standard error of the difference \cite{Fisher1990,David1938}
$$t_{\hat{\theta}} = {\frac{\hat{\theta}}{se_\hat{\theta}}} = {\frac{z_{MZ} - z_{DZ}}{\sqrt{(n_{MZ}-3)^{-1}+(n_{DZ}-3)^{-1}}}} $$

The type 2 error rate $\beta$ is estimated through comparison of this test statistic $t_{\hat{\theta}}$ to a reference value on the standard normal distribution.   Using the cumulative normal distribution function $\Phi$ (Phi), the reference score $q$ is calculated as the normal probability quantile corresponding to our $\alpha$ level divided by the sidedness of our test. 
 $$q = \Phi^{-1}(\alpha/\text{sidedness})$$
Where we refer to sidedness, we mean whether we are concerned with a single- or two-tailed probability.  Here, we are testing the hypothesis that $\rho_{MZ} = \rho_{DZ}$ using a two-tailed p-value, implying 'admissible alternatives' to be the case that $\rho_{MZ}$ is greater than or less than $\rho_{DZ}$, that is, $|\rho_{MZ} - \rho_{DZ}| > 0$ \cite{David1938}.  One-tailed consideration, for example that $\rho_{MZ} > \rho_{DZ}$, is not considered in this report however, the functions developed may be parameterised in this way if desired. 
\\
Employing the concepts detailed above, $\beta$ is calculated as 
$$\beta = \Phi \big( q - t_\theta \big)$$
Our power estimate for the detection of difference in correlations is $\power(\theta) = 1 - \beta$. Putting the above altogether, we calculate power using the Fisher's $z$ test statistic as,

$$ \power = 1- \Phi \Bigg(\Phi_{\alpha/2}^{-1} -  \abs\bigg(  \frac{\arctanh(r_{MZ}) - \arctanh(r_{DZ})}{\sqrt{(n_{MZ}-3)^{-1} + (n_{DZ}-3)^{-1}}}  \bigg)  \Bigg) $$ 

The code we used to implement the analytic Fisher's Z test approach to power calculation in $R$ is displayed in listing \ref{lst:fz_nosim}.

\begin{lstlisting}[float=h,caption={Fisher's Z test (analytic approach)},label={lst:fz_nosim}]
# Fishers Z test - no sim
fz_nosim <- function(r1,r2,n1,n2,
                     alpha = 0.05, sidedness=2,method = "pearson",
                     power = TRUE) {
  # Calculate Fisher's Z
  z1     <- atanh(r1)
  z2     <- atanh(r2)
  
  # Take difference
  zdiff  <- z1-z2
  
  # Calculate standard error and test statistic
  z_se   <- sqrt(1/(n1-3) + 1/(n2-3))
  z_test <- zdiff/z_se
  
  # Optionally return p-value for observing diff at least this large under H0
  z_p    <- sidedness*pnorm(-abs(z_test))
  if (power == FALSE) return("p" = z_p)
  z_ref   <- qnorm(1-alpha/sidedness)
  z_power <- 1-pnorm(z_ref - abs(z_test))
  return(z_power)
}
\end{lstlisting}

The above method is the de facto standard, as used for example in the Stata \code{power two correlations} \cite{StataCorp2013}.  However, other options for evaluating the difference in Pearson or Spearman correlations should be considered.  

\subsection{Fisher's Z test (simulation approach)}
Using a simulation approach we take our hypothesis tests and apply them to draws from simulated data designed to mimic our samples through parameterisation using the hypothesised underlying bivariate population distributions.  
\\
\\
So where in our formula we might plug in hypothesised sample coefficients of 0.2 and 0.5, in the simulation we use these values to represent the true correlations in the underlying population from which we draw our samples.  Over a large number of simulations of bivariate twin data the proportion of hypothesis tests returning p-values lower than our type 1 error threshold is our power estimate.
\\
\\
The simulation-based Fisher's Z test function $R$ code is displayed in listing \ref{lst:fz}.

\begin{lstlisting}[float=h,caption={Fisher's Z test (simulation approach)},label={lst:fz}]
# Fishers Z test
fz <- function(a,b,sidedness=2,method = "pearson") {
  # Two samples
  n1 <- nrow(a)
  n2 <- nrow(b)
   
  # Compute z-transformed sample correlation coefficients
  z1     <- atanh(cor(a,method = method)[2,1])
  z2     <- atanh(cor(b,method = method)[2,1])
  zdiff  <- z1-z2
  
  # calculate standard error and test statistic
  z_se   <- sqrt(1/(n1-3) + 1/(n2-3))
  z_test <- zdiff/z_se
  
  # return p-value
  z_p    <- sidedness*pnorm(-abs(z_test))
  return(z_p)
}
\end{lstlisting}


In addition to applying the Fisher Z test in a simulation context, alternate tests we identified and implemented for inclusion in our simulation study were as follows. 

\subsection{Zou's confidence interval}
Zou's confidence interval is used to calculate a confidence interval for the difference in two correlations, and a hypothesis test employing this method is featured in the R package \code{cocor} \cite{Zou2007,Diedenhofen2015}.  A hypothesis test using Zou's confidence interval evaluates whether zero lies within the lower and upper bounds of the interval estimate of the difference in correlations, returning 1 if so or otherwise zero.  Over a run of simulations this would be expected to return identical results to the Fisher Z test, but may be more efficient.
\\
\\
Zou's approach expands on earlier work \cite{Olkin1995} to calculate a confidence interval for a difference in correlations using a so-called Simple Asymptotic approach, using what Zou refers to as a Modified Asymptotic method \cite{Zou2007}.  Both approaches draw heavily on Fisher's earlier work \cite{Fisher1990}.  The modified asymptotic method of Zou consists of first calculating confidence intervals for the two respective z-transformed correlations (transformed as per Fisher's method, described above):
$$(l_{z_k}, u_{z_k}) = z_k \pm \sqrt{\frac{1}{n_k - 3}} \times \Phi_{\alpha/2}^{-1},\ \text{where} \ k \in \{1,2\}$$

Then, the lower (L) and upper (U) bounds of the modified asymptotic confidence interval for the difference in correlations are calculated:
$$L = r_1 - r_2 - \sqrt{(r_1 - \tanh(l_{z_1}))^2 + (\tanh(u_{z_2})- r_2)^2}$$
$$U = r_1 - r_2 + \sqrt{(\tanh(u_{z_1}) - r_1)^2 + (r_2 - \tanh(l_{z_2}))^2}$$

If zero is within the bounds of the confidence interval for the difference, the test returns as 1, and otherwise 0.
\\
\\
Our implemenation of the Zou's confidence interval test function is displayed in listing \ref{lst:zou}.

\begin{lstlisting}[float=h,caption={Zou's confidence interval},label={lst:zou}]
zou <- function(a,b,alpha = 0.05,sidedness=2,method = "pearson") {
  # From Zou (2007) and used in Cocor (note typo for U in paper; should be '+')
  #  However, really, this is equivalent to fz test for hypothesis testing purposes
  
  # compute z- transformed correlations and differences
  r  <- c(cor(a,method = method)[2,1], cor(b,method = method)[2,1])
  z  <- atanh(r)
  zdiff  <- z[1]-z[2]
  
  # calculate standard error for respective z scores
  n  <- c(nrow(a),nrow(b))
  z_se   <- sqrt(1/(n-3))

  # calculate reference threshold
  z_ref  <- qnorm(1-alpha/sidedness)
  
  # calculate respective confidence intervals
  ci_mat <- matrix(c(-1,-1,1,1),nrow = 2, ncol = 2, dimnames =list(c("Mz","Dz"),c("l","u")))
  z_ci   <- z + ci_mat * z_se * z_ref
  r_ci   <- tanh(z_ci)
  
  # calculate Zou's Modified Asymptoptic confidence interval for difference in correlations
  L      <- r[1]-r[2] - sqrt((r[1]      - r_ci[1,1])^2 + (r_ci[2,2] - r[2]     )^2)
  U      <- r[1]-r[2] + sqrt((r_ci[1,2] - r[1]     )^2 + (r[2]      - r_ci[2,1])^2)
  r_diff_ci <- c(L,U)
  
  # return test value (0 or 1, however, in the power context this resolves to same outcome as p)
  ci_test <- (L < 0) && (0 < U)
  return(c(ci_test,r_diff_ci))
}
\end{lstlisting}



\subsection{Generalised Variable Test}

The generalised variable (GV) test involves transformation of the simulated sample correlations into so-called pivotal quantities the difference of which is used to calculate a test statistic and p-value \cite{Krishnamoorthy2014}. Synthesising two reported approaches \cite{Krishnamoorthy2007,Kazemi2016} this test was first implemented as an example by my supervisor Enes Makalic in a Matlab script, and subsequently adapted by myself as a function in R.  A compiled version using RCPP to leverage C++ routines for random number draws was suggested by my colleague Koen Simons, and adopted to improve the function's run time. However, this later version was not compatible with the parallelised simulation approach, and in this context the non-RCCP 'GVT-r' version was used.
\\
\\
Given two bivariate normal samples $k\in\{1,2\}$, the sample correlation coefficients $r_k$ are used to estimate two respective quantities $r_k^* = \frac{r_k}{\sqrt(1-r_k^2)}$, and the generalised variables $G_{\rho_k}$:
$$G_{\rho_k} = \frac{r_k^*\sqrt{W_k} - U_k}{\sqrt{(r_k^*\sqrt(W_k) - U_k)^2 + V_k}}$$
where,
$$U_k \sim N(0,1) ,\ V_k \sim \chi_{n_k - 1}^2 ,\ \text{and} \ W_k \sim \chi_{n_k-2}^2$$
A p-value using the GV test is calculated as twice the value of the smaller of two quantities: the proportion of differences in $G_{\rho_k}$ less than 0, and the proportion greater than 0.

The GV test function $R$ code is displayed in listing \ref{lst:gvtr}.

\begin{lstlisting}[float=h,caption={GV test (R version)},label={lst:gvtr}]
gvt_r <- function(a,b,M=1e4,method = "pearson") {
  # Two samples
  n1 <- nrow(a)
  n2 <- nrow(b)
  
  # Compute sample correlation coefficients
  r1 <- cor(a,method = method)[2,1]
  r2 <- cor(b,method = method)[2,1]
  r  <- c(r1,r2)
  
  # Generate random numbers
  V2     <- matrix(data=0, nrow = M, ncol = 2)
  V2[,1] <- rchisq(M, df = n1-1, ncp = 1)
  V2[,2] <- rchisq(M, df = n2-1, ncp = 1)
  
  W2     <- matrix(data=0, nrow = M, ncol = 2)
  W2[,1] <- rchisq(M, df = n1-2, ncp = 1)
  W2[,2] <- rchisq(M, df = n2-2, ncp = 1)
  
  Z <-matrix(data = rnorm(2*M), nrow=M, ncol = 2)
  
  # Compute test statistic
  rstar <- r/sqrt(1-r^2)
  top   <- c(sqrt(W2[,1])*rstar[1],sqrt(W2[,2])*rstar[2]) - Z
  G     <- top / sqrt( top^2 + V2 )
  
  # Compute p value
  Grho <- G[,1] - G[,2];
  p    <- 2*min( mean(Grho<0), mean(Grho>0) ); 
  return(p)
}
\end{lstlisting}

\subsection{Signed log-likelihood ratio test}
The signed log likelihood ratio (SLR) test is formulated as the signed difference in sample correlation coefficients multiplied by the square root of the sum of respective coefficients' log-likelihoods.  The test here is a partial implementation of a recently reported modified signed log-likelihood ratio (MSLR) test  for differences in two bivariate normal correlations \cite{Kazemi2016}. The SLR and MSLR tests are well established general hypothesis tests \cite{Barndorff1986,Barndorff1991,Diciccio2001,Krishnamoorthy2014}, the novelty in Kazemi and Jafari's approach being the applied context of difference in correlations. However, we (myself, nor my supervisors) were unable to successfully replicate the 'modified' portion of Kazemi and Jafari's reported algorithm.  Due to time constraints, and noting that the 'unmodified' SLR test appeared to return p-values similar to the other hypothesis tests it was decided that inclusion of the SLR test would be a valid option to consider.
\\
\\
The SLR test function $R$ code is displayed in listing \ref{lst:slr}.

\begin{lstlisting}[float=h,caption={Signed log-likelihood ratio test},label={lst:slr}]
slr <- function(a,b,M=1e4,sidedness=2,method = "pearson") {
  # Signed Log-likelihood Ratio test (an 'unmodified' version of test 
  # described in Krishnamoorthy and Lee, Kazemi and Jafari , DiCiccio etc)
  # Two samples
  n  <- c(nrow(a),nrow(b))
  
  # Compute z-transformed sample correlation coefficients
  r  <- c(cor(a,method = method)[2,1], cor(b,method = method)[2,1])
  z  <- atanh(r)
  
  # Calculate average z as a plug in value
  rf <- tanh(mean(z))
  
  # calcaulte SLR
  slr <-sign(r[1]-r[2])*sqrt(sum(n*log(((1-rf*r)^2)/((1-r^2)*(1-rf^2)))))
  
  # return p-value
  p    <- 2 * (1 - pnorm(abs(slr))); 
  return(p)
}
\end{lstlisting}

\subsection{Permutation test}
The permutation test is a non-parametric approach which compares the absolute 
difference of the Z-transformed sample correlations with those using correlations 
from a series of group membership permutations using the sample rank orders as 
values.  Under a hypothesis of no difference in correlation, those differences arising from permutations would be assumed to be equally likely as those observed, or anticipated to be observed \cite{Efron1993}.  Across a series of $M$ permutations (in this study, 10,000), a $p$-value is calculated as the proportion of permutation derived absolute differences (($\abs(z_{MZ}^* - z_{DZ}^*)$)) of greater magnitude than $\abs(z_{MZ} - z_{DZ})$.
\\
\\
The implementation of this permutation test in $R$ is displayed in listing \ref{lst:pt}.

\begin{lstlisting}[float=h,caption={Permutation test},label={lst:pt}]
pt <- function(a,b,M=1e4,sidedness=2,method = "pearson") {
  # Based on Efron and Tibshirani, 1993
  # Store size, and calculate z-transformed correlations
  n  <- c(nrow(a),nrow(b))
  r  <- c(cor(a,method = method)[2,1], cor(b,method = method)[2,1])
  z  <- atanh(r)
  
  # Store rank-ordered vector representations, in one column
  v  <- cbind(rank(rbind(a[,1],b[,1]),ties.method = "random"),
              rank(rbind(a[,2],b[,2]),ties.method = "random"))
  # label rows
  rownames(v) <- c(rep("A",n[1]),rep("B",n[2]))
  
  # initial empty test vector
  rtest <- numeric(0)
  
  # run M permutations (default is 10,000),
  #  - returns test that absolute magnitude of difference
  #    is at least as great as that of the input z-transformed corr. diff.
  for (i in 1:M){
    permute <- cbind(v,rbinom(sum(n),1,0.5))
    rstar   <- c(cor(permute[permute[,3]==0,c(1,2)],method = method)[2,1],
                 cor(permute[permute[,3]==1,c(1,2)],method = method)[2,1])
    zstar   <- atanh(rstar)
    rtest   <- c(rtest,
                 abs(zstar[1]-zstar[2]) > abs(z[1]-z[2]))
    } 
  
  # return p-value: proportion of test results at least as large as obs'd
  p <- mean(rtest)
  return(p)
}
\end{lstlisting}

\section{Simulation}
\subsection{Approach for one simulation}
A function \code{corr\_diff\_test()} was developed to undertake a single comparative simulation of any of the above tests (listing \ref{lst:corr_diff}). Within a single simulation, each simulation based  test is evaluated using the same samples drawn from the two simulated bivariate populations as input, returning a $p$-value.  The analytic Fisher Z test returns either a $p$-value or a power estimate based on the population parameters.  Note that in the detailing of computational aspects of our methodology we use the term 'parameter' to refer to one of the options which may be specified within a function; an argument is the specific value which is passed to that parameter. For example, the (population) parameter $\rho$ for the respective MZ and DZ groups may be defined by specifying the argument \code{rho = c(-0.65,0.2)}. The main parameters which can be specified in the function call to \code{corr\_diff\_test()} are described with example arguments in table \ref{table:corr_params}.  In the following text we refer to the set of parameters that give rise to a simulation as a scenario.

\begin{lstlisting}[float=h,caption={Single run simulation code},label={lst:corr_diff}]
corr_diff_test <- function(rho = c(.2,.5), n = c(30,90), distr = "normal",
                    param1a = c(0,0), param1b = c(0,0),param2a = c(1,1), param2b = c(1,1),
                    alpha = 0.05, sidedness = 2, test = c("fz","gtv","pt","slr","zou"),
                    method ="pearson", lower.tri = FALSE) {
  if(lower.tri==TRUE){
    # optionally, only calculate results for lower matrix half 
    #   when comparing across all correlation combinations
    if(rho[1] < rho[2]) { 
      return(NA)
    }
  }
  # initialise empty results vector
  results <- list()
  
  # if requested, process analytical Fisher's Z
  if ("fz_nosim" %in% test) {
    results[["fz_nosim"]] <- fz_ns_compiled(rho[1],rho[2],n[1],n[2], 
                                      alpha = 0.05, sidedness = 2, method = method, power = FALSE)
    if(length(test)==1) return(results)
  }
  # process selected hypothesis tests, each using same draw of simulated data
  require("simstudy")
  a <- genCorGen(n[1], nvars = 2, params1 = param1a, params2 = param2a,  
                dist = distr, corMatrix = matrix(c(1, rho[1], rho[1], 1), ncol = 2), 
                wide = TRUE)[,2:3]
  b <- genCorGen(n[2], nvars = 2, params1 = param1b, params2 = param2b,  
                dist = distr, corMatrix = matrix(c(1, rho[2], rho[2], 1), ncol = 2), 
                wide = TRUE)[,2:3]
  if ("fz"       %in% test) results[["fz"]]       <- fz_compiled(a,b)
  if ("gtv"      %in% test) results[["gtv"]]      <- gtv(a,b) # uses rccp ; so elsewise compiled
  if ("gtvr"     %in% test) results[["gtvr"]]      <- gtv_compiled(a,b) 
  if ("pt"       %in% test) results[["pt"]]       <- pt_compiled(a,b)
  if ("slr"      %in% test) results[["slr"]]      <- slr_compiled(a,b)
  if ("zou"      %in% test) results[["zou"]]      <- zou_compiled(a,b)[1]
  return(rbind(results[test]))
}
\end{lstlisting}

The function uses the $R$ package \code{simstudy} function \code{genCorGen} to generate bivariate correlated data for the simulated MZ (group \code{a} in the code above) and DZ (group \code{b}) twin pair samples \cite{simstudy2018}.  The choice of available distributions and parameterisations is normal($\mu,\sigma$), binomial(probability $p$), Poisson(rate $\lambda$) gamma($\mu,\text{dispersion} \ k$), or uniform(min, max).  In addition to specifying sample size, distribution and parameterisation, a correlation matrix may be specified; this was used to parameterise the underlying population correlations from which bivariate samples should be drawn.  Three distinct distribution types were modelled in our simulation based power analysis: normal, 'mild' skew and 'extreme' skew. These are respectively explained in the captions of Figures \ref{fig:dist_norm}, \ref{fig:dist_gamma1}, and \ref{fig:dist_gamma2}, which illustrate example sample draws from these distributions.  

\begin{figure}[htbp]
\sidecaption[t]
%\centering
\fbox{\includegraphics[scale=0.52]{{../figs/distx_normal_60_120}.pdf}}
%\picplace{5cm}{2cm} % Give the correct figure height and width in cm
\caption{An example of the bivariate normal scenario, with distributional assumptions asymptotically met.  Both variables are standardised with mean $\mu=0$ and standard deviation $\sigma=1$.}
 % and distribution $\sim N\bigg{\boldsymbol\mu = \begin{pmatrix} 0 \\ 0 \end{pmatrix}, \quad \boldsymbol\Sigma = \begin{pmatrix} 1 & \rho \\ \rho  & 1 \end{pmatrix} \bigg)$.
% \caption{An example of the bivariate normal scenario, under which our distributional assumptions are asymptotically met.  This was specified with both variables standardised having mean $\mu=0$ and standard deviation $\sigma=1$ and distribution $\sim N\bigg{\boldsymbol\mu = \begin{pmatrix} 0 \\ 0 \end{pmatrix}, \quad \boldsymbol\Sigma = \begin{pmatrix} 1 & \rho \\ \rho  & 1 \end{pmatrix} \bigg)$.}
\label{fig:dist_norm}       % Give a unique label
\end{figure}

\begin{figure}[htbp]
\sidecaption[t]
%\centering
\fbox{\includegraphics[scale=0.52]{{../figs/distx_gamma_mildskew_60_120}.pdf}}
%\picplace{5cm}{2cm} % Give the correct figure height and width in cm
\caption{A 'mild skew' scenario based on a gamma distribution with mean 1.5 and dispersion 0.09 (which the genCorGen function uses to inform shape and scale parameters for the distribution).  This parameterisation was chosen through experimentation with the intent to represent a mild departure from an assumed normal population distribution, with a slight positive skew}
\label{fig:dist_gamma1}       % Give a unique label
\end{figure}

\begin{figure}[htbp]
\sidecaption[t]
%\centering
\fbox{\includegraphics[scale=0.52]{{../figs/distx_gamma_extrskew_60_120}.pdf}}
%\picplace{5cm}{2cm} % Give the correct figure height and width in cm
\caption{An 'extreme skew' scenario based on a gamma distribution with mean 1 and dispersion 5.  This results in an extreme positive skew to the distribution, analogous to that of biological processes where most observations will be clustered around a certain value, however some outliers may be extremely elevated.}
\label{fig:dist_gamma2}% Give a unique label
\end{figure}


\begin{table}\centering
\caption{Description of parameter options for single simulation \label{table:corr_params}}
\begin{tabular}{cll}
  \toprule
  \textbf{Parameter} & \textbf{Description} & \textbf{Example arguments} \\ [0.5ex] 
  \midrule
  \code{method}    & Correlation method to use for testing difference                  & \code{'pearson'}                 \\
  \code{rho}       & $\rho$ for each group's bivariate distribution                    & \code{c(-0.21,0.59)}             \\
  \code{n}         & Sample size for groups 1 (MZ) and  2 (DZ)                         & \code{c(30,60)}                  \\
  \code{dist}      & Distribution to be used for both groups' bivariate distribution \ & \code{'normal'}                  \\
  \code{param1a}   & Distribution parameter 1 for respective samples in group 1        & \code{c(0,0)}                    \\
  \code{param1b}   & Distribution parameter 1 for respective samples in group 2        & \code{c(0,0)}                    \\
  \code{param2a}   & Distribution parameter 2 for respective samples in group 1        & \code{c(1,1)}                    \\
  \code{param2b}   & Distribution parameter 2 for respective samples in group 2        & \code{c(1,1)}                    \\
  \code{test}      & Tests to be evaluated                                             & \code{c("fz\_nosim","fz","gtv")} \\
  \code{alpha}     & $\alpha$ value to use for hypothesis tests                        & \code{0.05}                      \\
  \code{sidedness} & Sidedness for hypothesis tests                                    & \code{2}                         \\
  \bottomrule 
\end{tabular}
\end{table}

% whether only to compute results for where $rho_1 < rho_2$, which across a series of correlation combinations conceived as a matrix can be used to optionally returne only the lower triangular matrix of results.

\subsection{Approach for multiple simulations}
The function described above runs a single simulation of drawing from samples from two bivariate populations.  However, for asymptotic normality to hold --- the long run approximation of a normal distribution due to the Central Limit Theorem \cite{Casella2002} --- we know we must run many more simulations to achieve a fair assessment of the proportion of null hypotheses rejected when false for when using a particular hypothesis test under a particular scenario (ie. set of parameterisations).
\\
\\
A wrapper function \code{corr\_power} to allows for the \code{corr\_diff\_test()} function to be called $M$ times for a given scenario, returning a power estimate for each test specified in the function call.  These power estimates are derived from the series of simulated $p$-values for each test considered, or directly in the case of the analytic Fisher's Z test.

\section{Scenario combinations}
The flexibility of the simulation commands we defined allows for an very broad array of scenarios to be considered.  A researcher could use these tools as is to aid in the development of a statistical analysis plan for their study.  For this report we had to decide on a limited subset of these possibilities.  There are two drivers for the choice made: the number of scenario combinations, and the time taken to run each of these.  Our initial plan for the series of scenarios is reported in table \ref{table:combos}.
\\

\begin{table}\centering
\caption{Description of parameter options for single simulation \label{table:combos}}
\begin{tabular}{llr}
  \toprule
  \textbf{Parameter} & \multicolumn{2}{c}{\textbf{Initial plan}} \\ 
  \cmidrule(lr){2-3} 
   & \textbf{Argument resolution} & \textbf{Combinations}	\\
  \midrule
  \code{method}    & Pearson and Spearman correlations                                 & 2 \\
  \code{rho}       & $\rho$ combinations\: -0.95 through 0.95 at 0.5 resolution         & $39^2 = 1521$ \\
  \code{n}         & group combinations\: 15, 30, 60, 120, 240, 480, 960                      & $7^2 = 49$ \\
  \code{dist}      & normal, 'mild skew', 'extreme skew'                               &  3 \\
  \code{param1a}   & dictated by distribution choice, above                            & -  \\
  \code{param1b}   & dictated by distribution choice, above (equal to param1a)         & -  \\
  \code{param2a}   & dictated by distribution choice, above                            & -  \\
  \code{param2b}   & dictated by distribution choice, above (equal to param2a)         & -  \\
  \code{test}      & Fisher's Z (analytic and sim), Zou's CI, GVT, SLR, PT             & 6  \\
  \code{alpha}     & .05                                                               & 1  \\
  \code{sidedness} & 2                                                                 & 1  \\
  \midrule
  Total scenarios &                                                            & 2,683,044  \\
  Total simulations & 1,000 simulations for each scenario                   & 2,683,044,000 \\
  Total simulations & 100, 1,000 and 10,000 simulations for each scenario (unrealistic!) &  29,781,788,400 \\
  \bottomrule 
\end{tabular}
\end{table}

\\
We decided \textit{a prior} to set some parameters as fixed, : we only conducted two-sided tests with $\alpha$ of 0.05;  having decided to limit ourselves to three basic distributional forms (normal, and two kinds of non-normal using distinct gamma distribution parameterisations) the distribution parameters were fixed to achieve these forms; the respective simulations of MZ and DZ twin pair samples each use the same distributional form, although the population correlation and sample size may vary (e.g. we don't compare bivariate normal MZ with a gamma skewed DZ); the simulated bivariate data for each twin group shares the same parameterisation (e.g. for normal data both variables have $\mu$ 0 and $\sigma$ 1).  Nevertheless, we were aware that our intial plan would be over-ambitious: were each scenario to be processed consecutively, each taking 1 second to process at a steady rate optimistically with no computer crashes we might expect 1,000 simulations of each scenario in Table \ref{table:combos} to take $2,683,044,000/60/60/24/365 \approx 85$ years!  Under advice from my colleague Koen Simons, I undertook time tests of 1,000 iterations of each function and their byte code compiled versions, employing only the most efficient versions.  The permutation test implementation was particularly inefficient, and given time constraints for refactoring code this was abandoned.  Noting that the resolution of correlation combinations was a major contributor to anticipated length of processing time, this was reduced to comparison of correlations from -0.9 through 0.9 at a 0.1 resolution resulting in 361 instead of 1,521 correlation combinations.  Based on the preliminary time tests, the initial and revised time estimates are displayed along with function time results in Table \ref{{table:times}}.  These estimates suggested an anticipated running time of 16 days, based on my personal Core2Duo i7 laptop with 16gb RAM.
\\

\begin{table}\centering
\caption{Description of parameter options for single simulation \label{table:times}}
\begin{tabular}{rcc}
  \toprule
  \textbf{Test} &	\multicolumn{2}{c}{\textbf{Time/1000 runs (secs)}} \\
  \cmidrule(lr){2-3} 
   & \textbf{as is} & \textbf{compiled}				\\
  \midrule										
  Fisher's z (no sim) &         0.03 &    0.02 \\
  Fisher's Z          &         0.56 &    0.36 \\
  GTV (r)             &        18.65 &   16.55 \\
  GTV (R C++)         &              &   12.08 \\
  Permutation (PT)    &      2473.18 & 2341.62 \\
  SLR                 &         0.49 &    0.33 \\
  Zou's CI            &         0.38 &    0.28 \\
  \midrule
  \multicolumn{1}{l}{Total 1 (PT, R-compiled GTV)}       &	2493.29 & 2359.16  \\
  \multicolumn{1}{l}{Total 2 (No PT, RCCP-compiled GTV)} &	       	& 13.07    \\
  \multicolumn{1}{l}{Total 3 (No PT, R-compiled GTV)}    & 	20.11	  & 17.54    \\
  \multicolumn{1}{l}{Initial scenario: Total 1 $\times 1521 \times 49 \times 3 \times 2/60/60/24/365$}  & 35 years  & 33 years \\
  \multicolumn{1}{l}{Best scenario: Total 2 $\times 361 \times 49\times 3 \times 2 /60/60/24$}  & 25 days  & 16 days \\
  \multicolumn{1}{l}{Next best scenario: Total 3 $\times 361 \times 49\times 3 \times 2 /60/60/24$}  &    & 21 days \\
\bottomrule 
\end{tabular}
\end{table}
\\
All parameterisations giving rise to a particular scenario were recorded to ensure accurate scenario retrieval and ability to interpolate power estimates based on a subset of scenarios.  The initial format we used in our post-time estimation run was the R \code{data.table} format, as it was reportedly well optimised 'under the hood'.  The data.table was set up in wide format with rows per scenario (106,134 rows) and columns per parameter with five additional columns to contain power estimates per test (13 columns, once processed).  This format was used to process sets of results using 100 simulations, and select scenarios using 1,000 simulations and 10,000 simulations.  The 100 simulation set of results was used as initial proof of concept for the approach; the scenarios using 1,000 results provide the main results for the inferences made in this report; the 10,000 simulations scenarios were produced in order to evaluate sensitivity of simulated power estimates to the number of simulation runs, and in particular establish validity of the choice of 1,000 simulations for main results.

\section{Evaluating power}
The results of the simulation analysis using the above options are comprised of more than half a million power estimates, when taken in long form.  These can be accessed to answer specific questions.  The task of this report is to evaluate subsets of these and demostrate the value and validity of the functions giving rise to them, and provide options to researchers such as those of Twins Australia for their analyses which may be expanded upon.


 comparisons vis impact on power and required sample size to achieve this: number
of simulations (justifies 1000 simulation approach); choice of ratio; choice of test method; impact of
non-normality; magnitude of correlations; etc. I have put more old style contour plots in appendix,
which may be referred to � specific examples for GTV and SLR


results in format of images
\\
\\
contour plot
\\
\\  
interpolation using monotonic increasing spline function
\\
\\
Required sample size to achieve 80% power
\\
\\  
Difference to achieve 80% power

\chapter*{3. Results}
\markboth{Results}{}
\setcounter{chapter}{3}
\setcounter{section}{0}
\addcontentsline{toc}{chapter}{3. Results}

\section{section title}


    \begin{itemize}
      \item 530,670 power estimates comparing: tests, sample size, group size ratio, normality, correlation combinations, correlation methods
      \item Under approximate normality / mild skew, on average: \\SLR test $ \sim 82\%$ power; others $\leq 75\%$. 
    \end{itemize}
    
    
    Power to detect $\hat\delta_{\rho}$ in MZ and DZ twins $\sim \mathcal{N}\bigg(\left[\begin{smallmatrix}0\\ 0\end{smallmatrix}\right],\, \left[\begin{smallmatrix}1 & {\rho} \\ {\rho} & 1\end{smallmatrix}\right] \bigg)$
    
     \begin{center}
      \begin{figure}[!htb]
        \minipage{0.49\textwidth}
          \includegraphics[width=\linewidth]{{../figs/corrxplot_slr_example}.pdf}
        \endminipage\hfill
        \minipage{0.49\textwidth}%
          \includegraphics[width=\linewidth]{{../figs/power_n_plot_example}.pdf}
        \endminipage
      \end{figure}
    \end{center}
    
    
   % \begin{itemize}
      % \item 530,670 power estimates comparing: tests, sample size, group size ratio, normality, correlation combinations, correlation methods
      % \item Under extreme skew: \\Fisher's Z formula underestimates req. sample size by $ \sim 50\%$
    % \end{itemize}
    % \linebreak
    % \small{Power to detect $\hat\delta_{\rho}$ in MZ and DZ twins $\sim \mathcal{G}\bigg(\left[\begin{smallmatrix}1\\ 1\end{smallmatrix}\right],\, \left[\begin{smallmatrix}5 & {\rho} \\ {\rho} & 5\end{smallmatrix}\right] \bigg)$}
    % \begin{center}
      % \begin{figure}[!htb]
        % \minipage{0.49\textwidth}
          % \includegraphics[width=\linewidth]{{../figs/corrxplot_slr_example2}.pdf}
        % \endminipage\hfill
        % \minipage{0.49\textwidth}%
          % \includegraphics[width=\linewidth]{{../figs/power_n_plot_example2}.pdf}
        % \endminipage
      % \end{figure}
    % \end{center}
\begin{landscape}   
  \begin{figure}[htb]
    \centering        
    \minipage{0.25\textwidth}%
      \includegraphics[width=\linewidth]{{../figs/distx_normal_60_120}.pdf}
      % \caption{Example bivariate normal draw with $\rho(0.2,0.5)$}
      % \label{fig:1}
    \endminipage\hfill       
    \minipage{0.25\textwidth}%
      \includegraphics[width=\linewidth]{{../figs/corrx_npower_norm_r.2_r.5_mzdz.5_s100}.pdf}
      % \caption{Sample size estimate: 100 simulations}
      % \label{fig:1}
    \endminipage\hfill
    \minipage{0.25\textwidth}% 
      \includegraphics[width=\linewidth]{{../figs/corrx_npower_norm_r.2_r.5_mzdz.5_s1000}.pdf}
      % \caption{Sample size estimate: 1,000 simulations}
      % \label{fig:2}
    \endminipage\hfill
    \minipage{0.25\textwidth}% 
      \includegraphics[width=\linewidth]{{../figs/corrx_npower_norm_r.2_r.5_mzdz.5_s10000}.pdf}
      % \caption{Sample size estimate: 10,000 simulations}
      % \label{fig:3}
    \endminipage\hfill
    \caption{Left to right: Example bivariate normal draw with $\rho(0.2,0.5)$, and sample size estimates to achieve 80\% power using 100, 1000, and 10000 simulation runs per scenario}
    \label{fig:images1}
    
    \medskip
    \minipage{0.25\textwidth}% 
      \includegraphics[width=\linewidth]{{../figs/distx_gamma_mildskew_60_120}.pdf}
      % \caption{Example bivariate gamma ("mild skew") draw with $\rho(0.2,0.5)$}
      % \label{fig:4}
    \endminipage\hfill
    \minipage{0.25\textwidth}% 
      \includegraphics[width=\linewidth]{{../figs/corrx_npower_mildskew_r.2_r.5_mzdz.5_s100}.pdf}
      % \caption{Sample size estimate: 100 simulations}
      % \label{fig:4}
    \endminipage\hfill
    \minipage{0.25\textwidth}% 
      \includegraphics[width=\linewidth]{{../figs/corrx_npower_mildskew_r.2_r.5_mzdz.5_s1000}.pdf}
      % \caption{Sample size estimate: 1,000 simulations}
      % \label{fig:5}
    \endminipage\hfill
    \minipage{0.25\textwidth}% 
      \includegraphics[width=\linewidth]{{../figs/corrx_npower_mildskew_r.2_r.5_mzdz.5_s10000}.pdf}
      % \caption{Sample size estimate: 10,000 simulations}
      % \label{fig:6}
    \endminipage\hfill
    \caption{Left to right: Example bivariate gamma ("mild skew") draw from population with $\rho(0.2,0.5)$, and sample size estimates to achieve 80\% power using 100, 1000, and 10000 simulation runs per scenario}
    \label{fig:images2}
    
    \medskip
    \minipage{0.25\textwidth}% 
      \includegraphics[width=\linewidth]{{../figs/distx_gamma_extrskew_60_120}.pdf}
      % \caption{Example bivariate gamma ("extreme skew") draw with $\rho(0.2,0.5)$}
      % \label{fig:4}
    \endminipage\hfill
    \minipage{0.25\textwidth}% 
       \includegraphics[width=\linewidth]{{../figs/corrx_npower_extrskew_r.2_r.5_mzdz.5_s100}.pdf}
      % \caption{Sample size estimate: 100 simulations}
      % \label{fig:4}
    \endminipage\hfill
    \minipage{0.25\textwidth}% 
      \includegraphics[width=\linewidth]{{../figs/corrx_npower_extrskew_r.2_r.5_mzdz.5_s1000}.pdf}
      % \caption{Sample size estimate: 1,000 simulations}
      % \label{fig:5}
    \endminipage\hfill
    \minipage{0.25\textwidth}% 
      \includegraphics[width=\linewidth]{{../figs/corrx_npower_extrskew_r.2_r.5_mzdz.5_s10000}.pdf}
      % \caption{Sample size estimate: 10,000 simulations}
      % \label{fig:6}
    \endminipage\hfill
    \caption{Left to right: Example bivariate gamma ("extreme skew") draw from population with $\rho(0.2,0.5)$, and sample size estimates to achieve 80\% power using 100, 1000, and 10000 simulation runs per scenario}
    \label{fig:images3}
  \end{figure}
\end{landscape}
\chapter*{4. Discussion}
\markboth{Discussion}{}
\setcounter{chapter}{4}
\setcounter{section}{0}
\addcontentsline{toc}{chapter}{4. Discussion}
% put in comments for each paragraph to remind yourself to stay on topic!!

% \cite{Price2007,Rahimi2014,Naing2014}

\section{section title}
Despite the reduced parameter set, our results are comprised of more than half a million power estimates which can be accessed to answer specific questions.

While the main use of such results are for considering particular scenarios, we can average over these for marginal estimates; under approximate normality the SLR test was estimated to have 82% power on average, while the other tests had approximately 75%.  This result which is reflected in this plot here for example, with the higher power estimate from SLR estimate suggesting a smaller required sample size, is surprising.

The fitted contour plot on the left here compares power to detect a difference using all correlation combinations given groups sizes, in this example, under a bivariate normal distribution.  The blue line indicates the 80% power threshold.

On the right sample size estimates to achieve 80% power given population correlation coefficients of 0.2 and 0.5 are compared across the 5 implemented tests.  Stata produces two correlation power estimates with plots like these using the Fisher Z formula.  We often assume that our tests are robust to departures from normality, but it is interesting to ask how would this perform if normality assumption were severely violated?

We can see here the extreme under estimation in sample size required to achieve 80% power given an extreme positively skewed bivariate distribution when using the formula based approach compared with the simulation-based tests. 

The SLR test appears a strong performer here - but its systematic elevation at tails of distribution is concerning -- does this reflect a systematic bias, and perhaps innacurate rather than truly improved performance?  

When no difference in correlations is present under bivariate normal distribution we would expect power equivalent to our alpha level of 0.05, which, approximately, the other tests have; however, the SLR test has approximately 20\%.  This suggests its estimates here are upwardly biased.

After the SLR test the GTV was the next most powerful, but only marginally moreso than the Fisher's Z based tests.



We developed a flexible and extensible architecture geared to solving future problems.


However, the programming and analysis took longer than anticipated.  The results I have shown today were processed using 100 simulations per parameter combination.  The 1000 simulation results should be completed to update my results in a day or so.

There is more we can do to make this of more particular use in the twin context, for example by evaluating correlations in simulations using multivariable regression methods we could account for partial correlations; if we did this using mixed effects methods we could also consider power for difference in intra-class correlations.

Improve efficiency to allow higher resolution estimation

\chapter*{5. Conclusions}
\markboth{Conclusions}{}
\setcounter{chapter}{5}
\setcounter{section}{0}
\addcontentsline{toc}{chapter}{5. Conclusion}

To sum up, 
\\
tests were overall quite similar
\\
simulation important for power analysis where notable violation of assumptions of bivariate normality is anticipated  (naive application of analytical Fisher's Z formula can be extremely over-optimistic in required sample estimates)
\\
SLR implementation is biased\; highlights importance of critical consideration power of methods, as higher power may reflect bias.  For this particular SLR test, power estimates in the case of unequal group ratios were upwardly biased.
\\
Power should be considered contextualised using planned conditions for study and subject matter knowledge, and critical consideration of impact of methods employed.
\\
We have created both the architecture for a process, as well as a database of simulation scenarios that can be interrogated.  Both can be expanded as required. I have trialled an interactive power calculator web app, and it is planned incorporate the pre-processed database into this to allow on the fly estimates informed by our pre-processed results.
 %%%%%%%%%%%%%%%%%%%%%%%%%%%%%%%%%%%%%%%%%%%%%%%%%%%%%%%%%%%%%%%
% From Springer template
% 
% 
%
%%%%%%%%%%%%%%%%%%%%%%%% Springer %%%%%%%%%%%%%%%%%%%%%%%%%%

\Extrachap{Glossary}
\markboth{Glossary}{}
\runinhead{correlation} A measure of the magnitude and direction of a linear relationship shared by two variables.
\runinhead{dizygotic} Non-identical twins arising from fertilisation of two seperate fertilised eggs, and as genetically alike as ordinary siblings.
\runinhead{heritability} The degree to which variation in a trait or phenotype, such as propensity to gain body weight, or become a centenarian, can be attributed to shared genetic effects.
\runinhead{monozygotic} Identical twins, developing from the same fertilised egg (zygote) and genetically very similar.
\runinhead{$r$} Sample estimate of the Pearson correlation coefficient.
\runinhead{$\rho$} (rho) The Pearson correlation coefficient in the population.




\bibliography{malEps_bib2}   


\end{document}
\renewcommand{\addcontentsline}{\oldaddcontentsline}
\renewcommand{\section}{\oldsection}


%%Springer template
%%%%%%%%%%%%%%%%%%%%% appendix.tex %%%%%%%%%%%%%%%%%%%%%%%%%%%%%%%%%
%
% sample appendix
%
% Use this file as a template for your own input.
%
%%%%%%%%%%%%%%%%%%%%%%%% Springer-Verlag %%%%%%%%%%%%%%%%%%%%%%%%%%

\part*{Appendices}
\renewcommand\thechapter{\Alph{chapter}}
\markboth{Appendix \thechapter}{}

\setcounter{chapter}{0}
\setcounter{section}{0}
\addcontentsline{toc}{part}{Appendices}

\begin{appendix}
\chapter{Alternative approaches to correlation}
\label{ch:alt} % Always give a unique label
\markboth{Appendix \thechapter}{}

The scope of the present study was restricted to Pearson and Spearman correlations, however alternative approaches to estimating correlation should be considered and for completeness the following are reviewed below: Kendall's $\tau$; partial correlations; and intra-class correlations.  The latter two are of particular relevance to twin studies, however were beyond scope for inclusion for this research project.

\section{Kendall's $\tau$}
An alternate non-parametric option is Kendall's $\tau$ (tau) which provides a summary measure of correlation based on concordance of trend across the sample: pairs are concordant if the product of consecutive rank pair differences is $> 0$, and discordant if this product is $< 0$;  the number of concordant ($C$) and discordant ($D$) pairs are tallied, and the difference ($C � D$) is the score $S$;   Kendall's $\tau_a$ is calculated as $\frac{S}{n(n-1)/2}$, while other variant formulas include further adjustment to account for ties \cite{Fieller1957,StataCorp2013}.

\section{Partial correlation}
Adjustment for additional covariates, say $x_3$, may be introduced resulting in what is known as a partial correlation representing the linear relationship between $x_1$ and $x_2$ adjusting for the  effects of $x_3$ \cite{Fisher1990}. An elegant computational approach to calculating partial correlations drawing on the output of multiple regression analysis using standard software packages uses the test statistic of interest $t = \frac{b}{se}$, the sample size $n$, and number of covariates $k$ \cite{StataCorp2013}.  Using the example where we are interested in the partial correlation of $x_1$ (dependent variable in our regression model) and $x_2$ adjusting for the  effects of $u_3$, or $\rho_{x_1 x_2 \cdot x_3}$, where $k = 2$:

$$\rho_{x_1 x_2 \cdot x_3} = \frac{t_{x_{2}}}{\sqrt{t_{x_{2}}^2 + n - k}}$$

\section{Intra-class correlations}
The correlation approaches described above may be considered to be inter-class correlations: in the twin context, this amounts to looking at variable(s) pertaining to a set of (arbitrary) first members of MZ twins, and comparing with the remaining MZ twins.  This approach overlooks the paired nature of twin data.  In contrast, intra-class correlations (ICCs) draw upon within-pair pooled mean and standard deviation \cite{Fisher1990}.  If we refer to twin membership using the subscript $j$, and let $n$ be number of twin pairs in our sample:

\begin{align*}
\bar{x} &= \frac{1}{2n}\sum_{i=1}^{n} x_{ij}+x_{ij+1}  \\
s^2 &= \frac{1}{2n}\sum_{i=1}^{n} (x_{ij}-\bar{x})^2+(x_{ij+1}-\bar{x})^2  \\
r_{\text{ICC}} &= \frac{1}{ns^2}\sum_{i=1}^{n}  (x_{ij}-\bar{x})(x_{ij+1}-\bar{x})
\end{align*}

The above is considered a more accurate approach to correlation in contexts such as those of twin pairs, which do not have a natural ordering \cite{Fisher1990}.  A number of approaches to statistical modelling can be taken to account for the paired twin data structure \cite{Carlin2005}; a frequently used approach involves the calculation of ICCs for both MZ and DZ twin pairs through use of mixed effects modelling.  Calculated in this way,  $r_{\text{ICC}}$ represents the estimated ratio of between pair variation in a phenotype to total variation; the degree to which $r_{\text{ICC:MZ}}$ is larger than $r_{\text{ICC:DZ}}$ can be inferred to relate to the shared genetic basis for variation in the phenotype \cite{Barrett2008}.

\chapter{Annotated PubMed searches}
\markboth{Appendix \thechapter}{}
\label{ch:lit}
\section{"twin pearson difference" - 15 May 2018}
\markboth{Appendix \thesection}{}
\includepdf[landscape=true,scale=.9, pagecommand={\pagestyle{fancy}},pages=-]{sections/pubmed_result__twin_pearson_difference.pdf}


\chapter{Simulation tables}
\markboth{Appendix \thechapter}{}
\label{ch:simtab}
\section{Colour scale legend}
\includepdf[scale=.3, pagecommand={\pagestyle{fancy}},pages=-]{sections/sa3_colour_scales.pdf}
\section{Power estimates by test, correlation method, distribution, MZ:DZ ratio and sample size holding rho = {0.2,0.5}, using 10,000 simulations per scenario}
\markboth{Appendix \thesection}{}
\label{ch:sim_est}
% \includepdf[scale=.75, pagecommand={\pagestyle{fancy}},pages=-]{sections/sa3_1k_normal.pdf}
% \includepdf[scale=.75, pagecommand={\pagestyle{fancy}},pages=-]{sections/sa3_1k_gamma1.pdf}
% \includepdf[scale=.75, pagecommand={\pagestyle{fancy}},pages=-]{sections/sa3_1k_gamma2.pdf}
\includepdf[scale=.75, pagecommand={\pagestyle{fancy}},pages=-]{sections/sa3_10k_normal.pdf}
\includepdf[scale=.75, pagecommand={\pagestyle{fancy}},pages=-]{sections/sa3_10k_gamma2.pdf}
\includepdf[scale=.75, pagecommand={\pagestyle{fancy}},pages=-]{sections/sa3_10k_gamma1.pdf}

\section{Difference in power estimate using simulations (1,000 simulation estimates - 10,000 simulations estimates) by test, correlation method, distribution, MZ:DZ ratio and sample size holding at rho = {0.2,0.5}.  The 10,000 simulation results are reported in Appendix \ref{ch:sim_est}.  For conciseness, the corresponding results using 1,000 simulations have been omitted but can be gleaned from the following table, for the purposes of establishing validity of using 1,000 simulations.}
\markboth{Appendix \thesection}{}
\label{ch:sim_dif}
\includepdf[scale=.75, pagecommand={\pagestyle{fancy}},pages=-]{sections/sa3_diff_normal.pdf}
\includepdf[scale=.75, pagecommand={\pagestyle{fancy}},pages=-]{sections/sa3_diff_gamma1.pdf}
\includepdf[scale=.75, pagecommand={\pagestyle{fancy}},pages=-]{sections/sa3_diff_gamma2.pdf}


\chapter{GTV test, varying ratio}
\markboth{Appendix \thechapter}{}
\label{ch:contour_gtv}
%% Contour - GTV - vary ratio
\begin{figure}[htb]
  \centering        
  \minipage{0.49\textwidth}%
    \includegraphics[width=\linewidth]{{../figs/corrx_contour_sA1_n180_mzdz.5_gtv_s1000}.pdf}
  \endminipage\hfill 
  \minipage{0.49\textwidth}% 
    \includegraphics[width=\linewidth]{{../figs/corrx_contour_sA2_n180_mzdz1_gtv_s1000}.pdf}
  \endminipage\hfill
  \caption{Left to right, above: GTV test power given rho contour plots for bivariate normal distribution, using MZ:DZ group size ratio of (a) 60:120, and (b) 90:90}
  \label{fig:contour_gtv_norm}  
  
  \medskip
  \minipage{0.49\textwidth}%
    \includegraphics[width=\linewidth]{{../figs/corrx_contour_gamma1_sA1_n180_mzdz.5_gtv_s1000}.pdf}
  \endminipage\hfill 
  \minipage{0.49\textwidth}% 
    \includegraphics[width=\linewidth]{{../figs/corrx_contour_gamma1_sA2_n180_mzdz1_gtv_s1000}.pdf}
  \endminipage\hfill
  \caption{Left to right, above: GTV test power given rho contour plots for bivariate "mild skew" gamma distribution, using MZ:DZ group size ratio of (a) 60:120, and (b) 90:90}
  \label{fig:contour_gtv_gamma1}
  
  \medskip
  \minipage{0.49\textwidth}%
    \includegraphics[width=\linewidth]{{../figs/corrx_contour_gamma2_sA1_n180_mzdz.5_gtv_s1000}.pdf}
  \endminipage\hfill 
  \minipage{0.49\textwidth}% 
    \includegraphics[width=\linewidth]{{../figs/corrx_contour_gamma2_sA2_n180_mzdz1_gtv_s1000}.pdf}
  \endminipage\hfill  
  \caption{Left to right, above: GTV test power given rho contour plots for bivariate "extreme skew" gamma distribution, using MZ:DZ group size ratio of (a) 60:120, and (b) 90:90}
  \label{fig:contour_gtv_gamma2}
\end{figure} 

\chapter{SLR contour plot, varying ratio}
\markboth{Appendix \thechapter}{}
\label{ch:contour_slr}

%% Contour - SLR - vary ratio
\begin{figure}[htb]
  \centering        
  \minipage{0.49\textwidth}%
    \includegraphics[width=\linewidth]{{../figs/corrx_contour_sA1_n180_mzdz.5_slr_s1000}.pdf}
  \endminipage\hfill 
  \minipage{0.49\textwidth}% 
    \includegraphics[width=\linewidth]{{../figs/corrx_contour_sA2_n180_mzdz1_slr_s1000}.pdf}
  \endminipage\hfill
  \caption{Left to right, above: SLR test power given rho contour plots for bivariate normal distribution, using MZ:DZ group size ratio of (a) 60:120, and (b) 90:90}
  \label{fig:contour_slr_norm}  
  
  \medskip
  \minipage{0.49\textwidth}%
    \includegraphics[width=\linewidth]{{../figs/corrx_contour_gamma1_sA1_n180_mzdz.5_slr_s1000}.pdf}
  \endminipage\hfill 
  \minipage{0.49\textwidth}% 
    \includegraphics[width=\linewidth]{{../figs/corrx_contour_gamma1_sA2_n180_mzdz1_slr_s1000}.pdf}
  \endminipage\hfill
  \caption{Left to right, above: SLR test power given rho contour plots for bivariate "mild skew" gamma distribution, using MZ:DZ group size ratio of (a) 60:120, and (b) 90:90}
  \label{fig:contour_slr_gamma1}
  
  \medskip
  \minipage{0.49\textwidth}%
    \includegraphics[width=\linewidth]{{../figs/corrx_contour_gamma2_sA1_n180_mzdz.5_slr_s1000}.pdf}
  \endminipage\hfill 
  \minipage{0.49\textwidth}% 
    \includegraphics[width=\linewidth]{{../figs/corrx_contour_gamma2_sA2_n180_mzdz1_slr_s1000}.pdf}
  \endminipage\hfill  
  \caption{Left to right, above: SLR test power given rho contour plots for bivariate "extreme skew" gamma distribution, using MZ:DZ group size ratio of (a) 60:120, and (b) 90:90}
  \label{fig:contour_slr_gamma2}
\end{figure} 



Code
\chapter{Code}
\markboth{Appendix \thechapter}{}
\label{ch:aDo}
\section{Bitbucket repository containing R scripts, project documentation and associated links.  Follow urls on following page to view code and version history, etc.}
\markboth{Appendix \thesection}{}
\includepdf[scale=0.85, pagecommand={\pagestyle{fancy}},pages=-]{sections/carlhiggs_bca_rp2_scripts_Bitbucket.pdf}


% Code
% \chapter{R scripts}
% \markboth{Appendix \thechapter}{}
% \label{ch:aDo}
% \section{Simulation}
% \markboth{Appendix \thesection}{}
% \includepdf[scale=0.75, pagecommand={\pagestyle{fancy}},pages=-]{sections/scripts/corr_power_CH.pdf}
% \section{Time testing}
% \label{sec:aDoTime}
% \thispagestyle{empty}
% \markboth{Appendix \thesection}{}
% \includepdf[scale=0.75, pagecommand={\pagestyle{fancy}},pages=-]{sections/scripts/corr_timetest.pdf}
% \section{Power plots}
% \label{sec:aDoPlots}
% \thispagestyle{empty}
% \markboth{Appendix \thesection}{}
% \includepdf[scale=0.75, pagecommand={\pagestyle{fancy}},pages=-]{sections/scripts/corr_power_plots.pdf}
% \section{Shiny app server}
% \label{sec:rshinyserver}
% \thispagestyle{empty}
% \markboth{Appendix \thesection}{}
% \includepdf[scale=0.75, pagecommand={\pagestyle{fancy}},pages=-]{sections/scripts/server.pdf}
% \section{Shiny app UI}
% \label{sec:rshinyui}
% \thispagestyle{empty}
% \markboth{Appendix \thesection}{}
% \includepdf[scale=0.75, pagecommand={\pagestyle{fancy}},pages=-]{sections/scripts/ui.pdf}
\end{appendix}
			

\printindex

%%%%%%%%%%%%%%%%%%%%%%%%%%%%%%%%%%%%%%%%%%%%%%%%%%%%%%%%%%%%%%%%%%%%%%

\end{document}